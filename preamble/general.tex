% preamble/general.tex
% ------------------------------------------------------------------
% General Preamble Settings
%
% This file loads general configuration files for your document.
% It includes meta information and ETH corporate colour definitions.
%
% Files referenced:
%   - preamble/meta.tex  : Contains meta data (titles, author info, etc.)
%   - preamble/eth.tex   : Contains ETH corporate colour definitions
%
% Include this file in your main document preamble:
%   % preamble/general.tex
% ------------------------------------------------------------------
% General Preamble Settings
%
% This file loads general configuration files for your document.
% It includes meta information and ETH corporate colour definitions.
%
% Files referenced:
%   - preamble/meta.tex  : Contains meta data (titles, author info, etc.)
%   - preamble/eth.tex   : Contains ETH corporate colour definitions
%
% Include this file in your main document preamble:
%   % preamble/general.tex
% ------------------------------------------------------------------
% General Preamble Settings
%
% This file loads general configuration files for your document.
% It includes meta information and ETH corporate colour definitions.
%
% Files referenced:
%   - preamble/meta.tex  : Contains meta data (titles, author info, etc.)
%   - preamble/eth.tex   : Contains ETH corporate colour definitions
%
% Include this file in your main document preamble:
%   % preamble/general.tex
% ------------------------------------------------------------------
% General Preamble Settings
%
% This file loads general configuration files for your document.
% It includes meta information and ETH corporate colour definitions.
%
% Files referenced:
%   - preamble/meta.tex  : Contains meta data (titles, author info, etc.)
%   - preamble/eth.tex   : Contains ETH corporate colour definitions
%
% Include this file in your main document preamble:
%   \input{preamble/general.tex}
% ------------------------------------------------------------------

% ****************************************************************************************************
% 1. Configure classicthesis for your needs here, e.g., remove "drafting" below
% in order to deactivate the time-stamp on the pages
% (see ClassicThesis.pdf for more information):
% ****************************************************************************************************
\PassOptionsToPackage{
  drafting=true,    % print version information on the bottom of the pages
  tocaligned=false, % the left column of the toc will be aligned (no indentation)
  dottedtoc=false,  % page numbers in ToC flushed right
  eulerchapternumbers=false, % use AMS Euler for chapter font (otherwise Palatino)
  floatperchapter=true,     % numbering per chapter for all floats (i.e., Figure 1.1)
  eulermath=false,  % use awesome Euler fonts for mathematical formulae (only with pdfLaTeX)
  beramono=true,    % toggle a nice monospaced font (w/ bold)
  palatino=true,    % deactivate standard font for loading another one, see the last section at the end of this file for suggestions
  %linedheaders=true, % obsolete / available for backwards compatibility
  style=classicthesis % classicthesis, arsclassica, linedheaders, plain
}{classicthesis}

% ****************************************************************************************************
% 2. Personal data and user ad-hoc commands (insert your own data here)
% ****************************************************************************************************



% \input{preamble/eth.tex}  % Load ETH corporate colours and shade definitions
\input{preamble/eth.tex}  % Load ETH corporate colours and shade definitions

\colorlet{CTurl}{ETHBlue}      % Override CTcitation with ETHBlue
\colorlet{CTtitle}{ETHBlue}      % Override CTcitation with ETHBlue


% Additional general configurations (packages, macros, etc.) can be added below.


% Biblatex
\input{preamble/biblatex}


\ExplSyntaxOn
% Define a function for string substitution
\cs_new:Npn \minna_replace:nn #1 #2
  {
    \tl_replace_all:Nnn \l_tmpa_tl {#1} {#2}
  }

% Wrapper macro for ease of use
\NewDocumentCommand{\ReplaceString}{ m m m }
  {
    \tl_set:Nn \l_tmpa_tl {#1}
    \minna_replace:nn {#2} {#3}
    \tl_use:N \l_tmpa_tl
  }
\ExplSyntaxOff



% ********************************************************************
% Fine-tune hyperreferences (hyperref should be called last)
% ********************************************************************

\usepackage[dvipsnames]{xcolor}


\PassOptionsToPackage{pdftex,hyperfootnotes=false,pdfpagelabels}{hyperref}
\usepackage{hyperref}  % backref linktocpage pagebackref
\pdfcompresslevel=9
\pdfadjustspacing=1

% \usepackage{hyperxmp}
%\pdfcompresslevel=9
%\pdfadjustspacing=1

% \usepackage{hyperref}  % backref linktocpage pagebackref

\hypersetup{%
  %draft, % hyperref's draft mode, for printing see below
  colorlinks=true, linktocpage=true, pdfstartpage=3, pdfstartview=FitV,%
  % uncomment the following line if you want to have black links (e.g., for printing)
  %colorlinks=false, linktocpage=false, pdfstartpage=3, pdfstartview=FitV, pdfborder={0 0 0},%
  breaklinks=true, pageanchor=true,%
  pdfpagemode=UseNone, %
  % pdfpagemode=UseOutlines,%
  plainpages=false, bookmarksnumbered, bookmarksopen=true, bookmarksopenlevel=1,%
  hypertexnames=true, pdfhighlight=/O,%nesting=true,%frenchlinks,%
  urlcolor=CTurl, linkcolor=CTlink, citecolor=CTcitation, %pagecolor=RoyalBlue,%
  %urlcolor=Black, linkcolor=Black, citecolor=Black, %pagecolor=Black,%
  pdftitle={\myTitle},%
  pdfauthor={\textcopyright\ \myName, \myUni, \myFaculty},%
  pdfsubject={},%
  pdfkeywords={},%
  pdfcreator={pdfLaTeX},%
  pdfproducer={LaTeX with hyperref and classicthesis}%
}


% ********************************************************************
% Setup autoreferences (hyperref and babel)
% ********************************************************************
% There are some issues regarding autorefnames
% http://www.tex.ac.uk/cgi-bin/texfaq2html?label=latexwords
% you have to redefine the macros for the
% language you use, e.g., american, ngerman
% (as chosen when loading babel/AtBeginDocument)
% ********************************************************************
 \makeatletter
 \@ifpackageloaded{babel}%
   {%
     \addto\extrasamerican{%
       \renewcommand*{\figureautorefname}{Figure}%
       \renewcommand*{\tableautorefname}{Table}%
       \renewcommand*{\partautorefname}{Part}%
       \renewcommand*{\chapterautorefname}{Chapter}%
       \renewcommand*{\sectionautorefname}{Section}%
       \renewcommand*{\subsectionautorefname}{Section}%
       \renewcommand*{\subsubsectionautorefname}{Section}%
     }%
     \addto\extrasngerman{%
       \renewcommand*{\paragraphautorefname}{Absatz}%
       \renewcommand*{\subparagraphautorefname}{Unterabsatz}%
       \renewcommand*{\footnoteautorefname}{Fu\"snote}%
       \renewcommand*{\FancyVerbLineautorefname}{Zeile}%
       \renewcommand*{\theoremautorefname}{Theorem}%
       \renewcommand*{\appendixautorefname}{Anhang}%
       \renewcommand*{\equationautorefname}{Gleichung}%
       \renewcommand*{\itemautorefname}{Punkt}%
     }%
       % Fix to getting autorefs for subfigures right (thanks to Belinda Vogt for changing the definition)
       \providecommand{\subfigureautorefname}{\figureautorefname}%
     }{\relax}
 \makeatother

% (Better) alternative to \autoref is \cref via the cleveref package
%\usepackage{cleveref}
%\crefformat{part}{Part #2\MakeUppercase{#1}#3}
% ------------------------------------------------------------------

% ****************************************************************************************************
% 1. Configure classicthesis for your needs here, e.g., remove "drafting" below
% in order to deactivate the time-stamp on the pages
% (see ClassicThesis.pdf for more information):
% ****************************************************************************************************
\PassOptionsToPackage{
  drafting=true,    % print version information on the bottom of the pages
  tocaligned=false, % the left column of the toc will be aligned (no indentation)
  dottedtoc=false,  % page numbers in ToC flushed right
  eulerchapternumbers=false, % use AMS Euler for chapter font (otherwise Palatino)
  floatperchapter=true,     % numbering per chapter for all floats (i.e., Figure 1.1)
  eulermath=false,  % use awesome Euler fonts for mathematical formulae (only with pdfLaTeX)
  beramono=true,    % toggle a nice monospaced font (w/ bold)
  palatino=true,    % deactivate standard font for loading another one, see the last section at the end of this file for suggestions
  %linedheaders=true, % obsolete / available for backwards compatibility
  style=classicthesis % classicthesis, arsclassica, linedheaders, plain
}{classicthesis}

% ****************************************************************************************************
% 2. Personal data and user ad-hoc commands (insert your own data here)
% ****************************************************************************************************



% % eth.tex
% Defines ETH brand colors based on:
% https://ethz.ch/staffnet/en/service/communication/corporate-design/colours.html
% 
% ------------------------------------------------------------------
% ETH Corporate Design – Primary Colors and Colour Shades Definitions
%
% PRIMARY ETH CORPORATE COLORS
%
% Colour       RGB             HEX       CMYK                 Pantone    RAL
% ----------------------------------------------------------------------------
% ETH Blue     33, 92, 175     #215CAF   100,57,0,0           2935       5005 Signalblau
% ETH Petrol   0, 120, 148     #007894   100,25,30,10         633        5009 Azurblau
% ETH Green    98, 115, 19     #627313   55,10,100,30         364        6010 Grasgrün
% ETH Bronze   142, 103, 19    #8E6713   30,36,100,25         4495       7008 Khakigrau
% ETH Red      183, 53, 45     #B7352D   0,90,80,17           1797       3031 Orientrot
% ETH Purple   167, 17, 122    #A7117A   22,100,0,10          234        4006 Verkehrspurpur
% ETH Grey     111, 111, 111   #6F6F6F   0,0,0,70             Cool Gray 11   7046 Telegrau 2
% ------------------------------------------------------------------
% ETH Corporate Design – Colour Shades Definitions
%
% 1. ETH Blue Shades
% 2. ETH Petrol Shades
% 3. ETH Green Shades
% 4. ETH Bronze Shades
% 5. ETH Red Shades
% 6. ETH Purple Shades
% 7. ETH Grey Shades
% 
% Last updated: 2025-03-08
%
% Usage:
%   \input{eth.tex}
%   \textcolor{ETHBlue}{Hello ETH!}
% Usage in your main .tex:
%   \usepackage{xcolor} % or colortbl, etc.
%   \input{eth.tex}
%   \textcolor{ETHBlue}{Hello from ETH!}
\NeedsTeXFormat{LaTeX2e}
\ProvidesFile{eth.tex}[2025/03/08 v1.0 ETH brand color definitions]

\RequirePackage{xcolor}

% ==============================================================
% Primary ETH Corporate Colors
% ==============================================================
\definecolor{ETHBlue}{HTML}{215CAF}    % ETH Blue: RGB: 33, 92, 175; CMYK: 100,57,0,0; Pantone: 2935; RAL: 5005 Signalblau
\definecolor{ETHPetrol}{HTML}{007894}   % ETH Petrol: RGB: 0,120,148; CMYK: 100,25,30,10; Pantone: 633; RAL: 5009 Azurblau
\definecolor{ETHGreen}{HTML}{627313}    % ETH Green: RGB: 98,115,19; CMYK: 55,10,100,30; Pantone: 364; RAL: 6010 Grasgrün
\definecolor{ETHBronze}{HTML}{8E6713}    % ETH Bronze: RGB: 142,103,19; CMYK: 30,36,100,25; Pantone: 4495; RAL: 7008 Khakigrau
\definecolor{ETHRed}{HTML}{B7352D}       % ETH Red: RGB: 183,53,45; CMYK: 0,90,80,17; Pantone: 1797; RAL: 3031 Orientrot
\definecolor{ETHPurple}{HTML}{A7117A}     % ETH Purple: RGB: 167,17,122; CMYK: 22,100,0,10; Pantone: 234; RAL: 4006 Verkehrspurpur
\definecolor{ETHGrey}{HTML}{6F6F6F}       % ETH Grey: RGB: 111,111,111; CMYK: 0,0,0,70; Pantone: Cool Gray 11; RAL: 7046 Telegrau 2


% ------------------------------------------------------------------
% Extended / Complementary Color Palette
% ------------------------------------------------------------------
\definecolor{ETHTeal}{HTML}{008C95}
\definecolor{ETHGreen}{HTML}{00B38B}
\definecolor{ETHDarkBlue}{HTML}{1D2447}
\definecolor{ETHLightBlue}{HTML}{5BB6D6}
\definecolor{ETHOrange}{HTML}{F39200}
\definecolor{ETHRed}{HTML}{C8002A}
\definecolor{ETHWarmGray}{HTML}{DAD7D2}
\definecolor{ETHBeige}{HTML}{D7CEC1}
\definecolor{ETHDarkBrown}{HTML}{7F4F3C}
\definecolor{ETHDarkPink}{HTML}{EB67BD}
\definecolor{ETHDarkPurple}{HTML}{5F2167}
\definecolor{ETHDarkMagenta}{HTML}{A3488E}
\definecolor{ETHDarkGray}{HTML}{333333}
\definecolor{ETHGray}{HTML}{75787B}
\definecolor{ETHLightGray}{HTML}{E2E2E2}
\definecolor{ETHWhite}{HTML}{FFFFFF}
\definecolor{ETHBlack}{HTML}{000000}

% ==============================================================
% 1. ETH Blue Shades
% ==============================================================
% Shade | RGB             | HEX      | CMYK
% ------|-----------------|----------|---------------
% 10%   | 233, 239, 247   | E9EFF7   | 10, 6, 0, 0
\definecolor{ETHBlue10}{HTML}{E9EFF7}
% 20%   | 211, 222, 239   | D3DEEF   | 20, 11, 0, 0
\definecolor{ETHBlue20}{HTML}{D3DEEF}
% 40%   | 166, 190, 223   | A6BEDF   | 40, 23, 0, 0
\definecolor{ETHBlue40}{HTML}{A6BEDF}
% 60%   | 122, 157, 207   | 7A9DCF   | 60, 34, 0, 0
\definecolor{ETHBlue60}{HTML}{7A9DCF}
% 80%   | 77, 125, 191    | 4D7DBF   | 80, 46, 0, 0
\definecolor{ETHBlue80}{HTML}{4D7DBF}
% 120%  | 8, 64, 126      | 08407E   | 100, 62, 0, 30
\definecolor{ETHBlue120}{HTML}{08407E}

% ==============================================================
% 2. ETH Petrol Shades
% ==============================================================
% Shade | RGB             | HEX      | CMYK
% ------|-----------------|----------|---------------
% 10%   | 231, 244, 247   | E7F4F7   | 12, 0, 5, 0
\definecolor{ETHPetrol10}{HTML}{E7F4F7}
% 20%   | 204, 228, 234   | CCE4EA   | 20, 3, 7, 0
\definecolor{ETHPetrol20}{HTML}{CCE4EA}
% 40%   | 153, 202, 213   | 99CAD5   | 40, 7, 12, 4
\definecolor{ETHPetrol40}{HTML}{99CAD5}
% 60%   | 102, 175, 192   | 66AFC0   | 60, 14, 18, 6
\definecolor{ETHPetrol60}{HTML}{66AFC0}
% 80%   | 51, 149, 171    | 3395AB   | 80, 20, 24, 8
\definecolor{ETHPetrol80}{HTML}{3395AB}
% 120%  | 0, 89, 109      | 00596D   | 100, 25, 30, 38
\definecolor{ETHPetrol120}{HTML}{00596D}

% ==============================================================
% 3. ETH Green Shades
% ==============================================================
% Shade | RGB             | HEX      | CMYK
% ------|-----------------|----------|---------------
% 10%   | 239, 241, 231   | EEF1E7   | 6, 1, 10, 3
\definecolor{ETHGreen10}{HTML}{EEF1E7}
% 20%   | 224, 227, 208   | E0E3D0   | 11, 2, 20, 6
\definecolor{ETHGreen20}{HTML}{E0E3D0}
% 40%   | 192, 199, 161   | C0C7A1   | 22, 4, 40, 12
\definecolor{ETHGreen40}{HTML}{C0C7A1}
% 60%   | 161, 171, 113   | A1AB71   | 33, 6, 60, 18
\definecolor{ETHGreen60}{HTML}{A1AB71}
% 80%   | 129, 143, 66    | 818F42   | 44, 8, 80, 24
\definecolor{ETHGreen80}{HTML}{818F42}
% 120%  | 54, 82, 19      | 365213   | 55, 10, 100, 65
\definecolor{ETHGreen120}{HTML}{365213}

% ==============================================================
% 4. ETH Bronze Shades
% ==============================================================
% Shade | RGB             | HEX      | CMYK
% ------|-----------------|----------|---------------
% 10%   | 244, 240, 231   | F4F0E7   | 3, 4, 10, 3
\definecolor{ETHBronze10}{HTML}{F4F0E7}
% 20%   | 232, 225, 208   | E8E1D0   | 6, 7, 20, 5
\definecolor{ETHBronze20}{HTML}{E8E1D0}
% 40%   | 210, 194, 161   | D2C2A1   | 12, 14, 40, 10
\definecolor{ETHBronze40}{HTML}{D2C2A1}
% 60%   | 187, 164, 113   | BBA471   | 18, 22, 60, 15
\definecolor{ETHBronze60}{HTML}{BBA471}
% 80%   | 165, 133, 66    | A58542   | 24, 29, 80, 20
\definecolor{ETHBronze80}{HTML}{A58542}
% 120%  | 112, 79, 18     | 704F12   | 30, 36, 100, 55
\definecolor{ETHBronze120}{HTML}{704F12}

% ==============================================================
% 5. ETH Red Shades
% ==============================================================
% Shade | RGB             | HEX      | CMYK
% ------|-----------------|----------|---------------
% 10%   | 248, 235, 234   | F8EBEA   | 0, 9, 6, 0
\definecolor{ETHRed10}{HTML}{F8EBEA}
% 20%   | 241, 215, 213   | F1D7D5   | 0, 18, 13, 4
\definecolor{ETHRed20}{HTML}{F1D7D5}
% 40%   | 226, 174, 171   | E2AEAB   | 0, 36, 26, 8
\definecolor{ETHRed40}{HTML}{E2AEAB}
% 60%   | 212, 134, 129   | D48681   | 0, 54, 39, 11
\definecolor{ETHRed60}{HTML}{D48681}
% 80%   | 197, 93, 87     | C55D57   | (using HEX)
\definecolor{ETHRed80}{HTML}{C55D57}
% 120%  | 150, 39, 45     | 96272D   | 0, 100, 80, 40
\definecolor{ETHRed120}{HTML}{96272D}

% ==============================================================
% 6. ETH Purple Shades
% ==============================================================
% Shade | RGB             | HEX      | CMYK
% ------|-----------------|----------|---------------
% 10%   | 248, 232, 243   | F8E8F3   | 2, 10, 0, 1
\definecolor{ETHPurple10}{HTML}{F8E8F3}
% 20%   | 239, 208, 227   | EFD0E3   | 4, 20, 0, 1
\definecolor{ETHPurple20}{HTML}{EFD0E3}
% 40%   | 220, 158, 201   | DC9EC9   | 7, 40, 0, 4
\definecolor{ETHPurple40}{HTML}{DC9EC9}
% 60%   | 202, 108, 174   | CA6CAE   | 13, 60, 0, 6
\definecolor{ETHPurple60}{HTML}{CA6CAE}
% 80%   | 183, 59, 146    | B73B92   | 18, 80, 0, 8
\definecolor{ETHPurple80}{HTML}{B73B92}
% 120%  | 140, 10, 89     | 8C0A59   | 22, 100, 0, 35
\definecolor{ETHPurple120}{HTML}{8C0A59}

% ==============================================================
% 7. ETH Grey Shades
% ==============================================================
% Shade | RGB             | HEX      | CMYK
% ------|-----------------|----------|---------------
% 10%   | 241, 241, 241   | F1F1F1   | 0, 0, 0, 7
\definecolor{ETHGrey10}{HTML}{F1F1F1}
% 20%   | 226, 226, 226   | E2E2E2   | 0, 0, 0, 14
\definecolor{ETHGrey20}{HTML}{E2E2E2}
% 40%   | 197, 197, 197   | C5C5C5   | 0, 0, 0, 28
\definecolor{ETHGrey40}{HTML}{C5C5C5}
% 60%   | 169, 169, 169   | A9A9A9   | 0, 0, 0, 42
\definecolor{ETHGrey60}{HTML}{A9A9A9}
% 80%   | 140, 140, 140   | 8C8C8C   | 0, 0, 0, 56
\definecolor{ETHGrey80}{HTML}{8C8C8C}
% 120%  | 87, 87, 87      | 575757   | 0, 0, 0, 81
\definecolor{ETHGrey120}{HTML}{575757}  % Load ETH corporate colours and shade definitions
% eth.tex
% Defines ETH brand colors based on:
% https://ethz.ch/staffnet/en/service/communication/corporate-design/colours.html
% 
% ------------------------------------------------------------------
% ETH Corporate Design – Primary Colors and Colour Shades Definitions
%
% PRIMARY ETH CORPORATE COLORS
%
% Colour       RGB             HEX       CMYK                 Pantone    RAL
% ----------------------------------------------------------------------------
% ETH Blue     33, 92, 175     #215CAF   100,57,0,0           2935       5005 Signalblau
% ETH Petrol   0, 120, 148     #007894   100,25,30,10         633        5009 Azurblau
% ETH Green    98, 115, 19     #627313   55,10,100,30         364        6010 Grasgrün
% ETH Bronze   142, 103, 19    #8E6713   30,36,100,25         4495       7008 Khakigrau
% ETH Red      183, 53, 45     #B7352D   0,90,80,17           1797       3031 Orientrot
% ETH Purple   167, 17, 122    #A7117A   22,100,0,10          234        4006 Verkehrspurpur
% ETH Grey     111, 111, 111   #6F6F6F   0,0,0,70             Cool Gray 11   7046 Telegrau 2
% ------------------------------------------------------------------
% ETH Corporate Design – Colour Shades Definitions
%
% 1. ETH Blue Shades
% 2. ETH Petrol Shades
% 3. ETH Green Shades
% 4. ETH Bronze Shades
% 5. ETH Red Shades
% 6. ETH Purple Shades
% 7. ETH Grey Shades
% 
% Last updated: 2025-03-08
%
% Usage:
%   \input{eth.tex}
%   \textcolor{ETHBlue}{Hello ETH!}
% Usage in your main .tex:
%   \usepackage{xcolor} % or colortbl, etc.
%   \input{eth.tex}
%   \textcolor{ETHBlue}{Hello from ETH!}
\NeedsTeXFormat{LaTeX2e}
\ProvidesFile{eth.tex}[2025/03/08 v1.0 ETH brand color definitions]

\RequirePackage{xcolor}

% ==============================================================
% Primary ETH Corporate Colors
% ==============================================================
\definecolor{ETHBlue}{HTML}{215CAF}    % ETH Blue: RGB: 33, 92, 175; CMYK: 100,57,0,0; Pantone: 2935; RAL: 5005 Signalblau
\definecolor{ETHPetrol}{HTML}{007894}   % ETH Petrol: RGB: 0,120,148; CMYK: 100,25,30,10; Pantone: 633; RAL: 5009 Azurblau
\definecolor{ETHGreen}{HTML}{627313}    % ETH Green: RGB: 98,115,19; CMYK: 55,10,100,30; Pantone: 364; RAL: 6010 Grasgrün
\definecolor{ETHBronze}{HTML}{8E6713}    % ETH Bronze: RGB: 142,103,19; CMYK: 30,36,100,25; Pantone: 4495; RAL: 7008 Khakigrau
\definecolor{ETHRed}{HTML}{B7352D}       % ETH Red: RGB: 183,53,45; CMYK: 0,90,80,17; Pantone: 1797; RAL: 3031 Orientrot
\definecolor{ETHPurple}{HTML}{A7117A}     % ETH Purple: RGB: 167,17,122; CMYK: 22,100,0,10; Pantone: 234; RAL: 4006 Verkehrspurpur
\definecolor{ETHGrey}{HTML}{6F6F6F}       % ETH Grey: RGB: 111,111,111; CMYK: 0,0,0,70; Pantone: Cool Gray 11; RAL: 7046 Telegrau 2


% ------------------------------------------------------------------
% Extended / Complementary Color Palette
% ------------------------------------------------------------------
\definecolor{ETHTeal}{HTML}{008C95}
\definecolor{ETHGreen}{HTML}{00B38B}
\definecolor{ETHDarkBlue}{HTML}{1D2447}
\definecolor{ETHLightBlue}{HTML}{5BB6D6}
\definecolor{ETHOrange}{HTML}{F39200}
\definecolor{ETHRed}{HTML}{C8002A}
\definecolor{ETHWarmGray}{HTML}{DAD7D2}
\definecolor{ETHBeige}{HTML}{D7CEC1}
\definecolor{ETHDarkBrown}{HTML}{7F4F3C}
\definecolor{ETHDarkPink}{HTML}{EB67BD}
\definecolor{ETHDarkPurple}{HTML}{5F2167}
\definecolor{ETHDarkMagenta}{HTML}{A3488E}
\definecolor{ETHDarkGray}{HTML}{333333}
\definecolor{ETHGray}{HTML}{75787B}
\definecolor{ETHLightGray}{HTML}{E2E2E2}
\definecolor{ETHWhite}{HTML}{FFFFFF}
\definecolor{ETHBlack}{HTML}{000000}

% ==============================================================
% 1. ETH Blue Shades
% ==============================================================
% Shade | RGB             | HEX      | CMYK
% ------|-----------------|----------|---------------
% 10%   | 233, 239, 247   | E9EFF7   | 10, 6, 0, 0
\definecolor{ETHBlue10}{HTML}{E9EFF7}
% 20%   | 211, 222, 239   | D3DEEF   | 20, 11, 0, 0
\definecolor{ETHBlue20}{HTML}{D3DEEF}
% 40%   | 166, 190, 223   | A6BEDF   | 40, 23, 0, 0
\definecolor{ETHBlue40}{HTML}{A6BEDF}
% 60%   | 122, 157, 207   | 7A9DCF   | 60, 34, 0, 0
\definecolor{ETHBlue60}{HTML}{7A9DCF}
% 80%   | 77, 125, 191    | 4D7DBF   | 80, 46, 0, 0
\definecolor{ETHBlue80}{HTML}{4D7DBF}
% 120%  | 8, 64, 126      | 08407E   | 100, 62, 0, 30
\definecolor{ETHBlue120}{HTML}{08407E}

% ==============================================================
% 2. ETH Petrol Shades
% ==============================================================
% Shade | RGB             | HEX      | CMYK
% ------|-----------------|----------|---------------
% 10%   | 231, 244, 247   | E7F4F7   | 12, 0, 5, 0
\definecolor{ETHPetrol10}{HTML}{E7F4F7}
% 20%   | 204, 228, 234   | CCE4EA   | 20, 3, 7, 0
\definecolor{ETHPetrol20}{HTML}{CCE4EA}
% 40%   | 153, 202, 213   | 99CAD5   | 40, 7, 12, 4
\definecolor{ETHPetrol40}{HTML}{99CAD5}
% 60%   | 102, 175, 192   | 66AFC0   | 60, 14, 18, 6
\definecolor{ETHPetrol60}{HTML}{66AFC0}
% 80%   | 51, 149, 171    | 3395AB   | 80, 20, 24, 8
\definecolor{ETHPetrol80}{HTML}{3395AB}
% 120%  | 0, 89, 109      | 00596D   | 100, 25, 30, 38
\definecolor{ETHPetrol120}{HTML}{00596D}

% ==============================================================
% 3. ETH Green Shades
% ==============================================================
% Shade | RGB             | HEX      | CMYK
% ------|-----------------|----------|---------------
% 10%   | 239, 241, 231   | EEF1E7   | 6, 1, 10, 3
\definecolor{ETHGreen10}{HTML}{EEF1E7}
% 20%   | 224, 227, 208   | E0E3D0   | 11, 2, 20, 6
\definecolor{ETHGreen20}{HTML}{E0E3D0}
% 40%   | 192, 199, 161   | C0C7A1   | 22, 4, 40, 12
\definecolor{ETHGreen40}{HTML}{C0C7A1}
% 60%   | 161, 171, 113   | A1AB71   | 33, 6, 60, 18
\definecolor{ETHGreen60}{HTML}{A1AB71}
% 80%   | 129, 143, 66    | 818F42   | 44, 8, 80, 24
\definecolor{ETHGreen80}{HTML}{818F42}
% 120%  | 54, 82, 19      | 365213   | 55, 10, 100, 65
\definecolor{ETHGreen120}{HTML}{365213}

% ==============================================================
% 4. ETH Bronze Shades
% ==============================================================
% Shade | RGB             | HEX      | CMYK
% ------|-----------------|----------|---------------
% 10%   | 244, 240, 231   | F4F0E7   | 3, 4, 10, 3
\definecolor{ETHBronze10}{HTML}{F4F0E7}
% 20%   | 232, 225, 208   | E8E1D0   | 6, 7, 20, 5
\definecolor{ETHBronze20}{HTML}{E8E1D0}
% 40%   | 210, 194, 161   | D2C2A1   | 12, 14, 40, 10
\definecolor{ETHBronze40}{HTML}{D2C2A1}
% 60%   | 187, 164, 113   | BBA471   | 18, 22, 60, 15
\definecolor{ETHBronze60}{HTML}{BBA471}
% 80%   | 165, 133, 66    | A58542   | 24, 29, 80, 20
\definecolor{ETHBronze80}{HTML}{A58542}
% 120%  | 112, 79, 18     | 704F12   | 30, 36, 100, 55
\definecolor{ETHBronze120}{HTML}{704F12}

% ==============================================================
% 5. ETH Red Shades
% ==============================================================
% Shade | RGB             | HEX      | CMYK
% ------|-----------------|----------|---------------
% 10%   | 248, 235, 234   | F8EBEA   | 0, 9, 6, 0
\definecolor{ETHRed10}{HTML}{F8EBEA}
% 20%   | 241, 215, 213   | F1D7D5   | 0, 18, 13, 4
\definecolor{ETHRed20}{HTML}{F1D7D5}
% 40%   | 226, 174, 171   | E2AEAB   | 0, 36, 26, 8
\definecolor{ETHRed40}{HTML}{E2AEAB}
% 60%   | 212, 134, 129   | D48681   | 0, 54, 39, 11
\definecolor{ETHRed60}{HTML}{D48681}
% 80%   | 197, 93, 87     | C55D57   | (using HEX)
\definecolor{ETHRed80}{HTML}{C55D57}
% 120%  | 150, 39, 45     | 96272D   | 0, 100, 80, 40
\definecolor{ETHRed120}{HTML}{96272D}

% ==============================================================
% 6. ETH Purple Shades
% ==============================================================
% Shade | RGB             | HEX      | CMYK
% ------|-----------------|----------|---------------
% 10%   | 248, 232, 243   | F8E8F3   | 2, 10, 0, 1
\definecolor{ETHPurple10}{HTML}{F8E8F3}
% 20%   | 239, 208, 227   | EFD0E3   | 4, 20, 0, 1
\definecolor{ETHPurple20}{HTML}{EFD0E3}
% 40%   | 220, 158, 201   | DC9EC9   | 7, 40, 0, 4
\definecolor{ETHPurple40}{HTML}{DC9EC9}
% 60%   | 202, 108, 174   | CA6CAE   | 13, 60, 0, 6
\definecolor{ETHPurple60}{HTML}{CA6CAE}
% 80%   | 183, 59, 146    | B73B92   | 18, 80, 0, 8
\definecolor{ETHPurple80}{HTML}{B73B92}
% 120%  | 140, 10, 89     | 8C0A59   | 22, 100, 0, 35
\definecolor{ETHPurple120}{HTML}{8C0A59}

% ==============================================================
% 7. ETH Grey Shades
% ==============================================================
% Shade | RGB             | HEX      | CMYK
% ------|-----------------|----------|---------------
% 10%   | 241, 241, 241   | F1F1F1   | 0, 0, 0, 7
\definecolor{ETHGrey10}{HTML}{F1F1F1}
% 20%   | 226, 226, 226   | E2E2E2   | 0, 0, 0, 14
\definecolor{ETHGrey20}{HTML}{E2E2E2}
% 40%   | 197, 197, 197   | C5C5C5   | 0, 0, 0, 28
\definecolor{ETHGrey40}{HTML}{C5C5C5}
% 60%   | 169, 169, 169   | A9A9A9   | 0, 0, 0, 42
\definecolor{ETHGrey60}{HTML}{A9A9A9}
% 80%   | 140, 140, 140   | 8C8C8C   | 0, 0, 0, 56
\definecolor{ETHGrey80}{HTML}{8C8C8C}
% 120%  | 87, 87, 87      | 575757   | 0, 0, 0, 81
\definecolor{ETHGrey120}{HTML}{575757}  % Load ETH corporate colours and shade definitions

\colorlet{CTurl}{ETHBlue}      % Override CTcitation with ETHBlue
\colorlet{CTtitle}{ETHBlue}      % Override CTcitation with ETHBlue


% Additional general configurations (packages, macros, etc.) can be added below.


% Biblatex
% \usepackage[
%   style=nature,%
%   %style=science, article-title=true,%
%   natbib=true,%
%   clearlang=true,%
%   backend=biber,%
% ]{biblatex}

% Add this line to suppress the split bibliography warning
\BiblatexSplitbibDefernumbersWarningOff

% https://mirrors.ibiblio.org/CTAN/macros/latex/contrib/biblatex/doc/biblatex.pdf
\ExecuteBibliographyOptions{%
  %--- Backend --- --- ---
  bibwarn=true, %
  bibencoding=auto, % (ascii, inputenc, <encoding>)
  %--- Sorting --- --- ---
  sorting=none, % (bib, los) The sorting order of the list of shorthands =nty, ntd, nyt, ndt, nyvt, ndvt, anyt, andt, anyvt, optandvt, ynt, dnt, ydnt, ddnt, none, count, debug,
  % other options: 
  % nty        Sort by name, title, year.
  % nyt        Sort by name, year, title.
  % nyvt       Sort by name, year, volume, title.
  % anyt       Sort by alphabetic label, name, year, title.
  % anyvt      Sort by alphabetic label, name, year, volume, title.
  % ynt        Sort by year, name, title.
  % ydnt       Sort by year (descending), name, title.
  % none       Do not sort at all. All entries are processed in citation order.
  % debug      Sort by entry key. This is intended for debugging only.
  %
  sortcase=true,
  sortcites=true, % do/do not sort citations according to bib	
  %--- Dates --- --- ---
  date=comp,  % (short, long, terse, comp, iso8601)
  %	origdate=
  %	eventdate=
  %	urldate=
  %	alldates=
  datezeros=true, %
  dateabbrev=true, %
  %--- General Options --- --- ---
  maxnames=3,
  minnames=1,
  maxbibnames=100, % do not abbreviate names in bibliography
  %	autocite= % (plain, inline, footnote, superscript) 
  autopunct=true,
  language=auto,
  autolang=none, % (none, hyphen, other, other*)
  block=none, % (none, space, par, nbpar, ragged)
  notetype=foot+end, % (foot+end, footonly, endonly)
  hyperref=true, % (true, false, auto)
  backref=false,
  backrefstyle=three, % (none, three, two, two+, three+, all+)
  backrefsetstyle=setonly, %
  indexing=false, % 
  % options:
  % true       Enable indexing globally.
  % false      Disable indexing globally.
  % cite       Enable indexing in citations only.
  % bib        Enable indexing in the bibliography only.
  refsection=chapter, % (none, part, chapter, section, subsection)
  refsegment=none, % (none, part, chapter, section, subsection)
  abbreviate=true, % (true, false)
  defernumbers=false, % 
  punctfont=false, % 
  arxiv=abs, % (ps, pdf, format)	
  %--- Style Options --- --- ---	
  isbn=false,%
  url=false,%
  doi=false,%
  eprint=false,%	
}%	

% Suppress all date fields except the year
\AtEveryBibitem{%
  \clearfield{day}%
  \clearfield{month}%
  \clearfield{endday}%
  \clearfield{endmonth}%
}

\DeclareRedundantLanguages{en,EN,English}{english}

% Use only the first page number in a given range
\DeclareFieldFormat{pages}{\mkfirstpage{#1}}


\ExplSyntaxOn
% Define a function for string substitution
\cs_new:Npn \minna_replace:nn #1 #2
  {
    \tl_replace_all:Nnn \l_tmpa_tl {#1} {#2}
  }

% Wrapper macro for ease of use
\NewDocumentCommand{\ReplaceString}{ m m m }
  {
    \tl_set:Nn \l_tmpa_tl {#1}
    \minna_replace:nn {#2} {#3}
    \tl_use:N \l_tmpa_tl
  }
\ExplSyntaxOff



% ********************************************************************
% Fine-tune hyperreferences (hyperref should be called last)
% ********************************************************************

\usepackage[dvipsnames]{xcolor}


\PassOptionsToPackage{pdftex,hyperfootnotes=false,pdfpagelabels}{hyperref}
\usepackage{hyperref}  % backref linktocpage pagebackref
\pdfcompresslevel=9
\pdfadjustspacing=1

% \usepackage{hyperxmp}
%\pdfcompresslevel=9
%\pdfadjustspacing=1

% \usepackage{hyperref}  % backref linktocpage pagebackref

\hypersetup{%
  %draft, % hyperref's draft mode, for printing see below
  colorlinks=true, linktocpage=true, pdfstartpage=3, pdfstartview=FitV,%
  % uncomment the following line if you want to have black links (e.g., for printing)
  %colorlinks=false, linktocpage=false, pdfstartpage=3, pdfstartview=FitV, pdfborder={0 0 0},%
  breaklinks=true, pageanchor=true,%
  pdfpagemode=UseNone, %
  % pdfpagemode=UseOutlines,%
  plainpages=false, bookmarksnumbered, bookmarksopen=true, bookmarksopenlevel=1,%
  hypertexnames=true, pdfhighlight=/O,%nesting=true,%frenchlinks,%
  urlcolor=CTurl, linkcolor=CTlink, citecolor=CTcitation, %pagecolor=RoyalBlue,%
  %urlcolor=Black, linkcolor=Black, citecolor=Black, %pagecolor=Black,%
  pdftitle={\myTitle},%
  pdfauthor={\textcopyright\ \myName, \myUni, \myFaculty},%
  pdfsubject={},%
  pdfkeywords={},%
  pdfcreator={pdfLaTeX},%
  pdfproducer={LaTeX with hyperref and classicthesis}%
}


% ********************************************************************
% Setup autoreferences (hyperref and babel)
% ********************************************************************
% There are some issues regarding autorefnames
% http://www.tex.ac.uk/cgi-bin/texfaq2html?label=latexwords
% you have to redefine the macros for the
% language you use, e.g., american, ngerman
% (as chosen when loading babel/AtBeginDocument)
% ********************************************************************
 \makeatletter
 \@ifpackageloaded{babel}%
   {%
     \addto\extrasamerican{%
       \renewcommand*{\figureautorefname}{Figure}%
       \renewcommand*{\tableautorefname}{Table}%
       \renewcommand*{\partautorefname}{Part}%
       \renewcommand*{\chapterautorefname}{Chapter}%
       \renewcommand*{\sectionautorefname}{Section}%
       \renewcommand*{\subsectionautorefname}{Section}%
       \renewcommand*{\subsubsectionautorefname}{Section}%
     }%
     \addto\extrasngerman{%
       \renewcommand*{\paragraphautorefname}{Absatz}%
       \renewcommand*{\subparagraphautorefname}{Unterabsatz}%
       \renewcommand*{\footnoteautorefname}{Fu\"snote}%
       \renewcommand*{\FancyVerbLineautorefname}{Zeile}%
       \renewcommand*{\theoremautorefname}{Theorem}%
       \renewcommand*{\appendixautorefname}{Anhang}%
       \renewcommand*{\equationautorefname}{Gleichung}%
       \renewcommand*{\itemautorefname}{Punkt}%
     }%
       % Fix to getting autorefs for subfigures right (thanks to Belinda Vogt for changing the definition)
       \providecommand{\subfigureautorefname}{\figureautorefname}%
     }{\relax}
 \makeatother

% (Better) alternative to \autoref is \cref via the cleveref package
%\usepackage{cleveref}
%\crefformat{part}{Part #2\MakeUppercase{#1}#3}
% ------------------------------------------------------------------

% ****************************************************************************************************
% 1. Configure classicthesis for your needs here, e.g., remove "drafting" below
% in order to deactivate the time-stamp on the pages
% (see ClassicThesis.pdf for more information):
% ****************************************************************************************************
\PassOptionsToPackage{
  drafting=true,    % print version information on the bottom of the pages
  tocaligned=false, % the left column of the toc will be aligned (no indentation)
  dottedtoc=false,  % page numbers in ToC flushed right
  eulerchapternumbers=false, % use AMS Euler for chapter font (otherwise Palatino)
  floatperchapter=true,     % numbering per chapter for all floats (i.e., Figure 1.1)
  eulermath=false,  % use awesome Euler fonts for mathematical formulae (only with pdfLaTeX)
  beramono=true,    % toggle a nice monospaced font (w/ bold)
  palatino=true,    % deactivate standard font for loading another one, see the last section at the end of this file for suggestions
  %linedheaders=true, % obsolete / available for backwards compatibility
  style=classicthesis % classicthesis, arsclassica, linedheaders, plain
}{classicthesis}

% ****************************************************************************************************
% 2. Personal data and user ad-hoc commands (insert your own data here)
% ****************************************************************************************************



% % eth.tex
% Defines ETH brand colors based on:
% https://ethz.ch/staffnet/en/service/communication/corporate-design/colours.html
% 
% ------------------------------------------------------------------
% ETH Corporate Design – Primary Colors and Colour Shades Definitions
%
% PRIMARY ETH CORPORATE COLORS
%
% Colour       RGB             HEX       CMYK                 Pantone    RAL
% ----------------------------------------------------------------------------
% ETH Blue     33, 92, 175     #215CAF   100,57,0,0           2935       5005 Signalblau
% ETH Petrol   0, 120, 148     #007894   100,25,30,10         633        5009 Azurblau
% ETH Green    98, 115, 19     #627313   55,10,100,30         364        6010 Grasgrün
% ETH Bronze   142, 103, 19    #8E6713   30,36,100,25         4495       7008 Khakigrau
% ETH Red      183, 53, 45     #B7352D   0,90,80,17           1797       3031 Orientrot
% ETH Purple   167, 17, 122    #A7117A   22,100,0,10          234        4006 Verkehrspurpur
% ETH Grey     111, 111, 111   #6F6F6F   0,0,0,70             Cool Gray 11   7046 Telegrau 2
% ------------------------------------------------------------------
% ETH Corporate Design – Colour Shades Definitions
%
% 1. ETH Blue Shades
% 2. ETH Petrol Shades
% 3. ETH Green Shades
% 4. ETH Bronze Shades
% 5. ETH Red Shades
% 6. ETH Purple Shades
% 7. ETH Grey Shades
% 
% Last updated: 2025-03-08
%
% Usage:
%   % eth.tex
% Defines ETH brand colors based on:
% https://ethz.ch/staffnet/en/service/communication/corporate-design/colours.html
% 
% ------------------------------------------------------------------
% ETH Corporate Design – Primary Colors and Colour Shades Definitions
%
% PRIMARY ETH CORPORATE COLORS
%
% Colour       RGB             HEX       CMYK                 Pantone    RAL
% ----------------------------------------------------------------------------
% ETH Blue     33, 92, 175     #215CAF   100,57,0,0           2935       5005 Signalblau
% ETH Petrol   0, 120, 148     #007894   100,25,30,10         633        5009 Azurblau
% ETH Green    98, 115, 19     #627313   55,10,100,30         364        6010 Grasgrün
% ETH Bronze   142, 103, 19    #8E6713   30,36,100,25         4495       7008 Khakigrau
% ETH Red      183, 53, 45     #B7352D   0,90,80,17           1797       3031 Orientrot
% ETH Purple   167, 17, 122    #A7117A   22,100,0,10          234        4006 Verkehrspurpur
% ETH Grey     111, 111, 111   #6F6F6F   0,0,0,70             Cool Gray 11   7046 Telegrau 2
% ------------------------------------------------------------------
% ETH Corporate Design – Colour Shades Definitions
%
% 1. ETH Blue Shades
% 2. ETH Petrol Shades
% 3. ETH Green Shades
% 4. ETH Bronze Shades
% 5. ETH Red Shades
% 6. ETH Purple Shades
% 7. ETH Grey Shades
% 
% Last updated: 2025-03-08
%
% Usage:
%   \input{eth.tex}
%   \textcolor{ETHBlue}{Hello ETH!}
% Usage in your main .tex:
%   \usepackage{xcolor} % or colortbl, etc.
%   \input{eth.tex}
%   \textcolor{ETHBlue}{Hello from ETH!}
\NeedsTeXFormat{LaTeX2e}
\ProvidesFile{eth.tex}[2025/03/08 v1.0 ETH brand color definitions]

\RequirePackage{xcolor}

% ==============================================================
% Primary ETH Corporate Colors
% ==============================================================
\definecolor{ETHBlue}{HTML}{215CAF}    % ETH Blue: RGB: 33, 92, 175; CMYK: 100,57,0,0; Pantone: 2935; RAL: 5005 Signalblau
\definecolor{ETHPetrol}{HTML}{007894}   % ETH Petrol: RGB: 0,120,148; CMYK: 100,25,30,10; Pantone: 633; RAL: 5009 Azurblau
\definecolor{ETHGreen}{HTML}{627313}    % ETH Green: RGB: 98,115,19; CMYK: 55,10,100,30; Pantone: 364; RAL: 6010 Grasgrün
\definecolor{ETHBronze}{HTML}{8E6713}    % ETH Bronze: RGB: 142,103,19; CMYK: 30,36,100,25; Pantone: 4495; RAL: 7008 Khakigrau
\definecolor{ETHRed}{HTML}{B7352D}       % ETH Red: RGB: 183,53,45; CMYK: 0,90,80,17; Pantone: 1797; RAL: 3031 Orientrot
\definecolor{ETHPurple}{HTML}{A7117A}     % ETH Purple: RGB: 167,17,122; CMYK: 22,100,0,10; Pantone: 234; RAL: 4006 Verkehrspurpur
\definecolor{ETHGrey}{HTML}{6F6F6F}       % ETH Grey: RGB: 111,111,111; CMYK: 0,0,0,70; Pantone: Cool Gray 11; RAL: 7046 Telegrau 2


% ------------------------------------------------------------------
% Extended / Complementary Color Palette
% ------------------------------------------------------------------
\definecolor{ETHTeal}{HTML}{008C95}
\definecolor{ETHGreen}{HTML}{00B38B}
\definecolor{ETHDarkBlue}{HTML}{1D2447}
\definecolor{ETHLightBlue}{HTML}{5BB6D6}
\definecolor{ETHOrange}{HTML}{F39200}
\definecolor{ETHRed}{HTML}{C8002A}
\definecolor{ETHWarmGray}{HTML}{DAD7D2}
\definecolor{ETHBeige}{HTML}{D7CEC1}
\definecolor{ETHDarkBrown}{HTML}{7F4F3C}
\definecolor{ETHDarkPink}{HTML}{EB67BD}
\definecolor{ETHDarkPurple}{HTML}{5F2167}
\definecolor{ETHDarkMagenta}{HTML}{A3488E}
\definecolor{ETHDarkGray}{HTML}{333333}
\definecolor{ETHGray}{HTML}{75787B}
\definecolor{ETHLightGray}{HTML}{E2E2E2}
\definecolor{ETHWhite}{HTML}{FFFFFF}
\definecolor{ETHBlack}{HTML}{000000}

% ==============================================================
% 1. ETH Blue Shades
% ==============================================================
% Shade | RGB             | HEX      | CMYK
% ------|-----------------|----------|---------------
% 10%   | 233, 239, 247   | E9EFF7   | 10, 6, 0, 0
\definecolor{ETHBlue10}{HTML}{E9EFF7}
% 20%   | 211, 222, 239   | D3DEEF   | 20, 11, 0, 0
\definecolor{ETHBlue20}{HTML}{D3DEEF}
% 40%   | 166, 190, 223   | A6BEDF   | 40, 23, 0, 0
\definecolor{ETHBlue40}{HTML}{A6BEDF}
% 60%   | 122, 157, 207   | 7A9DCF   | 60, 34, 0, 0
\definecolor{ETHBlue60}{HTML}{7A9DCF}
% 80%   | 77, 125, 191    | 4D7DBF   | 80, 46, 0, 0
\definecolor{ETHBlue80}{HTML}{4D7DBF}
% 120%  | 8, 64, 126      | 08407E   | 100, 62, 0, 30
\definecolor{ETHBlue120}{HTML}{08407E}

% ==============================================================
% 2. ETH Petrol Shades
% ==============================================================
% Shade | RGB             | HEX      | CMYK
% ------|-----------------|----------|---------------
% 10%   | 231, 244, 247   | E7F4F7   | 12, 0, 5, 0
\definecolor{ETHPetrol10}{HTML}{E7F4F7}
% 20%   | 204, 228, 234   | CCE4EA   | 20, 3, 7, 0
\definecolor{ETHPetrol20}{HTML}{CCE4EA}
% 40%   | 153, 202, 213   | 99CAD5   | 40, 7, 12, 4
\definecolor{ETHPetrol40}{HTML}{99CAD5}
% 60%   | 102, 175, 192   | 66AFC0   | 60, 14, 18, 6
\definecolor{ETHPetrol60}{HTML}{66AFC0}
% 80%   | 51, 149, 171    | 3395AB   | 80, 20, 24, 8
\definecolor{ETHPetrol80}{HTML}{3395AB}
% 120%  | 0, 89, 109      | 00596D   | 100, 25, 30, 38
\definecolor{ETHPetrol120}{HTML}{00596D}

% ==============================================================
% 3. ETH Green Shades
% ==============================================================
% Shade | RGB             | HEX      | CMYK
% ------|-----------------|----------|---------------
% 10%   | 239, 241, 231   | EEF1E7   | 6, 1, 10, 3
\definecolor{ETHGreen10}{HTML}{EEF1E7}
% 20%   | 224, 227, 208   | E0E3D0   | 11, 2, 20, 6
\definecolor{ETHGreen20}{HTML}{E0E3D0}
% 40%   | 192, 199, 161   | C0C7A1   | 22, 4, 40, 12
\definecolor{ETHGreen40}{HTML}{C0C7A1}
% 60%   | 161, 171, 113   | A1AB71   | 33, 6, 60, 18
\definecolor{ETHGreen60}{HTML}{A1AB71}
% 80%   | 129, 143, 66    | 818F42   | 44, 8, 80, 24
\definecolor{ETHGreen80}{HTML}{818F42}
% 120%  | 54, 82, 19      | 365213   | 55, 10, 100, 65
\definecolor{ETHGreen120}{HTML}{365213}

% ==============================================================
% 4. ETH Bronze Shades
% ==============================================================
% Shade | RGB             | HEX      | CMYK
% ------|-----------------|----------|---------------
% 10%   | 244, 240, 231   | F4F0E7   | 3, 4, 10, 3
\definecolor{ETHBronze10}{HTML}{F4F0E7}
% 20%   | 232, 225, 208   | E8E1D0   | 6, 7, 20, 5
\definecolor{ETHBronze20}{HTML}{E8E1D0}
% 40%   | 210, 194, 161   | D2C2A1   | 12, 14, 40, 10
\definecolor{ETHBronze40}{HTML}{D2C2A1}
% 60%   | 187, 164, 113   | BBA471   | 18, 22, 60, 15
\definecolor{ETHBronze60}{HTML}{BBA471}
% 80%   | 165, 133, 66    | A58542   | 24, 29, 80, 20
\definecolor{ETHBronze80}{HTML}{A58542}
% 120%  | 112, 79, 18     | 704F12   | 30, 36, 100, 55
\definecolor{ETHBronze120}{HTML}{704F12}

% ==============================================================
% 5. ETH Red Shades
% ==============================================================
% Shade | RGB             | HEX      | CMYK
% ------|-----------------|----------|---------------
% 10%   | 248, 235, 234   | F8EBEA   | 0, 9, 6, 0
\definecolor{ETHRed10}{HTML}{F8EBEA}
% 20%   | 241, 215, 213   | F1D7D5   | 0, 18, 13, 4
\definecolor{ETHRed20}{HTML}{F1D7D5}
% 40%   | 226, 174, 171   | E2AEAB   | 0, 36, 26, 8
\definecolor{ETHRed40}{HTML}{E2AEAB}
% 60%   | 212, 134, 129   | D48681   | 0, 54, 39, 11
\definecolor{ETHRed60}{HTML}{D48681}
% 80%   | 197, 93, 87     | C55D57   | (using HEX)
\definecolor{ETHRed80}{HTML}{C55D57}
% 120%  | 150, 39, 45     | 96272D   | 0, 100, 80, 40
\definecolor{ETHRed120}{HTML}{96272D}

% ==============================================================
% 6. ETH Purple Shades
% ==============================================================
% Shade | RGB             | HEX      | CMYK
% ------|-----------------|----------|---------------
% 10%   | 248, 232, 243   | F8E8F3   | 2, 10, 0, 1
\definecolor{ETHPurple10}{HTML}{F8E8F3}
% 20%   | 239, 208, 227   | EFD0E3   | 4, 20, 0, 1
\definecolor{ETHPurple20}{HTML}{EFD0E3}
% 40%   | 220, 158, 201   | DC9EC9   | 7, 40, 0, 4
\definecolor{ETHPurple40}{HTML}{DC9EC9}
% 60%   | 202, 108, 174   | CA6CAE   | 13, 60, 0, 6
\definecolor{ETHPurple60}{HTML}{CA6CAE}
% 80%   | 183, 59, 146    | B73B92   | 18, 80, 0, 8
\definecolor{ETHPurple80}{HTML}{B73B92}
% 120%  | 140, 10, 89     | 8C0A59   | 22, 100, 0, 35
\definecolor{ETHPurple120}{HTML}{8C0A59}

% ==============================================================
% 7. ETH Grey Shades
% ==============================================================
% Shade | RGB             | HEX      | CMYK
% ------|-----------------|----------|---------------
% 10%   | 241, 241, 241   | F1F1F1   | 0, 0, 0, 7
\definecolor{ETHGrey10}{HTML}{F1F1F1}
% 20%   | 226, 226, 226   | E2E2E2   | 0, 0, 0, 14
\definecolor{ETHGrey20}{HTML}{E2E2E2}
% 40%   | 197, 197, 197   | C5C5C5   | 0, 0, 0, 28
\definecolor{ETHGrey40}{HTML}{C5C5C5}
% 60%   | 169, 169, 169   | A9A9A9   | 0, 0, 0, 42
\definecolor{ETHGrey60}{HTML}{A9A9A9}
% 80%   | 140, 140, 140   | 8C8C8C   | 0, 0, 0, 56
\definecolor{ETHGrey80}{HTML}{8C8C8C}
% 120%  | 87, 87, 87      | 575757   | 0, 0, 0, 81
\definecolor{ETHGrey120}{HTML}{575757}
%   \textcolor{ETHBlue}{Hello ETH!}
% Usage in your main .tex:
%   \usepackage{xcolor} % or colortbl, etc.
%   % eth.tex
% Defines ETH brand colors based on:
% https://ethz.ch/staffnet/en/service/communication/corporate-design/colours.html
% 
% ------------------------------------------------------------------
% ETH Corporate Design – Primary Colors and Colour Shades Definitions
%
% PRIMARY ETH CORPORATE COLORS
%
% Colour       RGB             HEX       CMYK                 Pantone    RAL
% ----------------------------------------------------------------------------
% ETH Blue     33, 92, 175     #215CAF   100,57,0,0           2935       5005 Signalblau
% ETH Petrol   0, 120, 148     #007894   100,25,30,10         633        5009 Azurblau
% ETH Green    98, 115, 19     #627313   55,10,100,30         364        6010 Grasgrün
% ETH Bronze   142, 103, 19    #8E6713   30,36,100,25         4495       7008 Khakigrau
% ETH Red      183, 53, 45     #B7352D   0,90,80,17           1797       3031 Orientrot
% ETH Purple   167, 17, 122    #A7117A   22,100,0,10          234        4006 Verkehrspurpur
% ETH Grey     111, 111, 111   #6F6F6F   0,0,0,70             Cool Gray 11   7046 Telegrau 2
% ------------------------------------------------------------------
% ETH Corporate Design – Colour Shades Definitions
%
% 1. ETH Blue Shades
% 2. ETH Petrol Shades
% 3. ETH Green Shades
% 4. ETH Bronze Shades
% 5. ETH Red Shades
% 6. ETH Purple Shades
% 7. ETH Grey Shades
% 
% Last updated: 2025-03-08
%
% Usage:
%   \input{eth.tex}
%   \textcolor{ETHBlue}{Hello ETH!}
% Usage in your main .tex:
%   \usepackage{xcolor} % or colortbl, etc.
%   \input{eth.tex}
%   \textcolor{ETHBlue}{Hello from ETH!}
\NeedsTeXFormat{LaTeX2e}
\ProvidesFile{eth.tex}[2025/03/08 v1.0 ETH brand color definitions]

\RequirePackage{xcolor}

% ==============================================================
% Primary ETH Corporate Colors
% ==============================================================
\definecolor{ETHBlue}{HTML}{215CAF}    % ETH Blue: RGB: 33, 92, 175; CMYK: 100,57,0,0; Pantone: 2935; RAL: 5005 Signalblau
\definecolor{ETHPetrol}{HTML}{007894}   % ETH Petrol: RGB: 0,120,148; CMYK: 100,25,30,10; Pantone: 633; RAL: 5009 Azurblau
\definecolor{ETHGreen}{HTML}{627313}    % ETH Green: RGB: 98,115,19; CMYK: 55,10,100,30; Pantone: 364; RAL: 6010 Grasgrün
\definecolor{ETHBronze}{HTML}{8E6713}    % ETH Bronze: RGB: 142,103,19; CMYK: 30,36,100,25; Pantone: 4495; RAL: 7008 Khakigrau
\definecolor{ETHRed}{HTML}{B7352D}       % ETH Red: RGB: 183,53,45; CMYK: 0,90,80,17; Pantone: 1797; RAL: 3031 Orientrot
\definecolor{ETHPurple}{HTML}{A7117A}     % ETH Purple: RGB: 167,17,122; CMYK: 22,100,0,10; Pantone: 234; RAL: 4006 Verkehrspurpur
\definecolor{ETHGrey}{HTML}{6F6F6F}       % ETH Grey: RGB: 111,111,111; CMYK: 0,0,0,70; Pantone: Cool Gray 11; RAL: 7046 Telegrau 2


% ------------------------------------------------------------------
% Extended / Complementary Color Palette
% ------------------------------------------------------------------
\definecolor{ETHTeal}{HTML}{008C95}
\definecolor{ETHGreen}{HTML}{00B38B}
\definecolor{ETHDarkBlue}{HTML}{1D2447}
\definecolor{ETHLightBlue}{HTML}{5BB6D6}
\definecolor{ETHOrange}{HTML}{F39200}
\definecolor{ETHRed}{HTML}{C8002A}
\definecolor{ETHWarmGray}{HTML}{DAD7D2}
\definecolor{ETHBeige}{HTML}{D7CEC1}
\definecolor{ETHDarkBrown}{HTML}{7F4F3C}
\definecolor{ETHDarkPink}{HTML}{EB67BD}
\definecolor{ETHDarkPurple}{HTML}{5F2167}
\definecolor{ETHDarkMagenta}{HTML}{A3488E}
\definecolor{ETHDarkGray}{HTML}{333333}
\definecolor{ETHGray}{HTML}{75787B}
\definecolor{ETHLightGray}{HTML}{E2E2E2}
\definecolor{ETHWhite}{HTML}{FFFFFF}
\definecolor{ETHBlack}{HTML}{000000}

% ==============================================================
% 1. ETH Blue Shades
% ==============================================================
% Shade | RGB             | HEX      | CMYK
% ------|-----------------|----------|---------------
% 10%   | 233, 239, 247   | E9EFF7   | 10, 6, 0, 0
\definecolor{ETHBlue10}{HTML}{E9EFF7}
% 20%   | 211, 222, 239   | D3DEEF   | 20, 11, 0, 0
\definecolor{ETHBlue20}{HTML}{D3DEEF}
% 40%   | 166, 190, 223   | A6BEDF   | 40, 23, 0, 0
\definecolor{ETHBlue40}{HTML}{A6BEDF}
% 60%   | 122, 157, 207   | 7A9DCF   | 60, 34, 0, 0
\definecolor{ETHBlue60}{HTML}{7A9DCF}
% 80%   | 77, 125, 191    | 4D7DBF   | 80, 46, 0, 0
\definecolor{ETHBlue80}{HTML}{4D7DBF}
% 120%  | 8, 64, 126      | 08407E   | 100, 62, 0, 30
\definecolor{ETHBlue120}{HTML}{08407E}

% ==============================================================
% 2. ETH Petrol Shades
% ==============================================================
% Shade | RGB             | HEX      | CMYK
% ------|-----------------|----------|---------------
% 10%   | 231, 244, 247   | E7F4F7   | 12, 0, 5, 0
\definecolor{ETHPetrol10}{HTML}{E7F4F7}
% 20%   | 204, 228, 234   | CCE4EA   | 20, 3, 7, 0
\definecolor{ETHPetrol20}{HTML}{CCE4EA}
% 40%   | 153, 202, 213   | 99CAD5   | 40, 7, 12, 4
\definecolor{ETHPetrol40}{HTML}{99CAD5}
% 60%   | 102, 175, 192   | 66AFC0   | 60, 14, 18, 6
\definecolor{ETHPetrol60}{HTML}{66AFC0}
% 80%   | 51, 149, 171    | 3395AB   | 80, 20, 24, 8
\definecolor{ETHPetrol80}{HTML}{3395AB}
% 120%  | 0, 89, 109      | 00596D   | 100, 25, 30, 38
\definecolor{ETHPetrol120}{HTML}{00596D}

% ==============================================================
% 3. ETH Green Shades
% ==============================================================
% Shade | RGB             | HEX      | CMYK
% ------|-----------------|----------|---------------
% 10%   | 239, 241, 231   | EEF1E7   | 6, 1, 10, 3
\definecolor{ETHGreen10}{HTML}{EEF1E7}
% 20%   | 224, 227, 208   | E0E3D0   | 11, 2, 20, 6
\definecolor{ETHGreen20}{HTML}{E0E3D0}
% 40%   | 192, 199, 161   | C0C7A1   | 22, 4, 40, 12
\definecolor{ETHGreen40}{HTML}{C0C7A1}
% 60%   | 161, 171, 113   | A1AB71   | 33, 6, 60, 18
\definecolor{ETHGreen60}{HTML}{A1AB71}
% 80%   | 129, 143, 66    | 818F42   | 44, 8, 80, 24
\definecolor{ETHGreen80}{HTML}{818F42}
% 120%  | 54, 82, 19      | 365213   | 55, 10, 100, 65
\definecolor{ETHGreen120}{HTML}{365213}

% ==============================================================
% 4. ETH Bronze Shades
% ==============================================================
% Shade | RGB             | HEX      | CMYK
% ------|-----------------|----------|---------------
% 10%   | 244, 240, 231   | F4F0E7   | 3, 4, 10, 3
\definecolor{ETHBronze10}{HTML}{F4F0E7}
% 20%   | 232, 225, 208   | E8E1D0   | 6, 7, 20, 5
\definecolor{ETHBronze20}{HTML}{E8E1D0}
% 40%   | 210, 194, 161   | D2C2A1   | 12, 14, 40, 10
\definecolor{ETHBronze40}{HTML}{D2C2A1}
% 60%   | 187, 164, 113   | BBA471   | 18, 22, 60, 15
\definecolor{ETHBronze60}{HTML}{BBA471}
% 80%   | 165, 133, 66    | A58542   | 24, 29, 80, 20
\definecolor{ETHBronze80}{HTML}{A58542}
% 120%  | 112, 79, 18     | 704F12   | 30, 36, 100, 55
\definecolor{ETHBronze120}{HTML}{704F12}

% ==============================================================
% 5. ETH Red Shades
% ==============================================================
% Shade | RGB             | HEX      | CMYK
% ------|-----------------|----------|---------------
% 10%   | 248, 235, 234   | F8EBEA   | 0, 9, 6, 0
\definecolor{ETHRed10}{HTML}{F8EBEA}
% 20%   | 241, 215, 213   | F1D7D5   | 0, 18, 13, 4
\definecolor{ETHRed20}{HTML}{F1D7D5}
% 40%   | 226, 174, 171   | E2AEAB   | 0, 36, 26, 8
\definecolor{ETHRed40}{HTML}{E2AEAB}
% 60%   | 212, 134, 129   | D48681   | 0, 54, 39, 11
\definecolor{ETHRed60}{HTML}{D48681}
% 80%   | 197, 93, 87     | C55D57   | (using HEX)
\definecolor{ETHRed80}{HTML}{C55D57}
% 120%  | 150, 39, 45     | 96272D   | 0, 100, 80, 40
\definecolor{ETHRed120}{HTML}{96272D}

% ==============================================================
% 6. ETH Purple Shades
% ==============================================================
% Shade | RGB             | HEX      | CMYK
% ------|-----------------|----------|---------------
% 10%   | 248, 232, 243   | F8E8F3   | 2, 10, 0, 1
\definecolor{ETHPurple10}{HTML}{F8E8F3}
% 20%   | 239, 208, 227   | EFD0E3   | 4, 20, 0, 1
\definecolor{ETHPurple20}{HTML}{EFD0E3}
% 40%   | 220, 158, 201   | DC9EC9   | 7, 40, 0, 4
\definecolor{ETHPurple40}{HTML}{DC9EC9}
% 60%   | 202, 108, 174   | CA6CAE   | 13, 60, 0, 6
\definecolor{ETHPurple60}{HTML}{CA6CAE}
% 80%   | 183, 59, 146    | B73B92   | 18, 80, 0, 8
\definecolor{ETHPurple80}{HTML}{B73B92}
% 120%  | 140, 10, 89     | 8C0A59   | 22, 100, 0, 35
\definecolor{ETHPurple120}{HTML}{8C0A59}

% ==============================================================
% 7. ETH Grey Shades
% ==============================================================
% Shade | RGB             | HEX      | CMYK
% ------|-----------------|----------|---------------
% 10%   | 241, 241, 241   | F1F1F1   | 0, 0, 0, 7
\definecolor{ETHGrey10}{HTML}{F1F1F1}
% 20%   | 226, 226, 226   | E2E2E2   | 0, 0, 0, 14
\definecolor{ETHGrey20}{HTML}{E2E2E2}
% 40%   | 197, 197, 197   | C5C5C5   | 0, 0, 0, 28
\definecolor{ETHGrey40}{HTML}{C5C5C5}
% 60%   | 169, 169, 169   | A9A9A9   | 0, 0, 0, 42
\definecolor{ETHGrey60}{HTML}{A9A9A9}
% 80%   | 140, 140, 140   | 8C8C8C   | 0, 0, 0, 56
\definecolor{ETHGrey80}{HTML}{8C8C8C}
% 120%  | 87, 87, 87      | 575757   | 0, 0, 0, 81
\definecolor{ETHGrey120}{HTML}{575757}
%   \textcolor{ETHBlue}{Hello from ETH!}
\NeedsTeXFormat{LaTeX2e}
\ProvidesFile{eth.tex}[2025/03/08 v1.0 ETH brand color definitions]

\RequirePackage{xcolor}

% ==============================================================
% Primary ETH Corporate Colors
% ==============================================================
\definecolor{ETHBlue}{HTML}{215CAF}    % ETH Blue: RGB: 33, 92, 175; CMYK: 100,57,0,0; Pantone: 2935; RAL: 5005 Signalblau
\definecolor{ETHPetrol}{HTML}{007894}   % ETH Petrol: RGB: 0,120,148; CMYK: 100,25,30,10; Pantone: 633; RAL: 5009 Azurblau
\definecolor{ETHGreen}{HTML}{627313}    % ETH Green: RGB: 98,115,19; CMYK: 55,10,100,30; Pantone: 364; RAL: 6010 Grasgrün
\definecolor{ETHBronze}{HTML}{8E6713}    % ETH Bronze: RGB: 142,103,19; CMYK: 30,36,100,25; Pantone: 4495; RAL: 7008 Khakigrau
\definecolor{ETHRed}{HTML}{B7352D}       % ETH Red: RGB: 183,53,45; CMYK: 0,90,80,17; Pantone: 1797; RAL: 3031 Orientrot
\definecolor{ETHPurple}{HTML}{A7117A}     % ETH Purple: RGB: 167,17,122; CMYK: 22,100,0,10; Pantone: 234; RAL: 4006 Verkehrspurpur
\definecolor{ETHGrey}{HTML}{6F6F6F}       % ETH Grey: RGB: 111,111,111; CMYK: 0,0,0,70; Pantone: Cool Gray 11; RAL: 7046 Telegrau 2


% ------------------------------------------------------------------
% Extended / Complementary Color Palette
% ------------------------------------------------------------------
\definecolor{ETHTeal}{HTML}{008C95}
\definecolor{ETHGreen}{HTML}{00B38B}
\definecolor{ETHDarkBlue}{HTML}{1D2447}
\definecolor{ETHLightBlue}{HTML}{5BB6D6}
\definecolor{ETHOrange}{HTML}{F39200}
\definecolor{ETHRed}{HTML}{C8002A}
\definecolor{ETHWarmGray}{HTML}{DAD7D2}
\definecolor{ETHBeige}{HTML}{D7CEC1}
\definecolor{ETHDarkBrown}{HTML}{7F4F3C}
\definecolor{ETHDarkPink}{HTML}{EB67BD}
\definecolor{ETHDarkPurple}{HTML}{5F2167}
\definecolor{ETHDarkMagenta}{HTML}{A3488E}
\definecolor{ETHDarkGray}{HTML}{333333}
\definecolor{ETHGray}{HTML}{75787B}
\definecolor{ETHLightGray}{HTML}{E2E2E2}
\definecolor{ETHWhite}{HTML}{FFFFFF}
\definecolor{ETHBlack}{HTML}{000000}

% ==============================================================
% 1. ETH Blue Shades
% ==============================================================
% Shade | RGB             | HEX      | CMYK
% ------|-----------------|----------|---------------
% 10%   | 233, 239, 247   | E9EFF7   | 10, 6, 0, 0
\definecolor{ETHBlue10}{HTML}{E9EFF7}
% 20%   | 211, 222, 239   | D3DEEF   | 20, 11, 0, 0
\definecolor{ETHBlue20}{HTML}{D3DEEF}
% 40%   | 166, 190, 223   | A6BEDF   | 40, 23, 0, 0
\definecolor{ETHBlue40}{HTML}{A6BEDF}
% 60%   | 122, 157, 207   | 7A9DCF   | 60, 34, 0, 0
\definecolor{ETHBlue60}{HTML}{7A9DCF}
% 80%   | 77, 125, 191    | 4D7DBF   | 80, 46, 0, 0
\definecolor{ETHBlue80}{HTML}{4D7DBF}
% 120%  | 8, 64, 126      | 08407E   | 100, 62, 0, 30
\definecolor{ETHBlue120}{HTML}{08407E}

% ==============================================================
% 2. ETH Petrol Shades
% ==============================================================
% Shade | RGB             | HEX      | CMYK
% ------|-----------------|----------|---------------
% 10%   | 231, 244, 247   | E7F4F7   | 12, 0, 5, 0
\definecolor{ETHPetrol10}{HTML}{E7F4F7}
% 20%   | 204, 228, 234   | CCE4EA   | 20, 3, 7, 0
\definecolor{ETHPetrol20}{HTML}{CCE4EA}
% 40%   | 153, 202, 213   | 99CAD5   | 40, 7, 12, 4
\definecolor{ETHPetrol40}{HTML}{99CAD5}
% 60%   | 102, 175, 192   | 66AFC0   | 60, 14, 18, 6
\definecolor{ETHPetrol60}{HTML}{66AFC0}
% 80%   | 51, 149, 171    | 3395AB   | 80, 20, 24, 8
\definecolor{ETHPetrol80}{HTML}{3395AB}
% 120%  | 0, 89, 109      | 00596D   | 100, 25, 30, 38
\definecolor{ETHPetrol120}{HTML}{00596D}

% ==============================================================
% 3. ETH Green Shades
% ==============================================================
% Shade | RGB             | HEX      | CMYK
% ------|-----------------|----------|---------------
% 10%   | 239, 241, 231   | EEF1E7   | 6, 1, 10, 3
\definecolor{ETHGreen10}{HTML}{EEF1E7}
% 20%   | 224, 227, 208   | E0E3D0   | 11, 2, 20, 6
\definecolor{ETHGreen20}{HTML}{E0E3D0}
% 40%   | 192, 199, 161   | C0C7A1   | 22, 4, 40, 12
\definecolor{ETHGreen40}{HTML}{C0C7A1}
% 60%   | 161, 171, 113   | A1AB71   | 33, 6, 60, 18
\definecolor{ETHGreen60}{HTML}{A1AB71}
% 80%   | 129, 143, 66    | 818F42   | 44, 8, 80, 24
\definecolor{ETHGreen80}{HTML}{818F42}
% 120%  | 54, 82, 19      | 365213   | 55, 10, 100, 65
\definecolor{ETHGreen120}{HTML}{365213}

% ==============================================================
% 4. ETH Bronze Shades
% ==============================================================
% Shade | RGB             | HEX      | CMYK
% ------|-----------------|----------|---------------
% 10%   | 244, 240, 231   | F4F0E7   | 3, 4, 10, 3
\definecolor{ETHBronze10}{HTML}{F4F0E7}
% 20%   | 232, 225, 208   | E8E1D0   | 6, 7, 20, 5
\definecolor{ETHBronze20}{HTML}{E8E1D0}
% 40%   | 210, 194, 161   | D2C2A1   | 12, 14, 40, 10
\definecolor{ETHBronze40}{HTML}{D2C2A1}
% 60%   | 187, 164, 113   | BBA471   | 18, 22, 60, 15
\definecolor{ETHBronze60}{HTML}{BBA471}
% 80%   | 165, 133, 66    | A58542   | 24, 29, 80, 20
\definecolor{ETHBronze80}{HTML}{A58542}
% 120%  | 112, 79, 18     | 704F12   | 30, 36, 100, 55
\definecolor{ETHBronze120}{HTML}{704F12}

% ==============================================================
% 5. ETH Red Shades
% ==============================================================
% Shade | RGB             | HEX      | CMYK
% ------|-----------------|----------|---------------
% 10%   | 248, 235, 234   | F8EBEA   | 0, 9, 6, 0
\definecolor{ETHRed10}{HTML}{F8EBEA}
% 20%   | 241, 215, 213   | F1D7D5   | 0, 18, 13, 4
\definecolor{ETHRed20}{HTML}{F1D7D5}
% 40%   | 226, 174, 171   | E2AEAB   | 0, 36, 26, 8
\definecolor{ETHRed40}{HTML}{E2AEAB}
% 60%   | 212, 134, 129   | D48681   | 0, 54, 39, 11
\definecolor{ETHRed60}{HTML}{D48681}
% 80%   | 197, 93, 87     | C55D57   | (using HEX)
\definecolor{ETHRed80}{HTML}{C55D57}
% 120%  | 150, 39, 45     | 96272D   | 0, 100, 80, 40
\definecolor{ETHRed120}{HTML}{96272D}

% ==============================================================
% 6. ETH Purple Shades
% ==============================================================
% Shade | RGB             | HEX      | CMYK
% ------|-----------------|----------|---------------
% 10%   | 248, 232, 243   | F8E8F3   | 2, 10, 0, 1
\definecolor{ETHPurple10}{HTML}{F8E8F3}
% 20%   | 239, 208, 227   | EFD0E3   | 4, 20, 0, 1
\definecolor{ETHPurple20}{HTML}{EFD0E3}
% 40%   | 220, 158, 201   | DC9EC9   | 7, 40, 0, 4
\definecolor{ETHPurple40}{HTML}{DC9EC9}
% 60%   | 202, 108, 174   | CA6CAE   | 13, 60, 0, 6
\definecolor{ETHPurple60}{HTML}{CA6CAE}
% 80%   | 183, 59, 146    | B73B92   | 18, 80, 0, 8
\definecolor{ETHPurple80}{HTML}{B73B92}
% 120%  | 140, 10, 89     | 8C0A59   | 22, 100, 0, 35
\definecolor{ETHPurple120}{HTML}{8C0A59}

% ==============================================================
% 7. ETH Grey Shades
% ==============================================================
% Shade | RGB             | HEX      | CMYK
% ------|-----------------|----------|---------------
% 10%   | 241, 241, 241   | F1F1F1   | 0, 0, 0, 7
\definecolor{ETHGrey10}{HTML}{F1F1F1}
% 20%   | 226, 226, 226   | E2E2E2   | 0, 0, 0, 14
\definecolor{ETHGrey20}{HTML}{E2E2E2}
% 40%   | 197, 197, 197   | C5C5C5   | 0, 0, 0, 28
\definecolor{ETHGrey40}{HTML}{C5C5C5}
% 60%   | 169, 169, 169   | A9A9A9   | 0, 0, 0, 42
\definecolor{ETHGrey60}{HTML}{A9A9A9}
% 80%   | 140, 140, 140   | 8C8C8C   | 0, 0, 0, 56
\definecolor{ETHGrey80}{HTML}{8C8C8C}
% 120%  | 87, 87, 87      | 575757   | 0, 0, 0, 81
\definecolor{ETHGrey120}{HTML}{575757}  % Load ETH corporate colours and shade definitions
% eth.tex
% Defines ETH brand colors based on:
% https://ethz.ch/staffnet/en/service/communication/corporate-design/colours.html
% 
% ------------------------------------------------------------------
% ETH Corporate Design – Primary Colors and Colour Shades Definitions
%
% PRIMARY ETH CORPORATE COLORS
%
% Colour       RGB             HEX       CMYK                 Pantone    RAL
% ----------------------------------------------------------------------------
% ETH Blue     33, 92, 175     #215CAF   100,57,0,0           2935       5005 Signalblau
% ETH Petrol   0, 120, 148     #007894   100,25,30,10         633        5009 Azurblau
% ETH Green    98, 115, 19     #627313   55,10,100,30         364        6010 Grasgrün
% ETH Bronze   142, 103, 19    #8E6713   30,36,100,25         4495       7008 Khakigrau
% ETH Red      183, 53, 45     #B7352D   0,90,80,17           1797       3031 Orientrot
% ETH Purple   167, 17, 122    #A7117A   22,100,0,10          234        4006 Verkehrspurpur
% ETH Grey     111, 111, 111   #6F6F6F   0,0,0,70             Cool Gray 11   7046 Telegrau 2
% ------------------------------------------------------------------
% ETH Corporate Design – Colour Shades Definitions
%
% 1. ETH Blue Shades
% 2. ETH Petrol Shades
% 3. ETH Green Shades
% 4. ETH Bronze Shades
% 5. ETH Red Shades
% 6. ETH Purple Shades
% 7. ETH Grey Shades
% 
% Last updated: 2025-03-08
%
% Usage:
%   % eth.tex
% Defines ETH brand colors based on:
% https://ethz.ch/staffnet/en/service/communication/corporate-design/colours.html
% 
% ------------------------------------------------------------------
% ETH Corporate Design – Primary Colors and Colour Shades Definitions
%
% PRIMARY ETH CORPORATE COLORS
%
% Colour       RGB             HEX       CMYK                 Pantone    RAL
% ----------------------------------------------------------------------------
% ETH Blue     33, 92, 175     #215CAF   100,57,0,0           2935       5005 Signalblau
% ETH Petrol   0, 120, 148     #007894   100,25,30,10         633        5009 Azurblau
% ETH Green    98, 115, 19     #627313   55,10,100,30         364        6010 Grasgrün
% ETH Bronze   142, 103, 19    #8E6713   30,36,100,25         4495       7008 Khakigrau
% ETH Red      183, 53, 45     #B7352D   0,90,80,17           1797       3031 Orientrot
% ETH Purple   167, 17, 122    #A7117A   22,100,0,10          234        4006 Verkehrspurpur
% ETH Grey     111, 111, 111   #6F6F6F   0,0,0,70             Cool Gray 11   7046 Telegrau 2
% ------------------------------------------------------------------
% ETH Corporate Design – Colour Shades Definitions
%
% 1. ETH Blue Shades
% 2. ETH Petrol Shades
% 3. ETH Green Shades
% 4. ETH Bronze Shades
% 5. ETH Red Shades
% 6. ETH Purple Shades
% 7. ETH Grey Shades
% 
% Last updated: 2025-03-08
%
% Usage:
%   \input{eth.tex}
%   \textcolor{ETHBlue}{Hello ETH!}
% Usage in your main .tex:
%   \usepackage{xcolor} % or colortbl, etc.
%   \input{eth.tex}
%   \textcolor{ETHBlue}{Hello from ETH!}
\NeedsTeXFormat{LaTeX2e}
\ProvidesFile{eth.tex}[2025/03/08 v1.0 ETH brand color definitions]

\RequirePackage{xcolor}

% ==============================================================
% Primary ETH Corporate Colors
% ==============================================================
\definecolor{ETHBlue}{HTML}{215CAF}    % ETH Blue: RGB: 33, 92, 175; CMYK: 100,57,0,0; Pantone: 2935; RAL: 5005 Signalblau
\definecolor{ETHPetrol}{HTML}{007894}   % ETH Petrol: RGB: 0,120,148; CMYK: 100,25,30,10; Pantone: 633; RAL: 5009 Azurblau
\definecolor{ETHGreen}{HTML}{627313}    % ETH Green: RGB: 98,115,19; CMYK: 55,10,100,30; Pantone: 364; RAL: 6010 Grasgrün
\definecolor{ETHBronze}{HTML}{8E6713}    % ETH Bronze: RGB: 142,103,19; CMYK: 30,36,100,25; Pantone: 4495; RAL: 7008 Khakigrau
\definecolor{ETHRed}{HTML}{B7352D}       % ETH Red: RGB: 183,53,45; CMYK: 0,90,80,17; Pantone: 1797; RAL: 3031 Orientrot
\definecolor{ETHPurple}{HTML}{A7117A}     % ETH Purple: RGB: 167,17,122; CMYK: 22,100,0,10; Pantone: 234; RAL: 4006 Verkehrspurpur
\definecolor{ETHGrey}{HTML}{6F6F6F}       % ETH Grey: RGB: 111,111,111; CMYK: 0,0,0,70; Pantone: Cool Gray 11; RAL: 7046 Telegrau 2


% ------------------------------------------------------------------
% Extended / Complementary Color Palette
% ------------------------------------------------------------------
\definecolor{ETHTeal}{HTML}{008C95}
\definecolor{ETHGreen}{HTML}{00B38B}
\definecolor{ETHDarkBlue}{HTML}{1D2447}
\definecolor{ETHLightBlue}{HTML}{5BB6D6}
\definecolor{ETHOrange}{HTML}{F39200}
\definecolor{ETHRed}{HTML}{C8002A}
\definecolor{ETHWarmGray}{HTML}{DAD7D2}
\definecolor{ETHBeige}{HTML}{D7CEC1}
\definecolor{ETHDarkBrown}{HTML}{7F4F3C}
\definecolor{ETHDarkPink}{HTML}{EB67BD}
\definecolor{ETHDarkPurple}{HTML}{5F2167}
\definecolor{ETHDarkMagenta}{HTML}{A3488E}
\definecolor{ETHDarkGray}{HTML}{333333}
\definecolor{ETHGray}{HTML}{75787B}
\definecolor{ETHLightGray}{HTML}{E2E2E2}
\definecolor{ETHWhite}{HTML}{FFFFFF}
\definecolor{ETHBlack}{HTML}{000000}

% ==============================================================
% 1. ETH Blue Shades
% ==============================================================
% Shade | RGB             | HEX      | CMYK
% ------|-----------------|----------|---------------
% 10%   | 233, 239, 247   | E9EFF7   | 10, 6, 0, 0
\definecolor{ETHBlue10}{HTML}{E9EFF7}
% 20%   | 211, 222, 239   | D3DEEF   | 20, 11, 0, 0
\definecolor{ETHBlue20}{HTML}{D3DEEF}
% 40%   | 166, 190, 223   | A6BEDF   | 40, 23, 0, 0
\definecolor{ETHBlue40}{HTML}{A6BEDF}
% 60%   | 122, 157, 207   | 7A9DCF   | 60, 34, 0, 0
\definecolor{ETHBlue60}{HTML}{7A9DCF}
% 80%   | 77, 125, 191    | 4D7DBF   | 80, 46, 0, 0
\definecolor{ETHBlue80}{HTML}{4D7DBF}
% 120%  | 8, 64, 126      | 08407E   | 100, 62, 0, 30
\definecolor{ETHBlue120}{HTML}{08407E}

% ==============================================================
% 2. ETH Petrol Shades
% ==============================================================
% Shade | RGB             | HEX      | CMYK
% ------|-----------------|----------|---------------
% 10%   | 231, 244, 247   | E7F4F7   | 12, 0, 5, 0
\definecolor{ETHPetrol10}{HTML}{E7F4F7}
% 20%   | 204, 228, 234   | CCE4EA   | 20, 3, 7, 0
\definecolor{ETHPetrol20}{HTML}{CCE4EA}
% 40%   | 153, 202, 213   | 99CAD5   | 40, 7, 12, 4
\definecolor{ETHPetrol40}{HTML}{99CAD5}
% 60%   | 102, 175, 192   | 66AFC0   | 60, 14, 18, 6
\definecolor{ETHPetrol60}{HTML}{66AFC0}
% 80%   | 51, 149, 171    | 3395AB   | 80, 20, 24, 8
\definecolor{ETHPetrol80}{HTML}{3395AB}
% 120%  | 0, 89, 109      | 00596D   | 100, 25, 30, 38
\definecolor{ETHPetrol120}{HTML}{00596D}

% ==============================================================
% 3. ETH Green Shades
% ==============================================================
% Shade | RGB             | HEX      | CMYK
% ------|-----------------|----------|---------------
% 10%   | 239, 241, 231   | EEF1E7   | 6, 1, 10, 3
\definecolor{ETHGreen10}{HTML}{EEF1E7}
% 20%   | 224, 227, 208   | E0E3D0   | 11, 2, 20, 6
\definecolor{ETHGreen20}{HTML}{E0E3D0}
% 40%   | 192, 199, 161   | C0C7A1   | 22, 4, 40, 12
\definecolor{ETHGreen40}{HTML}{C0C7A1}
% 60%   | 161, 171, 113   | A1AB71   | 33, 6, 60, 18
\definecolor{ETHGreen60}{HTML}{A1AB71}
% 80%   | 129, 143, 66    | 818F42   | 44, 8, 80, 24
\definecolor{ETHGreen80}{HTML}{818F42}
% 120%  | 54, 82, 19      | 365213   | 55, 10, 100, 65
\definecolor{ETHGreen120}{HTML}{365213}

% ==============================================================
% 4. ETH Bronze Shades
% ==============================================================
% Shade | RGB             | HEX      | CMYK
% ------|-----------------|----------|---------------
% 10%   | 244, 240, 231   | F4F0E7   | 3, 4, 10, 3
\definecolor{ETHBronze10}{HTML}{F4F0E7}
% 20%   | 232, 225, 208   | E8E1D0   | 6, 7, 20, 5
\definecolor{ETHBronze20}{HTML}{E8E1D0}
% 40%   | 210, 194, 161   | D2C2A1   | 12, 14, 40, 10
\definecolor{ETHBronze40}{HTML}{D2C2A1}
% 60%   | 187, 164, 113   | BBA471   | 18, 22, 60, 15
\definecolor{ETHBronze60}{HTML}{BBA471}
% 80%   | 165, 133, 66    | A58542   | 24, 29, 80, 20
\definecolor{ETHBronze80}{HTML}{A58542}
% 120%  | 112, 79, 18     | 704F12   | 30, 36, 100, 55
\definecolor{ETHBronze120}{HTML}{704F12}

% ==============================================================
% 5. ETH Red Shades
% ==============================================================
% Shade | RGB             | HEX      | CMYK
% ------|-----------------|----------|---------------
% 10%   | 248, 235, 234   | F8EBEA   | 0, 9, 6, 0
\definecolor{ETHRed10}{HTML}{F8EBEA}
% 20%   | 241, 215, 213   | F1D7D5   | 0, 18, 13, 4
\definecolor{ETHRed20}{HTML}{F1D7D5}
% 40%   | 226, 174, 171   | E2AEAB   | 0, 36, 26, 8
\definecolor{ETHRed40}{HTML}{E2AEAB}
% 60%   | 212, 134, 129   | D48681   | 0, 54, 39, 11
\definecolor{ETHRed60}{HTML}{D48681}
% 80%   | 197, 93, 87     | C55D57   | (using HEX)
\definecolor{ETHRed80}{HTML}{C55D57}
% 120%  | 150, 39, 45     | 96272D   | 0, 100, 80, 40
\definecolor{ETHRed120}{HTML}{96272D}

% ==============================================================
% 6. ETH Purple Shades
% ==============================================================
% Shade | RGB             | HEX      | CMYK
% ------|-----------------|----------|---------------
% 10%   | 248, 232, 243   | F8E8F3   | 2, 10, 0, 1
\definecolor{ETHPurple10}{HTML}{F8E8F3}
% 20%   | 239, 208, 227   | EFD0E3   | 4, 20, 0, 1
\definecolor{ETHPurple20}{HTML}{EFD0E3}
% 40%   | 220, 158, 201   | DC9EC9   | 7, 40, 0, 4
\definecolor{ETHPurple40}{HTML}{DC9EC9}
% 60%   | 202, 108, 174   | CA6CAE   | 13, 60, 0, 6
\definecolor{ETHPurple60}{HTML}{CA6CAE}
% 80%   | 183, 59, 146    | B73B92   | 18, 80, 0, 8
\definecolor{ETHPurple80}{HTML}{B73B92}
% 120%  | 140, 10, 89     | 8C0A59   | 22, 100, 0, 35
\definecolor{ETHPurple120}{HTML}{8C0A59}

% ==============================================================
% 7. ETH Grey Shades
% ==============================================================
% Shade | RGB             | HEX      | CMYK
% ------|-----------------|----------|---------------
% 10%   | 241, 241, 241   | F1F1F1   | 0, 0, 0, 7
\definecolor{ETHGrey10}{HTML}{F1F1F1}
% 20%   | 226, 226, 226   | E2E2E2   | 0, 0, 0, 14
\definecolor{ETHGrey20}{HTML}{E2E2E2}
% 40%   | 197, 197, 197   | C5C5C5   | 0, 0, 0, 28
\definecolor{ETHGrey40}{HTML}{C5C5C5}
% 60%   | 169, 169, 169   | A9A9A9   | 0, 0, 0, 42
\definecolor{ETHGrey60}{HTML}{A9A9A9}
% 80%   | 140, 140, 140   | 8C8C8C   | 0, 0, 0, 56
\definecolor{ETHGrey80}{HTML}{8C8C8C}
% 120%  | 87, 87, 87      | 575757   | 0, 0, 0, 81
\definecolor{ETHGrey120}{HTML}{575757}
%   \textcolor{ETHBlue}{Hello ETH!}
% Usage in your main .tex:
%   \usepackage{xcolor} % or colortbl, etc.
%   % eth.tex
% Defines ETH brand colors based on:
% https://ethz.ch/staffnet/en/service/communication/corporate-design/colours.html
% 
% ------------------------------------------------------------------
% ETH Corporate Design – Primary Colors and Colour Shades Definitions
%
% PRIMARY ETH CORPORATE COLORS
%
% Colour       RGB             HEX       CMYK                 Pantone    RAL
% ----------------------------------------------------------------------------
% ETH Blue     33, 92, 175     #215CAF   100,57,0,0           2935       5005 Signalblau
% ETH Petrol   0, 120, 148     #007894   100,25,30,10         633        5009 Azurblau
% ETH Green    98, 115, 19     #627313   55,10,100,30         364        6010 Grasgrün
% ETH Bronze   142, 103, 19    #8E6713   30,36,100,25         4495       7008 Khakigrau
% ETH Red      183, 53, 45     #B7352D   0,90,80,17           1797       3031 Orientrot
% ETH Purple   167, 17, 122    #A7117A   22,100,0,10          234        4006 Verkehrspurpur
% ETH Grey     111, 111, 111   #6F6F6F   0,0,0,70             Cool Gray 11   7046 Telegrau 2
% ------------------------------------------------------------------
% ETH Corporate Design – Colour Shades Definitions
%
% 1. ETH Blue Shades
% 2. ETH Petrol Shades
% 3. ETH Green Shades
% 4. ETH Bronze Shades
% 5. ETH Red Shades
% 6. ETH Purple Shades
% 7. ETH Grey Shades
% 
% Last updated: 2025-03-08
%
% Usage:
%   \input{eth.tex}
%   \textcolor{ETHBlue}{Hello ETH!}
% Usage in your main .tex:
%   \usepackage{xcolor} % or colortbl, etc.
%   \input{eth.tex}
%   \textcolor{ETHBlue}{Hello from ETH!}
\NeedsTeXFormat{LaTeX2e}
\ProvidesFile{eth.tex}[2025/03/08 v1.0 ETH brand color definitions]

\RequirePackage{xcolor}

% ==============================================================
% Primary ETH Corporate Colors
% ==============================================================
\definecolor{ETHBlue}{HTML}{215CAF}    % ETH Blue: RGB: 33, 92, 175; CMYK: 100,57,0,0; Pantone: 2935; RAL: 5005 Signalblau
\definecolor{ETHPetrol}{HTML}{007894}   % ETH Petrol: RGB: 0,120,148; CMYK: 100,25,30,10; Pantone: 633; RAL: 5009 Azurblau
\definecolor{ETHGreen}{HTML}{627313}    % ETH Green: RGB: 98,115,19; CMYK: 55,10,100,30; Pantone: 364; RAL: 6010 Grasgrün
\definecolor{ETHBronze}{HTML}{8E6713}    % ETH Bronze: RGB: 142,103,19; CMYK: 30,36,100,25; Pantone: 4495; RAL: 7008 Khakigrau
\definecolor{ETHRed}{HTML}{B7352D}       % ETH Red: RGB: 183,53,45; CMYK: 0,90,80,17; Pantone: 1797; RAL: 3031 Orientrot
\definecolor{ETHPurple}{HTML}{A7117A}     % ETH Purple: RGB: 167,17,122; CMYK: 22,100,0,10; Pantone: 234; RAL: 4006 Verkehrspurpur
\definecolor{ETHGrey}{HTML}{6F6F6F}       % ETH Grey: RGB: 111,111,111; CMYK: 0,0,0,70; Pantone: Cool Gray 11; RAL: 7046 Telegrau 2


% ------------------------------------------------------------------
% Extended / Complementary Color Palette
% ------------------------------------------------------------------
\definecolor{ETHTeal}{HTML}{008C95}
\definecolor{ETHGreen}{HTML}{00B38B}
\definecolor{ETHDarkBlue}{HTML}{1D2447}
\definecolor{ETHLightBlue}{HTML}{5BB6D6}
\definecolor{ETHOrange}{HTML}{F39200}
\definecolor{ETHRed}{HTML}{C8002A}
\definecolor{ETHWarmGray}{HTML}{DAD7D2}
\definecolor{ETHBeige}{HTML}{D7CEC1}
\definecolor{ETHDarkBrown}{HTML}{7F4F3C}
\definecolor{ETHDarkPink}{HTML}{EB67BD}
\definecolor{ETHDarkPurple}{HTML}{5F2167}
\definecolor{ETHDarkMagenta}{HTML}{A3488E}
\definecolor{ETHDarkGray}{HTML}{333333}
\definecolor{ETHGray}{HTML}{75787B}
\definecolor{ETHLightGray}{HTML}{E2E2E2}
\definecolor{ETHWhite}{HTML}{FFFFFF}
\definecolor{ETHBlack}{HTML}{000000}

% ==============================================================
% 1. ETH Blue Shades
% ==============================================================
% Shade | RGB             | HEX      | CMYK
% ------|-----------------|----------|---------------
% 10%   | 233, 239, 247   | E9EFF7   | 10, 6, 0, 0
\definecolor{ETHBlue10}{HTML}{E9EFF7}
% 20%   | 211, 222, 239   | D3DEEF   | 20, 11, 0, 0
\definecolor{ETHBlue20}{HTML}{D3DEEF}
% 40%   | 166, 190, 223   | A6BEDF   | 40, 23, 0, 0
\definecolor{ETHBlue40}{HTML}{A6BEDF}
% 60%   | 122, 157, 207   | 7A9DCF   | 60, 34, 0, 0
\definecolor{ETHBlue60}{HTML}{7A9DCF}
% 80%   | 77, 125, 191    | 4D7DBF   | 80, 46, 0, 0
\definecolor{ETHBlue80}{HTML}{4D7DBF}
% 120%  | 8, 64, 126      | 08407E   | 100, 62, 0, 30
\definecolor{ETHBlue120}{HTML}{08407E}

% ==============================================================
% 2. ETH Petrol Shades
% ==============================================================
% Shade | RGB             | HEX      | CMYK
% ------|-----------------|----------|---------------
% 10%   | 231, 244, 247   | E7F4F7   | 12, 0, 5, 0
\definecolor{ETHPetrol10}{HTML}{E7F4F7}
% 20%   | 204, 228, 234   | CCE4EA   | 20, 3, 7, 0
\definecolor{ETHPetrol20}{HTML}{CCE4EA}
% 40%   | 153, 202, 213   | 99CAD5   | 40, 7, 12, 4
\definecolor{ETHPetrol40}{HTML}{99CAD5}
% 60%   | 102, 175, 192   | 66AFC0   | 60, 14, 18, 6
\definecolor{ETHPetrol60}{HTML}{66AFC0}
% 80%   | 51, 149, 171    | 3395AB   | 80, 20, 24, 8
\definecolor{ETHPetrol80}{HTML}{3395AB}
% 120%  | 0, 89, 109      | 00596D   | 100, 25, 30, 38
\definecolor{ETHPetrol120}{HTML}{00596D}

% ==============================================================
% 3. ETH Green Shades
% ==============================================================
% Shade | RGB             | HEX      | CMYK
% ------|-----------------|----------|---------------
% 10%   | 239, 241, 231   | EEF1E7   | 6, 1, 10, 3
\definecolor{ETHGreen10}{HTML}{EEF1E7}
% 20%   | 224, 227, 208   | E0E3D0   | 11, 2, 20, 6
\definecolor{ETHGreen20}{HTML}{E0E3D0}
% 40%   | 192, 199, 161   | C0C7A1   | 22, 4, 40, 12
\definecolor{ETHGreen40}{HTML}{C0C7A1}
% 60%   | 161, 171, 113   | A1AB71   | 33, 6, 60, 18
\definecolor{ETHGreen60}{HTML}{A1AB71}
% 80%   | 129, 143, 66    | 818F42   | 44, 8, 80, 24
\definecolor{ETHGreen80}{HTML}{818F42}
% 120%  | 54, 82, 19      | 365213   | 55, 10, 100, 65
\definecolor{ETHGreen120}{HTML}{365213}

% ==============================================================
% 4. ETH Bronze Shades
% ==============================================================
% Shade | RGB             | HEX      | CMYK
% ------|-----------------|----------|---------------
% 10%   | 244, 240, 231   | F4F0E7   | 3, 4, 10, 3
\definecolor{ETHBronze10}{HTML}{F4F0E7}
% 20%   | 232, 225, 208   | E8E1D0   | 6, 7, 20, 5
\definecolor{ETHBronze20}{HTML}{E8E1D0}
% 40%   | 210, 194, 161   | D2C2A1   | 12, 14, 40, 10
\definecolor{ETHBronze40}{HTML}{D2C2A1}
% 60%   | 187, 164, 113   | BBA471   | 18, 22, 60, 15
\definecolor{ETHBronze60}{HTML}{BBA471}
% 80%   | 165, 133, 66    | A58542   | 24, 29, 80, 20
\definecolor{ETHBronze80}{HTML}{A58542}
% 120%  | 112, 79, 18     | 704F12   | 30, 36, 100, 55
\definecolor{ETHBronze120}{HTML}{704F12}

% ==============================================================
% 5. ETH Red Shades
% ==============================================================
% Shade | RGB             | HEX      | CMYK
% ------|-----------------|----------|---------------
% 10%   | 248, 235, 234   | F8EBEA   | 0, 9, 6, 0
\definecolor{ETHRed10}{HTML}{F8EBEA}
% 20%   | 241, 215, 213   | F1D7D5   | 0, 18, 13, 4
\definecolor{ETHRed20}{HTML}{F1D7D5}
% 40%   | 226, 174, 171   | E2AEAB   | 0, 36, 26, 8
\definecolor{ETHRed40}{HTML}{E2AEAB}
% 60%   | 212, 134, 129   | D48681   | 0, 54, 39, 11
\definecolor{ETHRed60}{HTML}{D48681}
% 80%   | 197, 93, 87     | C55D57   | (using HEX)
\definecolor{ETHRed80}{HTML}{C55D57}
% 120%  | 150, 39, 45     | 96272D   | 0, 100, 80, 40
\definecolor{ETHRed120}{HTML}{96272D}

% ==============================================================
% 6. ETH Purple Shades
% ==============================================================
% Shade | RGB             | HEX      | CMYK
% ------|-----------------|----------|---------------
% 10%   | 248, 232, 243   | F8E8F3   | 2, 10, 0, 1
\definecolor{ETHPurple10}{HTML}{F8E8F3}
% 20%   | 239, 208, 227   | EFD0E3   | 4, 20, 0, 1
\definecolor{ETHPurple20}{HTML}{EFD0E3}
% 40%   | 220, 158, 201   | DC9EC9   | 7, 40, 0, 4
\definecolor{ETHPurple40}{HTML}{DC9EC9}
% 60%   | 202, 108, 174   | CA6CAE   | 13, 60, 0, 6
\definecolor{ETHPurple60}{HTML}{CA6CAE}
% 80%   | 183, 59, 146    | B73B92   | 18, 80, 0, 8
\definecolor{ETHPurple80}{HTML}{B73B92}
% 120%  | 140, 10, 89     | 8C0A59   | 22, 100, 0, 35
\definecolor{ETHPurple120}{HTML}{8C0A59}

% ==============================================================
% 7. ETH Grey Shades
% ==============================================================
% Shade | RGB             | HEX      | CMYK
% ------|-----------------|----------|---------------
% 10%   | 241, 241, 241   | F1F1F1   | 0, 0, 0, 7
\definecolor{ETHGrey10}{HTML}{F1F1F1}
% 20%   | 226, 226, 226   | E2E2E2   | 0, 0, 0, 14
\definecolor{ETHGrey20}{HTML}{E2E2E2}
% 40%   | 197, 197, 197   | C5C5C5   | 0, 0, 0, 28
\definecolor{ETHGrey40}{HTML}{C5C5C5}
% 60%   | 169, 169, 169   | A9A9A9   | 0, 0, 0, 42
\definecolor{ETHGrey60}{HTML}{A9A9A9}
% 80%   | 140, 140, 140   | 8C8C8C   | 0, 0, 0, 56
\definecolor{ETHGrey80}{HTML}{8C8C8C}
% 120%  | 87, 87, 87      | 575757   | 0, 0, 0, 81
\definecolor{ETHGrey120}{HTML}{575757}
%   \textcolor{ETHBlue}{Hello from ETH!}
\NeedsTeXFormat{LaTeX2e}
\ProvidesFile{eth.tex}[2025/03/08 v1.0 ETH brand color definitions]

\RequirePackage{xcolor}

% ==============================================================
% Primary ETH Corporate Colors
% ==============================================================
\definecolor{ETHBlue}{HTML}{215CAF}    % ETH Blue: RGB: 33, 92, 175; CMYK: 100,57,0,0; Pantone: 2935; RAL: 5005 Signalblau
\definecolor{ETHPetrol}{HTML}{007894}   % ETH Petrol: RGB: 0,120,148; CMYK: 100,25,30,10; Pantone: 633; RAL: 5009 Azurblau
\definecolor{ETHGreen}{HTML}{627313}    % ETH Green: RGB: 98,115,19; CMYK: 55,10,100,30; Pantone: 364; RAL: 6010 Grasgrün
\definecolor{ETHBronze}{HTML}{8E6713}    % ETH Bronze: RGB: 142,103,19; CMYK: 30,36,100,25; Pantone: 4495; RAL: 7008 Khakigrau
\definecolor{ETHRed}{HTML}{B7352D}       % ETH Red: RGB: 183,53,45; CMYK: 0,90,80,17; Pantone: 1797; RAL: 3031 Orientrot
\definecolor{ETHPurple}{HTML}{A7117A}     % ETH Purple: RGB: 167,17,122; CMYK: 22,100,0,10; Pantone: 234; RAL: 4006 Verkehrspurpur
\definecolor{ETHGrey}{HTML}{6F6F6F}       % ETH Grey: RGB: 111,111,111; CMYK: 0,0,0,70; Pantone: Cool Gray 11; RAL: 7046 Telegrau 2


% ------------------------------------------------------------------
% Extended / Complementary Color Palette
% ------------------------------------------------------------------
\definecolor{ETHTeal}{HTML}{008C95}
\definecolor{ETHGreen}{HTML}{00B38B}
\definecolor{ETHDarkBlue}{HTML}{1D2447}
\definecolor{ETHLightBlue}{HTML}{5BB6D6}
\definecolor{ETHOrange}{HTML}{F39200}
\definecolor{ETHRed}{HTML}{C8002A}
\definecolor{ETHWarmGray}{HTML}{DAD7D2}
\definecolor{ETHBeige}{HTML}{D7CEC1}
\definecolor{ETHDarkBrown}{HTML}{7F4F3C}
\definecolor{ETHDarkPink}{HTML}{EB67BD}
\definecolor{ETHDarkPurple}{HTML}{5F2167}
\definecolor{ETHDarkMagenta}{HTML}{A3488E}
\definecolor{ETHDarkGray}{HTML}{333333}
\definecolor{ETHGray}{HTML}{75787B}
\definecolor{ETHLightGray}{HTML}{E2E2E2}
\definecolor{ETHWhite}{HTML}{FFFFFF}
\definecolor{ETHBlack}{HTML}{000000}

% ==============================================================
% 1. ETH Blue Shades
% ==============================================================
% Shade | RGB             | HEX      | CMYK
% ------|-----------------|----------|---------------
% 10%   | 233, 239, 247   | E9EFF7   | 10, 6, 0, 0
\definecolor{ETHBlue10}{HTML}{E9EFF7}
% 20%   | 211, 222, 239   | D3DEEF   | 20, 11, 0, 0
\definecolor{ETHBlue20}{HTML}{D3DEEF}
% 40%   | 166, 190, 223   | A6BEDF   | 40, 23, 0, 0
\definecolor{ETHBlue40}{HTML}{A6BEDF}
% 60%   | 122, 157, 207   | 7A9DCF   | 60, 34, 0, 0
\definecolor{ETHBlue60}{HTML}{7A9DCF}
% 80%   | 77, 125, 191    | 4D7DBF   | 80, 46, 0, 0
\definecolor{ETHBlue80}{HTML}{4D7DBF}
% 120%  | 8, 64, 126      | 08407E   | 100, 62, 0, 30
\definecolor{ETHBlue120}{HTML}{08407E}

% ==============================================================
% 2. ETH Petrol Shades
% ==============================================================
% Shade | RGB             | HEX      | CMYK
% ------|-----------------|----------|---------------
% 10%   | 231, 244, 247   | E7F4F7   | 12, 0, 5, 0
\definecolor{ETHPetrol10}{HTML}{E7F4F7}
% 20%   | 204, 228, 234   | CCE4EA   | 20, 3, 7, 0
\definecolor{ETHPetrol20}{HTML}{CCE4EA}
% 40%   | 153, 202, 213   | 99CAD5   | 40, 7, 12, 4
\definecolor{ETHPetrol40}{HTML}{99CAD5}
% 60%   | 102, 175, 192   | 66AFC0   | 60, 14, 18, 6
\definecolor{ETHPetrol60}{HTML}{66AFC0}
% 80%   | 51, 149, 171    | 3395AB   | 80, 20, 24, 8
\definecolor{ETHPetrol80}{HTML}{3395AB}
% 120%  | 0, 89, 109      | 00596D   | 100, 25, 30, 38
\definecolor{ETHPetrol120}{HTML}{00596D}

% ==============================================================
% 3. ETH Green Shades
% ==============================================================
% Shade | RGB             | HEX      | CMYK
% ------|-----------------|----------|---------------
% 10%   | 239, 241, 231   | EEF1E7   | 6, 1, 10, 3
\definecolor{ETHGreen10}{HTML}{EEF1E7}
% 20%   | 224, 227, 208   | E0E3D0   | 11, 2, 20, 6
\definecolor{ETHGreen20}{HTML}{E0E3D0}
% 40%   | 192, 199, 161   | C0C7A1   | 22, 4, 40, 12
\definecolor{ETHGreen40}{HTML}{C0C7A1}
% 60%   | 161, 171, 113   | A1AB71   | 33, 6, 60, 18
\definecolor{ETHGreen60}{HTML}{A1AB71}
% 80%   | 129, 143, 66    | 818F42   | 44, 8, 80, 24
\definecolor{ETHGreen80}{HTML}{818F42}
% 120%  | 54, 82, 19      | 365213   | 55, 10, 100, 65
\definecolor{ETHGreen120}{HTML}{365213}

% ==============================================================
% 4. ETH Bronze Shades
% ==============================================================
% Shade | RGB             | HEX      | CMYK
% ------|-----------------|----------|---------------
% 10%   | 244, 240, 231   | F4F0E7   | 3, 4, 10, 3
\definecolor{ETHBronze10}{HTML}{F4F0E7}
% 20%   | 232, 225, 208   | E8E1D0   | 6, 7, 20, 5
\definecolor{ETHBronze20}{HTML}{E8E1D0}
% 40%   | 210, 194, 161   | D2C2A1   | 12, 14, 40, 10
\definecolor{ETHBronze40}{HTML}{D2C2A1}
% 60%   | 187, 164, 113   | BBA471   | 18, 22, 60, 15
\definecolor{ETHBronze60}{HTML}{BBA471}
% 80%   | 165, 133, 66    | A58542   | 24, 29, 80, 20
\definecolor{ETHBronze80}{HTML}{A58542}
% 120%  | 112, 79, 18     | 704F12   | 30, 36, 100, 55
\definecolor{ETHBronze120}{HTML}{704F12}

% ==============================================================
% 5. ETH Red Shades
% ==============================================================
% Shade | RGB             | HEX      | CMYK
% ------|-----------------|----------|---------------
% 10%   | 248, 235, 234   | F8EBEA   | 0, 9, 6, 0
\definecolor{ETHRed10}{HTML}{F8EBEA}
% 20%   | 241, 215, 213   | F1D7D5   | 0, 18, 13, 4
\definecolor{ETHRed20}{HTML}{F1D7D5}
% 40%   | 226, 174, 171   | E2AEAB   | 0, 36, 26, 8
\definecolor{ETHRed40}{HTML}{E2AEAB}
% 60%   | 212, 134, 129   | D48681   | 0, 54, 39, 11
\definecolor{ETHRed60}{HTML}{D48681}
% 80%   | 197, 93, 87     | C55D57   | (using HEX)
\definecolor{ETHRed80}{HTML}{C55D57}
% 120%  | 150, 39, 45     | 96272D   | 0, 100, 80, 40
\definecolor{ETHRed120}{HTML}{96272D}

% ==============================================================
% 6. ETH Purple Shades
% ==============================================================
% Shade | RGB             | HEX      | CMYK
% ------|-----------------|----------|---------------
% 10%   | 248, 232, 243   | F8E8F3   | 2, 10, 0, 1
\definecolor{ETHPurple10}{HTML}{F8E8F3}
% 20%   | 239, 208, 227   | EFD0E3   | 4, 20, 0, 1
\definecolor{ETHPurple20}{HTML}{EFD0E3}
% 40%   | 220, 158, 201   | DC9EC9   | 7, 40, 0, 4
\definecolor{ETHPurple40}{HTML}{DC9EC9}
% 60%   | 202, 108, 174   | CA6CAE   | 13, 60, 0, 6
\definecolor{ETHPurple60}{HTML}{CA6CAE}
% 80%   | 183, 59, 146    | B73B92   | 18, 80, 0, 8
\definecolor{ETHPurple80}{HTML}{B73B92}
% 120%  | 140, 10, 89     | 8C0A59   | 22, 100, 0, 35
\definecolor{ETHPurple120}{HTML}{8C0A59}

% ==============================================================
% 7. ETH Grey Shades
% ==============================================================
% Shade | RGB             | HEX      | CMYK
% ------|-----------------|----------|---------------
% 10%   | 241, 241, 241   | F1F1F1   | 0, 0, 0, 7
\definecolor{ETHGrey10}{HTML}{F1F1F1}
% 20%   | 226, 226, 226   | E2E2E2   | 0, 0, 0, 14
\definecolor{ETHGrey20}{HTML}{E2E2E2}
% 40%   | 197, 197, 197   | C5C5C5   | 0, 0, 0, 28
\definecolor{ETHGrey40}{HTML}{C5C5C5}
% 60%   | 169, 169, 169   | A9A9A9   | 0, 0, 0, 42
\definecolor{ETHGrey60}{HTML}{A9A9A9}
% 80%   | 140, 140, 140   | 8C8C8C   | 0, 0, 0, 56
\definecolor{ETHGrey80}{HTML}{8C8C8C}
% 120%  | 87, 87, 87      | 575757   | 0, 0, 0, 81
\definecolor{ETHGrey120}{HTML}{575757}  % Load ETH corporate colours and shade definitions

\colorlet{CTurl}{ETHBlue}      % Override CTcitation with ETHBlue
\colorlet{CTtitle}{ETHBlue}      % Override CTcitation with ETHBlue


% Additional general configurations (packages, macros, etc.) can be added below.


% Biblatex
% \usepackage[
%   style=nature,%
%   %style=science, article-title=true,%
%   natbib=true,%
%   clearlang=true,%
%   backend=biber,%
% ]{biblatex}

% Add this line to suppress the split bibliography warning
\BiblatexSplitbibDefernumbersWarningOff

% https://mirrors.ibiblio.org/CTAN/macros/latex/contrib/biblatex/doc/biblatex.pdf
\ExecuteBibliographyOptions{%
  %--- Backend --- --- ---
  bibwarn=true, %
  bibencoding=auto, % (ascii, inputenc, <encoding>)
  %--- Sorting --- --- ---
  sorting=none, % (bib, los) The sorting order of the list of shorthands =nty, ntd, nyt, ndt, nyvt, ndvt, anyt, andt, anyvt, optandvt, ynt, dnt, ydnt, ddnt, none, count, debug,
  % other options: 
  % nty        Sort by name, title, year.
  % nyt        Sort by name, year, title.
  % nyvt       Sort by name, year, volume, title.
  % anyt       Sort by alphabetic label, name, year, title.
  % anyvt      Sort by alphabetic label, name, year, volume, title.
  % ynt        Sort by year, name, title.
  % ydnt       Sort by year (descending), name, title.
  % none       Do not sort at all. All entries are processed in citation order.
  % debug      Sort by entry key. This is intended for debugging only.
  %
  sortcase=true,
  sortcites=true, % do/do not sort citations according to bib	
  %--- Dates --- --- ---
  date=comp,  % (short, long, terse, comp, iso8601)
  %	origdate=
  %	eventdate=
  %	urldate=
  %	alldates=
  datezeros=true, %
  dateabbrev=true, %
  %--- General Options --- --- ---
  maxnames=3,
  minnames=1,
  maxbibnames=100, % do not abbreviate names in bibliography
  %	autocite= % (plain, inline, footnote, superscript) 
  autopunct=true,
  language=auto,
  autolang=none, % (none, hyphen, other, other*)
  block=none, % (none, space, par, nbpar, ragged)
  notetype=foot+end, % (foot+end, footonly, endonly)
  hyperref=true, % (true, false, auto)
  backref=false,
  backrefstyle=three, % (none, three, two, two+, three+, all+)
  backrefsetstyle=setonly, %
  indexing=false, % 
  % options:
  % true       Enable indexing globally.
  % false      Disable indexing globally.
  % cite       Enable indexing in citations only.
  % bib        Enable indexing in the bibliography only.
  refsection=chapter, % (none, part, chapter, section, subsection)
  refsegment=none, % (none, part, chapter, section, subsection)
  abbreviate=true, % (true, false)
  defernumbers=false, % 
  punctfont=false, % 
  arxiv=abs, % (ps, pdf, format)	
  %--- Style Options --- --- ---	
  isbn=false,%
  url=false,%
  doi=false,%
  eprint=false,%	
}%	

% Suppress all date fields except the year
\AtEveryBibitem{%
  \clearfield{day}%
  \clearfield{month}%
  \clearfield{endday}%
  \clearfield{endmonth}%
}

\DeclareRedundantLanguages{en,EN,English}{english}

% Use only the first page number in a given range
\DeclareFieldFormat{pages}{\mkfirstpage{#1}}


\ExplSyntaxOn
% Define a function for string substitution
\cs_new:Npn \minna_replace:nn #1 #2
  {
    \tl_replace_all:Nnn \l_tmpa_tl {#1} {#2}
  }

% Wrapper macro for ease of use
\NewDocumentCommand{\ReplaceString}{ m m m }
  {
    \tl_set:Nn \l_tmpa_tl {#1}
    \minna_replace:nn {#2} {#3}
    \tl_use:N \l_tmpa_tl
  }
\ExplSyntaxOff



% ********************************************************************
% Fine-tune hyperreferences (hyperref should be called last)
% ********************************************************************

\usepackage[dvipsnames]{xcolor}


\PassOptionsToPackage{pdftex,hyperfootnotes=false,pdfpagelabels}{hyperref}
\usepackage{hyperref}  % backref linktocpage pagebackref
\pdfcompresslevel=9
\pdfadjustspacing=1

% \usepackage{hyperxmp}
%\pdfcompresslevel=9
%\pdfadjustspacing=1

% \usepackage{hyperref}  % backref linktocpage pagebackref

\hypersetup{%
  %draft, % hyperref's draft mode, for printing see below
  colorlinks=true, linktocpage=true, pdfstartpage=3, pdfstartview=FitV,%
  % uncomment the following line if you want to have black links (e.g., for printing)
  %colorlinks=false, linktocpage=false, pdfstartpage=3, pdfstartview=FitV, pdfborder={0 0 0},%
  breaklinks=true, pageanchor=true,%
  pdfpagemode=UseNone, %
  % pdfpagemode=UseOutlines,%
  plainpages=false, bookmarksnumbered, bookmarksopen=true, bookmarksopenlevel=1,%
  hypertexnames=true, pdfhighlight=/O,%nesting=true,%frenchlinks,%
  urlcolor=CTurl, linkcolor=CTlink, citecolor=CTcitation, %pagecolor=RoyalBlue,%
  %urlcolor=Black, linkcolor=Black, citecolor=Black, %pagecolor=Black,%
  pdftitle={\myTitle},%
  pdfauthor={\textcopyright\ \myName, \myUni, \myFaculty},%
  pdfsubject={},%
  pdfkeywords={},%
  pdfcreator={pdfLaTeX},%
  pdfproducer={LaTeX with hyperref and classicthesis}%
}


% ********************************************************************
% Setup autoreferences (hyperref and babel)
% ********************************************************************
% There are some issues regarding autorefnames
% http://www.tex.ac.uk/cgi-bin/texfaq2html?label=latexwords
% you have to redefine the macros for the
% language you use, e.g., american, ngerman
% (as chosen when loading babel/AtBeginDocument)
% ********************************************************************
 \makeatletter
 \@ifpackageloaded{babel}%
   {%
     \addto\extrasamerican{%
       \renewcommand*{\figureautorefname}{Figure}%
       \renewcommand*{\tableautorefname}{Table}%
       \renewcommand*{\partautorefname}{Part}%
       \renewcommand*{\chapterautorefname}{Chapter}%
       \renewcommand*{\sectionautorefname}{Section}%
       \renewcommand*{\subsectionautorefname}{Section}%
       \renewcommand*{\subsubsectionautorefname}{Section}%
     }%
     \addto\extrasngerman{%
       \renewcommand*{\paragraphautorefname}{Absatz}%
       \renewcommand*{\subparagraphautorefname}{Unterabsatz}%
       \renewcommand*{\footnoteautorefname}{Fu\"snote}%
       \renewcommand*{\FancyVerbLineautorefname}{Zeile}%
       \renewcommand*{\theoremautorefname}{Theorem}%
       \renewcommand*{\appendixautorefname}{Anhang}%
       \renewcommand*{\equationautorefname}{Gleichung}%
       \renewcommand*{\itemautorefname}{Punkt}%
     }%
       % Fix to getting autorefs for subfigures right (thanks to Belinda Vogt for changing the definition)
       \providecommand{\subfigureautorefname}{\figureautorefname}%
     }{\relax}
 \makeatother

% (Better) alternative to \autoref is \cref via the cleveref package
%\usepackage{cleveref}
%\crefformat{part}{Part #2\MakeUppercase{#1}#3}
% ------------------------------------------------------------------

% ****************************************************************************************************
% 1. Configure classicthesis for your needs here, e.g., remove "drafting" below
% in order to deactivate the time-stamp on the pages
% (see ClassicThesis.pdf for more information):
% ****************************************************************************************************
\PassOptionsToPackage{
  drafting=true,    % print version information on the bottom of the pages
  tocaligned=false, % the left column of the toc will be aligned (no indentation)
  dottedtoc=false,  % page numbers in ToC flushed right
  eulerchapternumbers=false, % use AMS Euler for chapter font (otherwise Palatino)
  floatperchapter=true,     % numbering per chapter for all floats (i.e., Figure 1.1)
  eulermath=false,  % use awesome Euler fonts for mathematical formulae (only with pdfLaTeX)
  beramono=true,    % toggle a nice monospaced font (w/ bold)
  palatino=true,    % deactivate standard font for loading another one, see the last section at the end of this file for suggestions
  %linedheaders=true, % obsolete / available for backwards compatibility
  style=classicthesis % classicthesis, arsclassica, linedheaders, plain
}{classicthesis}

% ****************************************************************************************************
% 2. Personal data and user ad-hoc commands (insert your own data here)
% ****************************************************************************************************



% % eth.tex
% Defines ETH brand colors based on:
% https://ethz.ch/staffnet/en/service/communication/corporate-design/colours.html
% 
% ------------------------------------------------------------------
% ETH Corporate Design – Primary Colors and Colour Shades Definitions
%
% PRIMARY ETH CORPORATE COLORS
%
% Colour       RGB             HEX       CMYK                 Pantone    RAL
% ----------------------------------------------------------------------------
% ETH Blue     33, 92, 175     #215CAF   100,57,0,0           2935       5005 Signalblau
% ETH Petrol   0, 120, 148     #007894   100,25,30,10         633        5009 Azurblau
% ETH Green    98, 115, 19     #627313   55,10,100,30         364        6010 Grasgrün
% ETH Bronze   142, 103, 19    #8E6713   30,36,100,25         4495       7008 Khakigrau
% ETH Red      183, 53, 45     #B7352D   0,90,80,17           1797       3031 Orientrot
% ETH Purple   167, 17, 122    #A7117A   22,100,0,10          234        4006 Verkehrspurpur
% ETH Grey     111, 111, 111   #6F6F6F   0,0,0,70             Cool Gray 11   7046 Telegrau 2
% ------------------------------------------------------------------
% ETH Corporate Design – Colour Shades Definitions
%
% 1. ETH Blue Shades
% 2. ETH Petrol Shades
% 3. ETH Green Shades
% 4. ETH Bronze Shades
% 5. ETH Red Shades
% 6. ETH Purple Shades
% 7. ETH Grey Shades
% 
% Last updated: 2025-03-08
%
% Usage:
%   % eth.tex
% Defines ETH brand colors based on:
% https://ethz.ch/staffnet/en/service/communication/corporate-design/colours.html
% 
% ------------------------------------------------------------------
% ETH Corporate Design – Primary Colors and Colour Shades Definitions
%
% PRIMARY ETH CORPORATE COLORS
%
% Colour       RGB             HEX       CMYK                 Pantone    RAL
% ----------------------------------------------------------------------------
% ETH Blue     33, 92, 175     #215CAF   100,57,0,0           2935       5005 Signalblau
% ETH Petrol   0, 120, 148     #007894   100,25,30,10         633        5009 Azurblau
% ETH Green    98, 115, 19     #627313   55,10,100,30         364        6010 Grasgrün
% ETH Bronze   142, 103, 19    #8E6713   30,36,100,25         4495       7008 Khakigrau
% ETH Red      183, 53, 45     #B7352D   0,90,80,17           1797       3031 Orientrot
% ETH Purple   167, 17, 122    #A7117A   22,100,0,10          234        4006 Verkehrspurpur
% ETH Grey     111, 111, 111   #6F6F6F   0,0,0,70             Cool Gray 11   7046 Telegrau 2
% ------------------------------------------------------------------
% ETH Corporate Design – Colour Shades Definitions
%
% 1. ETH Blue Shades
% 2. ETH Petrol Shades
% 3. ETH Green Shades
% 4. ETH Bronze Shades
% 5. ETH Red Shades
% 6. ETH Purple Shades
% 7. ETH Grey Shades
% 
% Last updated: 2025-03-08
%
% Usage:
%   % eth.tex
% Defines ETH brand colors based on:
% https://ethz.ch/staffnet/en/service/communication/corporate-design/colours.html
% 
% ------------------------------------------------------------------
% ETH Corporate Design – Primary Colors and Colour Shades Definitions
%
% PRIMARY ETH CORPORATE COLORS
%
% Colour       RGB             HEX       CMYK                 Pantone    RAL
% ----------------------------------------------------------------------------
% ETH Blue     33, 92, 175     #215CAF   100,57,0,0           2935       5005 Signalblau
% ETH Petrol   0, 120, 148     #007894   100,25,30,10         633        5009 Azurblau
% ETH Green    98, 115, 19     #627313   55,10,100,30         364        6010 Grasgrün
% ETH Bronze   142, 103, 19    #8E6713   30,36,100,25         4495       7008 Khakigrau
% ETH Red      183, 53, 45     #B7352D   0,90,80,17           1797       3031 Orientrot
% ETH Purple   167, 17, 122    #A7117A   22,100,0,10          234        4006 Verkehrspurpur
% ETH Grey     111, 111, 111   #6F6F6F   0,0,0,70             Cool Gray 11   7046 Telegrau 2
% ------------------------------------------------------------------
% ETH Corporate Design – Colour Shades Definitions
%
% 1. ETH Blue Shades
% 2. ETH Petrol Shades
% 3. ETH Green Shades
% 4. ETH Bronze Shades
% 5. ETH Red Shades
% 6. ETH Purple Shades
% 7. ETH Grey Shades
% 
% Last updated: 2025-03-08
%
% Usage:
%   \input{eth.tex}
%   \textcolor{ETHBlue}{Hello ETH!}
% Usage in your main .tex:
%   \usepackage{xcolor} % or colortbl, etc.
%   \input{eth.tex}
%   \textcolor{ETHBlue}{Hello from ETH!}
\NeedsTeXFormat{LaTeX2e}
\ProvidesFile{eth.tex}[2025/03/08 v1.0 ETH brand color definitions]

\RequirePackage{xcolor}

% ==============================================================
% Primary ETH Corporate Colors
% ==============================================================
\definecolor{ETHBlue}{HTML}{215CAF}    % ETH Blue: RGB: 33, 92, 175; CMYK: 100,57,0,0; Pantone: 2935; RAL: 5005 Signalblau
\definecolor{ETHPetrol}{HTML}{007894}   % ETH Petrol: RGB: 0,120,148; CMYK: 100,25,30,10; Pantone: 633; RAL: 5009 Azurblau
\definecolor{ETHGreen}{HTML}{627313}    % ETH Green: RGB: 98,115,19; CMYK: 55,10,100,30; Pantone: 364; RAL: 6010 Grasgrün
\definecolor{ETHBronze}{HTML}{8E6713}    % ETH Bronze: RGB: 142,103,19; CMYK: 30,36,100,25; Pantone: 4495; RAL: 7008 Khakigrau
\definecolor{ETHRed}{HTML}{B7352D}       % ETH Red: RGB: 183,53,45; CMYK: 0,90,80,17; Pantone: 1797; RAL: 3031 Orientrot
\definecolor{ETHPurple}{HTML}{A7117A}     % ETH Purple: RGB: 167,17,122; CMYK: 22,100,0,10; Pantone: 234; RAL: 4006 Verkehrspurpur
\definecolor{ETHGrey}{HTML}{6F6F6F}       % ETH Grey: RGB: 111,111,111; CMYK: 0,0,0,70; Pantone: Cool Gray 11; RAL: 7046 Telegrau 2


% ------------------------------------------------------------------
% Extended / Complementary Color Palette
% ------------------------------------------------------------------
\definecolor{ETHTeal}{HTML}{008C95}
\definecolor{ETHGreen}{HTML}{00B38B}
\definecolor{ETHDarkBlue}{HTML}{1D2447}
\definecolor{ETHLightBlue}{HTML}{5BB6D6}
\definecolor{ETHOrange}{HTML}{F39200}
\definecolor{ETHRed}{HTML}{C8002A}
\definecolor{ETHWarmGray}{HTML}{DAD7D2}
\definecolor{ETHBeige}{HTML}{D7CEC1}
\definecolor{ETHDarkBrown}{HTML}{7F4F3C}
\definecolor{ETHDarkPink}{HTML}{EB67BD}
\definecolor{ETHDarkPurple}{HTML}{5F2167}
\definecolor{ETHDarkMagenta}{HTML}{A3488E}
\definecolor{ETHDarkGray}{HTML}{333333}
\definecolor{ETHGray}{HTML}{75787B}
\definecolor{ETHLightGray}{HTML}{E2E2E2}
\definecolor{ETHWhite}{HTML}{FFFFFF}
\definecolor{ETHBlack}{HTML}{000000}

% ==============================================================
% 1. ETH Blue Shades
% ==============================================================
% Shade | RGB             | HEX      | CMYK
% ------|-----------------|----------|---------------
% 10%   | 233, 239, 247   | E9EFF7   | 10, 6, 0, 0
\definecolor{ETHBlue10}{HTML}{E9EFF7}
% 20%   | 211, 222, 239   | D3DEEF   | 20, 11, 0, 0
\definecolor{ETHBlue20}{HTML}{D3DEEF}
% 40%   | 166, 190, 223   | A6BEDF   | 40, 23, 0, 0
\definecolor{ETHBlue40}{HTML}{A6BEDF}
% 60%   | 122, 157, 207   | 7A9DCF   | 60, 34, 0, 0
\definecolor{ETHBlue60}{HTML}{7A9DCF}
% 80%   | 77, 125, 191    | 4D7DBF   | 80, 46, 0, 0
\definecolor{ETHBlue80}{HTML}{4D7DBF}
% 120%  | 8, 64, 126      | 08407E   | 100, 62, 0, 30
\definecolor{ETHBlue120}{HTML}{08407E}

% ==============================================================
% 2. ETH Petrol Shades
% ==============================================================
% Shade | RGB             | HEX      | CMYK
% ------|-----------------|----------|---------------
% 10%   | 231, 244, 247   | E7F4F7   | 12, 0, 5, 0
\definecolor{ETHPetrol10}{HTML}{E7F4F7}
% 20%   | 204, 228, 234   | CCE4EA   | 20, 3, 7, 0
\definecolor{ETHPetrol20}{HTML}{CCE4EA}
% 40%   | 153, 202, 213   | 99CAD5   | 40, 7, 12, 4
\definecolor{ETHPetrol40}{HTML}{99CAD5}
% 60%   | 102, 175, 192   | 66AFC0   | 60, 14, 18, 6
\definecolor{ETHPetrol60}{HTML}{66AFC0}
% 80%   | 51, 149, 171    | 3395AB   | 80, 20, 24, 8
\definecolor{ETHPetrol80}{HTML}{3395AB}
% 120%  | 0, 89, 109      | 00596D   | 100, 25, 30, 38
\definecolor{ETHPetrol120}{HTML}{00596D}

% ==============================================================
% 3. ETH Green Shades
% ==============================================================
% Shade | RGB             | HEX      | CMYK
% ------|-----------------|----------|---------------
% 10%   | 239, 241, 231   | EEF1E7   | 6, 1, 10, 3
\definecolor{ETHGreen10}{HTML}{EEF1E7}
% 20%   | 224, 227, 208   | E0E3D0   | 11, 2, 20, 6
\definecolor{ETHGreen20}{HTML}{E0E3D0}
% 40%   | 192, 199, 161   | C0C7A1   | 22, 4, 40, 12
\definecolor{ETHGreen40}{HTML}{C0C7A1}
% 60%   | 161, 171, 113   | A1AB71   | 33, 6, 60, 18
\definecolor{ETHGreen60}{HTML}{A1AB71}
% 80%   | 129, 143, 66    | 818F42   | 44, 8, 80, 24
\definecolor{ETHGreen80}{HTML}{818F42}
% 120%  | 54, 82, 19      | 365213   | 55, 10, 100, 65
\definecolor{ETHGreen120}{HTML}{365213}

% ==============================================================
% 4. ETH Bronze Shades
% ==============================================================
% Shade | RGB             | HEX      | CMYK
% ------|-----------------|----------|---------------
% 10%   | 244, 240, 231   | F4F0E7   | 3, 4, 10, 3
\definecolor{ETHBronze10}{HTML}{F4F0E7}
% 20%   | 232, 225, 208   | E8E1D0   | 6, 7, 20, 5
\definecolor{ETHBronze20}{HTML}{E8E1D0}
% 40%   | 210, 194, 161   | D2C2A1   | 12, 14, 40, 10
\definecolor{ETHBronze40}{HTML}{D2C2A1}
% 60%   | 187, 164, 113   | BBA471   | 18, 22, 60, 15
\definecolor{ETHBronze60}{HTML}{BBA471}
% 80%   | 165, 133, 66    | A58542   | 24, 29, 80, 20
\definecolor{ETHBronze80}{HTML}{A58542}
% 120%  | 112, 79, 18     | 704F12   | 30, 36, 100, 55
\definecolor{ETHBronze120}{HTML}{704F12}

% ==============================================================
% 5. ETH Red Shades
% ==============================================================
% Shade | RGB             | HEX      | CMYK
% ------|-----------------|----------|---------------
% 10%   | 248, 235, 234   | F8EBEA   | 0, 9, 6, 0
\definecolor{ETHRed10}{HTML}{F8EBEA}
% 20%   | 241, 215, 213   | F1D7D5   | 0, 18, 13, 4
\definecolor{ETHRed20}{HTML}{F1D7D5}
% 40%   | 226, 174, 171   | E2AEAB   | 0, 36, 26, 8
\definecolor{ETHRed40}{HTML}{E2AEAB}
% 60%   | 212, 134, 129   | D48681   | 0, 54, 39, 11
\definecolor{ETHRed60}{HTML}{D48681}
% 80%   | 197, 93, 87     | C55D57   | (using HEX)
\definecolor{ETHRed80}{HTML}{C55D57}
% 120%  | 150, 39, 45     | 96272D   | 0, 100, 80, 40
\definecolor{ETHRed120}{HTML}{96272D}

% ==============================================================
% 6. ETH Purple Shades
% ==============================================================
% Shade | RGB             | HEX      | CMYK
% ------|-----------------|----------|---------------
% 10%   | 248, 232, 243   | F8E8F3   | 2, 10, 0, 1
\definecolor{ETHPurple10}{HTML}{F8E8F3}
% 20%   | 239, 208, 227   | EFD0E3   | 4, 20, 0, 1
\definecolor{ETHPurple20}{HTML}{EFD0E3}
% 40%   | 220, 158, 201   | DC9EC9   | 7, 40, 0, 4
\definecolor{ETHPurple40}{HTML}{DC9EC9}
% 60%   | 202, 108, 174   | CA6CAE   | 13, 60, 0, 6
\definecolor{ETHPurple60}{HTML}{CA6CAE}
% 80%   | 183, 59, 146    | B73B92   | 18, 80, 0, 8
\definecolor{ETHPurple80}{HTML}{B73B92}
% 120%  | 140, 10, 89     | 8C0A59   | 22, 100, 0, 35
\definecolor{ETHPurple120}{HTML}{8C0A59}

% ==============================================================
% 7. ETH Grey Shades
% ==============================================================
% Shade | RGB             | HEX      | CMYK
% ------|-----------------|----------|---------------
% 10%   | 241, 241, 241   | F1F1F1   | 0, 0, 0, 7
\definecolor{ETHGrey10}{HTML}{F1F1F1}
% 20%   | 226, 226, 226   | E2E2E2   | 0, 0, 0, 14
\definecolor{ETHGrey20}{HTML}{E2E2E2}
% 40%   | 197, 197, 197   | C5C5C5   | 0, 0, 0, 28
\definecolor{ETHGrey40}{HTML}{C5C5C5}
% 60%   | 169, 169, 169   | A9A9A9   | 0, 0, 0, 42
\definecolor{ETHGrey60}{HTML}{A9A9A9}
% 80%   | 140, 140, 140   | 8C8C8C   | 0, 0, 0, 56
\definecolor{ETHGrey80}{HTML}{8C8C8C}
% 120%  | 87, 87, 87      | 575757   | 0, 0, 0, 81
\definecolor{ETHGrey120}{HTML}{575757}
%   \textcolor{ETHBlue}{Hello ETH!}
% Usage in your main .tex:
%   \usepackage{xcolor} % or colortbl, etc.
%   % eth.tex
% Defines ETH brand colors based on:
% https://ethz.ch/staffnet/en/service/communication/corporate-design/colours.html
% 
% ------------------------------------------------------------------
% ETH Corporate Design – Primary Colors and Colour Shades Definitions
%
% PRIMARY ETH CORPORATE COLORS
%
% Colour       RGB             HEX       CMYK                 Pantone    RAL
% ----------------------------------------------------------------------------
% ETH Blue     33, 92, 175     #215CAF   100,57,0,0           2935       5005 Signalblau
% ETH Petrol   0, 120, 148     #007894   100,25,30,10         633        5009 Azurblau
% ETH Green    98, 115, 19     #627313   55,10,100,30         364        6010 Grasgrün
% ETH Bronze   142, 103, 19    #8E6713   30,36,100,25         4495       7008 Khakigrau
% ETH Red      183, 53, 45     #B7352D   0,90,80,17           1797       3031 Orientrot
% ETH Purple   167, 17, 122    #A7117A   22,100,0,10          234        4006 Verkehrspurpur
% ETH Grey     111, 111, 111   #6F6F6F   0,0,0,70             Cool Gray 11   7046 Telegrau 2
% ------------------------------------------------------------------
% ETH Corporate Design – Colour Shades Definitions
%
% 1. ETH Blue Shades
% 2. ETH Petrol Shades
% 3. ETH Green Shades
% 4. ETH Bronze Shades
% 5. ETH Red Shades
% 6. ETH Purple Shades
% 7. ETH Grey Shades
% 
% Last updated: 2025-03-08
%
% Usage:
%   \input{eth.tex}
%   \textcolor{ETHBlue}{Hello ETH!}
% Usage in your main .tex:
%   \usepackage{xcolor} % or colortbl, etc.
%   \input{eth.tex}
%   \textcolor{ETHBlue}{Hello from ETH!}
\NeedsTeXFormat{LaTeX2e}
\ProvidesFile{eth.tex}[2025/03/08 v1.0 ETH brand color definitions]

\RequirePackage{xcolor}

% ==============================================================
% Primary ETH Corporate Colors
% ==============================================================
\definecolor{ETHBlue}{HTML}{215CAF}    % ETH Blue: RGB: 33, 92, 175; CMYK: 100,57,0,0; Pantone: 2935; RAL: 5005 Signalblau
\definecolor{ETHPetrol}{HTML}{007894}   % ETH Petrol: RGB: 0,120,148; CMYK: 100,25,30,10; Pantone: 633; RAL: 5009 Azurblau
\definecolor{ETHGreen}{HTML}{627313}    % ETH Green: RGB: 98,115,19; CMYK: 55,10,100,30; Pantone: 364; RAL: 6010 Grasgrün
\definecolor{ETHBronze}{HTML}{8E6713}    % ETH Bronze: RGB: 142,103,19; CMYK: 30,36,100,25; Pantone: 4495; RAL: 7008 Khakigrau
\definecolor{ETHRed}{HTML}{B7352D}       % ETH Red: RGB: 183,53,45; CMYK: 0,90,80,17; Pantone: 1797; RAL: 3031 Orientrot
\definecolor{ETHPurple}{HTML}{A7117A}     % ETH Purple: RGB: 167,17,122; CMYK: 22,100,0,10; Pantone: 234; RAL: 4006 Verkehrspurpur
\definecolor{ETHGrey}{HTML}{6F6F6F}       % ETH Grey: RGB: 111,111,111; CMYK: 0,0,0,70; Pantone: Cool Gray 11; RAL: 7046 Telegrau 2


% ------------------------------------------------------------------
% Extended / Complementary Color Palette
% ------------------------------------------------------------------
\definecolor{ETHTeal}{HTML}{008C95}
\definecolor{ETHGreen}{HTML}{00B38B}
\definecolor{ETHDarkBlue}{HTML}{1D2447}
\definecolor{ETHLightBlue}{HTML}{5BB6D6}
\definecolor{ETHOrange}{HTML}{F39200}
\definecolor{ETHRed}{HTML}{C8002A}
\definecolor{ETHWarmGray}{HTML}{DAD7D2}
\definecolor{ETHBeige}{HTML}{D7CEC1}
\definecolor{ETHDarkBrown}{HTML}{7F4F3C}
\definecolor{ETHDarkPink}{HTML}{EB67BD}
\definecolor{ETHDarkPurple}{HTML}{5F2167}
\definecolor{ETHDarkMagenta}{HTML}{A3488E}
\definecolor{ETHDarkGray}{HTML}{333333}
\definecolor{ETHGray}{HTML}{75787B}
\definecolor{ETHLightGray}{HTML}{E2E2E2}
\definecolor{ETHWhite}{HTML}{FFFFFF}
\definecolor{ETHBlack}{HTML}{000000}

% ==============================================================
% 1. ETH Blue Shades
% ==============================================================
% Shade | RGB             | HEX      | CMYK
% ------|-----------------|----------|---------------
% 10%   | 233, 239, 247   | E9EFF7   | 10, 6, 0, 0
\definecolor{ETHBlue10}{HTML}{E9EFF7}
% 20%   | 211, 222, 239   | D3DEEF   | 20, 11, 0, 0
\definecolor{ETHBlue20}{HTML}{D3DEEF}
% 40%   | 166, 190, 223   | A6BEDF   | 40, 23, 0, 0
\definecolor{ETHBlue40}{HTML}{A6BEDF}
% 60%   | 122, 157, 207   | 7A9DCF   | 60, 34, 0, 0
\definecolor{ETHBlue60}{HTML}{7A9DCF}
% 80%   | 77, 125, 191    | 4D7DBF   | 80, 46, 0, 0
\definecolor{ETHBlue80}{HTML}{4D7DBF}
% 120%  | 8, 64, 126      | 08407E   | 100, 62, 0, 30
\definecolor{ETHBlue120}{HTML}{08407E}

% ==============================================================
% 2. ETH Petrol Shades
% ==============================================================
% Shade | RGB             | HEX      | CMYK
% ------|-----------------|----------|---------------
% 10%   | 231, 244, 247   | E7F4F7   | 12, 0, 5, 0
\definecolor{ETHPetrol10}{HTML}{E7F4F7}
% 20%   | 204, 228, 234   | CCE4EA   | 20, 3, 7, 0
\definecolor{ETHPetrol20}{HTML}{CCE4EA}
% 40%   | 153, 202, 213   | 99CAD5   | 40, 7, 12, 4
\definecolor{ETHPetrol40}{HTML}{99CAD5}
% 60%   | 102, 175, 192   | 66AFC0   | 60, 14, 18, 6
\definecolor{ETHPetrol60}{HTML}{66AFC0}
% 80%   | 51, 149, 171    | 3395AB   | 80, 20, 24, 8
\definecolor{ETHPetrol80}{HTML}{3395AB}
% 120%  | 0, 89, 109      | 00596D   | 100, 25, 30, 38
\definecolor{ETHPetrol120}{HTML}{00596D}

% ==============================================================
% 3. ETH Green Shades
% ==============================================================
% Shade | RGB             | HEX      | CMYK
% ------|-----------------|----------|---------------
% 10%   | 239, 241, 231   | EEF1E7   | 6, 1, 10, 3
\definecolor{ETHGreen10}{HTML}{EEF1E7}
% 20%   | 224, 227, 208   | E0E3D0   | 11, 2, 20, 6
\definecolor{ETHGreen20}{HTML}{E0E3D0}
% 40%   | 192, 199, 161   | C0C7A1   | 22, 4, 40, 12
\definecolor{ETHGreen40}{HTML}{C0C7A1}
% 60%   | 161, 171, 113   | A1AB71   | 33, 6, 60, 18
\definecolor{ETHGreen60}{HTML}{A1AB71}
% 80%   | 129, 143, 66    | 818F42   | 44, 8, 80, 24
\definecolor{ETHGreen80}{HTML}{818F42}
% 120%  | 54, 82, 19      | 365213   | 55, 10, 100, 65
\definecolor{ETHGreen120}{HTML}{365213}

% ==============================================================
% 4. ETH Bronze Shades
% ==============================================================
% Shade | RGB             | HEX      | CMYK
% ------|-----------------|----------|---------------
% 10%   | 244, 240, 231   | F4F0E7   | 3, 4, 10, 3
\definecolor{ETHBronze10}{HTML}{F4F0E7}
% 20%   | 232, 225, 208   | E8E1D0   | 6, 7, 20, 5
\definecolor{ETHBronze20}{HTML}{E8E1D0}
% 40%   | 210, 194, 161   | D2C2A1   | 12, 14, 40, 10
\definecolor{ETHBronze40}{HTML}{D2C2A1}
% 60%   | 187, 164, 113   | BBA471   | 18, 22, 60, 15
\definecolor{ETHBronze60}{HTML}{BBA471}
% 80%   | 165, 133, 66    | A58542   | 24, 29, 80, 20
\definecolor{ETHBronze80}{HTML}{A58542}
% 120%  | 112, 79, 18     | 704F12   | 30, 36, 100, 55
\definecolor{ETHBronze120}{HTML}{704F12}

% ==============================================================
% 5. ETH Red Shades
% ==============================================================
% Shade | RGB             | HEX      | CMYK
% ------|-----------------|----------|---------------
% 10%   | 248, 235, 234   | F8EBEA   | 0, 9, 6, 0
\definecolor{ETHRed10}{HTML}{F8EBEA}
% 20%   | 241, 215, 213   | F1D7D5   | 0, 18, 13, 4
\definecolor{ETHRed20}{HTML}{F1D7D5}
% 40%   | 226, 174, 171   | E2AEAB   | 0, 36, 26, 8
\definecolor{ETHRed40}{HTML}{E2AEAB}
% 60%   | 212, 134, 129   | D48681   | 0, 54, 39, 11
\definecolor{ETHRed60}{HTML}{D48681}
% 80%   | 197, 93, 87     | C55D57   | (using HEX)
\definecolor{ETHRed80}{HTML}{C55D57}
% 120%  | 150, 39, 45     | 96272D   | 0, 100, 80, 40
\definecolor{ETHRed120}{HTML}{96272D}

% ==============================================================
% 6. ETH Purple Shades
% ==============================================================
% Shade | RGB             | HEX      | CMYK
% ------|-----------------|----------|---------------
% 10%   | 248, 232, 243   | F8E8F3   | 2, 10, 0, 1
\definecolor{ETHPurple10}{HTML}{F8E8F3}
% 20%   | 239, 208, 227   | EFD0E3   | 4, 20, 0, 1
\definecolor{ETHPurple20}{HTML}{EFD0E3}
% 40%   | 220, 158, 201   | DC9EC9   | 7, 40, 0, 4
\definecolor{ETHPurple40}{HTML}{DC9EC9}
% 60%   | 202, 108, 174   | CA6CAE   | 13, 60, 0, 6
\definecolor{ETHPurple60}{HTML}{CA6CAE}
% 80%   | 183, 59, 146    | B73B92   | 18, 80, 0, 8
\definecolor{ETHPurple80}{HTML}{B73B92}
% 120%  | 140, 10, 89     | 8C0A59   | 22, 100, 0, 35
\definecolor{ETHPurple120}{HTML}{8C0A59}

% ==============================================================
% 7. ETH Grey Shades
% ==============================================================
% Shade | RGB             | HEX      | CMYK
% ------|-----------------|----------|---------------
% 10%   | 241, 241, 241   | F1F1F1   | 0, 0, 0, 7
\definecolor{ETHGrey10}{HTML}{F1F1F1}
% 20%   | 226, 226, 226   | E2E2E2   | 0, 0, 0, 14
\definecolor{ETHGrey20}{HTML}{E2E2E2}
% 40%   | 197, 197, 197   | C5C5C5   | 0, 0, 0, 28
\definecolor{ETHGrey40}{HTML}{C5C5C5}
% 60%   | 169, 169, 169   | A9A9A9   | 0, 0, 0, 42
\definecolor{ETHGrey60}{HTML}{A9A9A9}
% 80%   | 140, 140, 140   | 8C8C8C   | 0, 0, 0, 56
\definecolor{ETHGrey80}{HTML}{8C8C8C}
% 120%  | 87, 87, 87      | 575757   | 0, 0, 0, 81
\definecolor{ETHGrey120}{HTML}{575757}
%   \textcolor{ETHBlue}{Hello from ETH!}
\NeedsTeXFormat{LaTeX2e}
\ProvidesFile{eth.tex}[2025/03/08 v1.0 ETH brand color definitions]

\RequirePackage{xcolor}

% ==============================================================
% Primary ETH Corporate Colors
% ==============================================================
\definecolor{ETHBlue}{HTML}{215CAF}    % ETH Blue: RGB: 33, 92, 175; CMYK: 100,57,0,0; Pantone: 2935; RAL: 5005 Signalblau
\definecolor{ETHPetrol}{HTML}{007894}   % ETH Petrol: RGB: 0,120,148; CMYK: 100,25,30,10; Pantone: 633; RAL: 5009 Azurblau
\definecolor{ETHGreen}{HTML}{627313}    % ETH Green: RGB: 98,115,19; CMYK: 55,10,100,30; Pantone: 364; RAL: 6010 Grasgrün
\definecolor{ETHBronze}{HTML}{8E6713}    % ETH Bronze: RGB: 142,103,19; CMYK: 30,36,100,25; Pantone: 4495; RAL: 7008 Khakigrau
\definecolor{ETHRed}{HTML}{B7352D}       % ETH Red: RGB: 183,53,45; CMYK: 0,90,80,17; Pantone: 1797; RAL: 3031 Orientrot
\definecolor{ETHPurple}{HTML}{A7117A}     % ETH Purple: RGB: 167,17,122; CMYK: 22,100,0,10; Pantone: 234; RAL: 4006 Verkehrspurpur
\definecolor{ETHGrey}{HTML}{6F6F6F}       % ETH Grey: RGB: 111,111,111; CMYK: 0,0,0,70; Pantone: Cool Gray 11; RAL: 7046 Telegrau 2


% ------------------------------------------------------------------
% Extended / Complementary Color Palette
% ------------------------------------------------------------------
\definecolor{ETHTeal}{HTML}{008C95}
\definecolor{ETHGreen}{HTML}{00B38B}
\definecolor{ETHDarkBlue}{HTML}{1D2447}
\definecolor{ETHLightBlue}{HTML}{5BB6D6}
\definecolor{ETHOrange}{HTML}{F39200}
\definecolor{ETHRed}{HTML}{C8002A}
\definecolor{ETHWarmGray}{HTML}{DAD7D2}
\definecolor{ETHBeige}{HTML}{D7CEC1}
\definecolor{ETHDarkBrown}{HTML}{7F4F3C}
\definecolor{ETHDarkPink}{HTML}{EB67BD}
\definecolor{ETHDarkPurple}{HTML}{5F2167}
\definecolor{ETHDarkMagenta}{HTML}{A3488E}
\definecolor{ETHDarkGray}{HTML}{333333}
\definecolor{ETHGray}{HTML}{75787B}
\definecolor{ETHLightGray}{HTML}{E2E2E2}
\definecolor{ETHWhite}{HTML}{FFFFFF}
\definecolor{ETHBlack}{HTML}{000000}

% ==============================================================
% 1. ETH Blue Shades
% ==============================================================
% Shade | RGB             | HEX      | CMYK
% ------|-----------------|----------|---------------
% 10%   | 233, 239, 247   | E9EFF7   | 10, 6, 0, 0
\definecolor{ETHBlue10}{HTML}{E9EFF7}
% 20%   | 211, 222, 239   | D3DEEF   | 20, 11, 0, 0
\definecolor{ETHBlue20}{HTML}{D3DEEF}
% 40%   | 166, 190, 223   | A6BEDF   | 40, 23, 0, 0
\definecolor{ETHBlue40}{HTML}{A6BEDF}
% 60%   | 122, 157, 207   | 7A9DCF   | 60, 34, 0, 0
\definecolor{ETHBlue60}{HTML}{7A9DCF}
% 80%   | 77, 125, 191    | 4D7DBF   | 80, 46, 0, 0
\definecolor{ETHBlue80}{HTML}{4D7DBF}
% 120%  | 8, 64, 126      | 08407E   | 100, 62, 0, 30
\definecolor{ETHBlue120}{HTML}{08407E}

% ==============================================================
% 2. ETH Petrol Shades
% ==============================================================
% Shade | RGB             | HEX      | CMYK
% ------|-----------------|----------|---------------
% 10%   | 231, 244, 247   | E7F4F7   | 12, 0, 5, 0
\definecolor{ETHPetrol10}{HTML}{E7F4F7}
% 20%   | 204, 228, 234   | CCE4EA   | 20, 3, 7, 0
\definecolor{ETHPetrol20}{HTML}{CCE4EA}
% 40%   | 153, 202, 213   | 99CAD5   | 40, 7, 12, 4
\definecolor{ETHPetrol40}{HTML}{99CAD5}
% 60%   | 102, 175, 192   | 66AFC0   | 60, 14, 18, 6
\definecolor{ETHPetrol60}{HTML}{66AFC0}
% 80%   | 51, 149, 171    | 3395AB   | 80, 20, 24, 8
\definecolor{ETHPetrol80}{HTML}{3395AB}
% 120%  | 0, 89, 109      | 00596D   | 100, 25, 30, 38
\definecolor{ETHPetrol120}{HTML}{00596D}

% ==============================================================
% 3. ETH Green Shades
% ==============================================================
% Shade | RGB             | HEX      | CMYK
% ------|-----------------|----------|---------------
% 10%   | 239, 241, 231   | EEF1E7   | 6, 1, 10, 3
\definecolor{ETHGreen10}{HTML}{EEF1E7}
% 20%   | 224, 227, 208   | E0E3D0   | 11, 2, 20, 6
\definecolor{ETHGreen20}{HTML}{E0E3D0}
% 40%   | 192, 199, 161   | C0C7A1   | 22, 4, 40, 12
\definecolor{ETHGreen40}{HTML}{C0C7A1}
% 60%   | 161, 171, 113   | A1AB71   | 33, 6, 60, 18
\definecolor{ETHGreen60}{HTML}{A1AB71}
% 80%   | 129, 143, 66    | 818F42   | 44, 8, 80, 24
\definecolor{ETHGreen80}{HTML}{818F42}
% 120%  | 54, 82, 19      | 365213   | 55, 10, 100, 65
\definecolor{ETHGreen120}{HTML}{365213}

% ==============================================================
% 4. ETH Bronze Shades
% ==============================================================
% Shade | RGB             | HEX      | CMYK
% ------|-----------------|----------|---------------
% 10%   | 244, 240, 231   | F4F0E7   | 3, 4, 10, 3
\definecolor{ETHBronze10}{HTML}{F4F0E7}
% 20%   | 232, 225, 208   | E8E1D0   | 6, 7, 20, 5
\definecolor{ETHBronze20}{HTML}{E8E1D0}
% 40%   | 210, 194, 161   | D2C2A1   | 12, 14, 40, 10
\definecolor{ETHBronze40}{HTML}{D2C2A1}
% 60%   | 187, 164, 113   | BBA471   | 18, 22, 60, 15
\definecolor{ETHBronze60}{HTML}{BBA471}
% 80%   | 165, 133, 66    | A58542   | 24, 29, 80, 20
\definecolor{ETHBronze80}{HTML}{A58542}
% 120%  | 112, 79, 18     | 704F12   | 30, 36, 100, 55
\definecolor{ETHBronze120}{HTML}{704F12}

% ==============================================================
% 5. ETH Red Shades
% ==============================================================
% Shade | RGB             | HEX      | CMYK
% ------|-----------------|----------|---------------
% 10%   | 248, 235, 234   | F8EBEA   | 0, 9, 6, 0
\definecolor{ETHRed10}{HTML}{F8EBEA}
% 20%   | 241, 215, 213   | F1D7D5   | 0, 18, 13, 4
\definecolor{ETHRed20}{HTML}{F1D7D5}
% 40%   | 226, 174, 171   | E2AEAB   | 0, 36, 26, 8
\definecolor{ETHRed40}{HTML}{E2AEAB}
% 60%   | 212, 134, 129   | D48681   | 0, 54, 39, 11
\definecolor{ETHRed60}{HTML}{D48681}
% 80%   | 197, 93, 87     | C55D57   | (using HEX)
\definecolor{ETHRed80}{HTML}{C55D57}
% 120%  | 150, 39, 45     | 96272D   | 0, 100, 80, 40
\definecolor{ETHRed120}{HTML}{96272D}

% ==============================================================
% 6. ETH Purple Shades
% ==============================================================
% Shade | RGB             | HEX      | CMYK
% ------|-----------------|----------|---------------
% 10%   | 248, 232, 243   | F8E8F3   | 2, 10, 0, 1
\definecolor{ETHPurple10}{HTML}{F8E8F3}
% 20%   | 239, 208, 227   | EFD0E3   | 4, 20, 0, 1
\definecolor{ETHPurple20}{HTML}{EFD0E3}
% 40%   | 220, 158, 201   | DC9EC9   | 7, 40, 0, 4
\definecolor{ETHPurple40}{HTML}{DC9EC9}
% 60%   | 202, 108, 174   | CA6CAE   | 13, 60, 0, 6
\definecolor{ETHPurple60}{HTML}{CA6CAE}
% 80%   | 183, 59, 146    | B73B92   | 18, 80, 0, 8
\definecolor{ETHPurple80}{HTML}{B73B92}
% 120%  | 140, 10, 89     | 8C0A59   | 22, 100, 0, 35
\definecolor{ETHPurple120}{HTML}{8C0A59}

% ==============================================================
% 7. ETH Grey Shades
% ==============================================================
% Shade | RGB             | HEX      | CMYK
% ------|-----------------|----------|---------------
% 10%   | 241, 241, 241   | F1F1F1   | 0, 0, 0, 7
\definecolor{ETHGrey10}{HTML}{F1F1F1}
% 20%   | 226, 226, 226   | E2E2E2   | 0, 0, 0, 14
\definecolor{ETHGrey20}{HTML}{E2E2E2}
% 40%   | 197, 197, 197   | C5C5C5   | 0, 0, 0, 28
\definecolor{ETHGrey40}{HTML}{C5C5C5}
% 60%   | 169, 169, 169   | A9A9A9   | 0, 0, 0, 42
\definecolor{ETHGrey60}{HTML}{A9A9A9}
% 80%   | 140, 140, 140   | 8C8C8C   | 0, 0, 0, 56
\definecolor{ETHGrey80}{HTML}{8C8C8C}
% 120%  | 87, 87, 87      | 575757   | 0, 0, 0, 81
\definecolor{ETHGrey120}{HTML}{575757}
%   \textcolor{ETHBlue}{Hello ETH!}
% Usage in your main .tex:
%   \usepackage{xcolor} % or colortbl, etc.
%   % eth.tex
% Defines ETH brand colors based on:
% https://ethz.ch/staffnet/en/service/communication/corporate-design/colours.html
% 
% ------------------------------------------------------------------
% ETH Corporate Design – Primary Colors and Colour Shades Definitions
%
% PRIMARY ETH CORPORATE COLORS
%
% Colour       RGB             HEX       CMYK                 Pantone    RAL
% ----------------------------------------------------------------------------
% ETH Blue     33, 92, 175     #215CAF   100,57,0,0           2935       5005 Signalblau
% ETH Petrol   0, 120, 148     #007894   100,25,30,10         633        5009 Azurblau
% ETH Green    98, 115, 19     #627313   55,10,100,30         364        6010 Grasgrün
% ETH Bronze   142, 103, 19    #8E6713   30,36,100,25         4495       7008 Khakigrau
% ETH Red      183, 53, 45     #B7352D   0,90,80,17           1797       3031 Orientrot
% ETH Purple   167, 17, 122    #A7117A   22,100,0,10          234        4006 Verkehrspurpur
% ETH Grey     111, 111, 111   #6F6F6F   0,0,0,70             Cool Gray 11   7046 Telegrau 2
% ------------------------------------------------------------------
% ETH Corporate Design – Colour Shades Definitions
%
% 1. ETH Blue Shades
% 2. ETH Petrol Shades
% 3. ETH Green Shades
% 4. ETH Bronze Shades
% 5. ETH Red Shades
% 6. ETH Purple Shades
% 7. ETH Grey Shades
% 
% Last updated: 2025-03-08
%
% Usage:
%   % eth.tex
% Defines ETH brand colors based on:
% https://ethz.ch/staffnet/en/service/communication/corporate-design/colours.html
% 
% ------------------------------------------------------------------
% ETH Corporate Design – Primary Colors and Colour Shades Definitions
%
% PRIMARY ETH CORPORATE COLORS
%
% Colour       RGB             HEX       CMYK                 Pantone    RAL
% ----------------------------------------------------------------------------
% ETH Blue     33, 92, 175     #215CAF   100,57,0,0           2935       5005 Signalblau
% ETH Petrol   0, 120, 148     #007894   100,25,30,10         633        5009 Azurblau
% ETH Green    98, 115, 19     #627313   55,10,100,30         364        6010 Grasgrün
% ETH Bronze   142, 103, 19    #8E6713   30,36,100,25         4495       7008 Khakigrau
% ETH Red      183, 53, 45     #B7352D   0,90,80,17           1797       3031 Orientrot
% ETH Purple   167, 17, 122    #A7117A   22,100,0,10          234        4006 Verkehrspurpur
% ETH Grey     111, 111, 111   #6F6F6F   0,0,0,70             Cool Gray 11   7046 Telegrau 2
% ------------------------------------------------------------------
% ETH Corporate Design – Colour Shades Definitions
%
% 1. ETH Blue Shades
% 2. ETH Petrol Shades
% 3. ETH Green Shades
% 4. ETH Bronze Shades
% 5. ETH Red Shades
% 6. ETH Purple Shades
% 7. ETH Grey Shades
% 
% Last updated: 2025-03-08
%
% Usage:
%   \input{eth.tex}
%   \textcolor{ETHBlue}{Hello ETH!}
% Usage in your main .tex:
%   \usepackage{xcolor} % or colortbl, etc.
%   \input{eth.tex}
%   \textcolor{ETHBlue}{Hello from ETH!}
\NeedsTeXFormat{LaTeX2e}
\ProvidesFile{eth.tex}[2025/03/08 v1.0 ETH brand color definitions]

\RequirePackage{xcolor}

% ==============================================================
% Primary ETH Corporate Colors
% ==============================================================
\definecolor{ETHBlue}{HTML}{215CAF}    % ETH Blue: RGB: 33, 92, 175; CMYK: 100,57,0,0; Pantone: 2935; RAL: 5005 Signalblau
\definecolor{ETHPetrol}{HTML}{007894}   % ETH Petrol: RGB: 0,120,148; CMYK: 100,25,30,10; Pantone: 633; RAL: 5009 Azurblau
\definecolor{ETHGreen}{HTML}{627313}    % ETH Green: RGB: 98,115,19; CMYK: 55,10,100,30; Pantone: 364; RAL: 6010 Grasgrün
\definecolor{ETHBronze}{HTML}{8E6713}    % ETH Bronze: RGB: 142,103,19; CMYK: 30,36,100,25; Pantone: 4495; RAL: 7008 Khakigrau
\definecolor{ETHRed}{HTML}{B7352D}       % ETH Red: RGB: 183,53,45; CMYK: 0,90,80,17; Pantone: 1797; RAL: 3031 Orientrot
\definecolor{ETHPurple}{HTML}{A7117A}     % ETH Purple: RGB: 167,17,122; CMYK: 22,100,0,10; Pantone: 234; RAL: 4006 Verkehrspurpur
\definecolor{ETHGrey}{HTML}{6F6F6F}       % ETH Grey: RGB: 111,111,111; CMYK: 0,0,0,70; Pantone: Cool Gray 11; RAL: 7046 Telegrau 2


% ------------------------------------------------------------------
% Extended / Complementary Color Palette
% ------------------------------------------------------------------
\definecolor{ETHTeal}{HTML}{008C95}
\definecolor{ETHGreen}{HTML}{00B38B}
\definecolor{ETHDarkBlue}{HTML}{1D2447}
\definecolor{ETHLightBlue}{HTML}{5BB6D6}
\definecolor{ETHOrange}{HTML}{F39200}
\definecolor{ETHRed}{HTML}{C8002A}
\definecolor{ETHWarmGray}{HTML}{DAD7D2}
\definecolor{ETHBeige}{HTML}{D7CEC1}
\definecolor{ETHDarkBrown}{HTML}{7F4F3C}
\definecolor{ETHDarkPink}{HTML}{EB67BD}
\definecolor{ETHDarkPurple}{HTML}{5F2167}
\definecolor{ETHDarkMagenta}{HTML}{A3488E}
\definecolor{ETHDarkGray}{HTML}{333333}
\definecolor{ETHGray}{HTML}{75787B}
\definecolor{ETHLightGray}{HTML}{E2E2E2}
\definecolor{ETHWhite}{HTML}{FFFFFF}
\definecolor{ETHBlack}{HTML}{000000}

% ==============================================================
% 1. ETH Blue Shades
% ==============================================================
% Shade | RGB             | HEX      | CMYK
% ------|-----------------|----------|---------------
% 10%   | 233, 239, 247   | E9EFF7   | 10, 6, 0, 0
\definecolor{ETHBlue10}{HTML}{E9EFF7}
% 20%   | 211, 222, 239   | D3DEEF   | 20, 11, 0, 0
\definecolor{ETHBlue20}{HTML}{D3DEEF}
% 40%   | 166, 190, 223   | A6BEDF   | 40, 23, 0, 0
\definecolor{ETHBlue40}{HTML}{A6BEDF}
% 60%   | 122, 157, 207   | 7A9DCF   | 60, 34, 0, 0
\definecolor{ETHBlue60}{HTML}{7A9DCF}
% 80%   | 77, 125, 191    | 4D7DBF   | 80, 46, 0, 0
\definecolor{ETHBlue80}{HTML}{4D7DBF}
% 120%  | 8, 64, 126      | 08407E   | 100, 62, 0, 30
\definecolor{ETHBlue120}{HTML}{08407E}

% ==============================================================
% 2. ETH Petrol Shades
% ==============================================================
% Shade | RGB             | HEX      | CMYK
% ------|-----------------|----------|---------------
% 10%   | 231, 244, 247   | E7F4F7   | 12, 0, 5, 0
\definecolor{ETHPetrol10}{HTML}{E7F4F7}
% 20%   | 204, 228, 234   | CCE4EA   | 20, 3, 7, 0
\definecolor{ETHPetrol20}{HTML}{CCE4EA}
% 40%   | 153, 202, 213   | 99CAD5   | 40, 7, 12, 4
\definecolor{ETHPetrol40}{HTML}{99CAD5}
% 60%   | 102, 175, 192   | 66AFC0   | 60, 14, 18, 6
\definecolor{ETHPetrol60}{HTML}{66AFC0}
% 80%   | 51, 149, 171    | 3395AB   | 80, 20, 24, 8
\definecolor{ETHPetrol80}{HTML}{3395AB}
% 120%  | 0, 89, 109      | 00596D   | 100, 25, 30, 38
\definecolor{ETHPetrol120}{HTML}{00596D}

% ==============================================================
% 3. ETH Green Shades
% ==============================================================
% Shade | RGB             | HEX      | CMYK
% ------|-----------------|----------|---------------
% 10%   | 239, 241, 231   | EEF1E7   | 6, 1, 10, 3
\definecolor{ETHGreen10}{HTML}{EEF1E7}
% 20%   | 224, 227, 208   | E0E3D0   | 11, 2, 20, 6
\definecolor{ETHGreen20}{HTML}{E0E3D0}
% 40%   | 192, 199, 161   | C0C7A1   | 22, 4, 40, 12
\definecolor{ETHGreen40}{HTML}{C0C7A1}
% 60%   | 161, 171, 113   | A1AB71   | 33, 6, 60, 18
\definecolor{ETHGreen60}{HTML}{A1AB71}
% 80%   | 129, 143, 66    | 818F42   | 44, 8, 80, 24
\definecolor{ETHGreen80}{HTML}{818F42}
% 120%  | 54, 82, 19      | 365213   | 55, 10, 100, 65
\definecolor{ETHGreen120}{HTML}{365213}

% ==============================================================
% 4. ETH Bronze Shades
% ==============================================================
% Shade | RGB             | HEX      | CMYK
% ------|-----------------|----------|---------------
% 10%   | 244, 240, 231   | F4F0E7   | 3, 4, 10, 3
\definecolor{ETHBronze10}{HTML}{F4F0E7}
% 20%   | 232, 225, 208   | E8E1D0   | 6, 7, 20, 5
\definecolor{ETHBronze20}{HTML}{E8E1D0}
% 40%   | 210, 194, 161   | D2C2A1   | 12, 14, 40, 10
\definecolor{ETHBronze40}{HTML}{D2C2A1}
% 60%   | 187, 164, 113   | BBA471   | 18, 22, 60, 15
\definecolor{ETHBronze60}{HTML}{BBA471}
% 80%   | 165, 133, 66    | A58542   | 24, 29, 80, 20
\definecolor{ETHBronze80}{HTML}{A58542}
% 120%  | 112, 79, 18     | 704F12   | 30, 36, 100, 55
\definecolor{ETHBronze120}{HTML}{704F12}

% ==============================================================
% 5. ETH Red Shades
% ==============================================================
% Shade | RGB             | HEX      | CMYK
% ------|-----------------|----------|---------------
% 10%   | 248, 235, 234   | F8EBEA   | 0, 9, 6, 0
\definecolor{ETHRed10}{HTML}{F8EBEA}
% 20%   | 241, 215, 213   | F1D7D5   | 0, 18, 13, 4
\definecolor{ETHRed20}{HTML}{F1D7D5}
% 40%   | 226, 174, 171   | E2AEAB   | 0, 36, 26, 8
\definecolor{ETHRed40}{HTML}{E2AEAB}
% 60%   | 212, 134, 129   | D48681   | 0, 54, 39, 11
\definecolor{ETHRed60}{HTML}{D48681}
% 80%   | 197, 93, 87     | C55D57   | (using HEX)
\definecolor{ETHRed80}{HTML}{C55D57}
% 120%  | 150, 39, 45     | 96272D   | 0, 100, 80, 40
\definecolor{ETHRed120}{HTML}{96272D}

% ==============================================================
% 6. ETH Purple Shades
% ==============================================================
% Shade | RGB             | HEX      | CMYK
% ------|-----------------|----------|---------------
% 10%   | 248, 232, 243   | F8E8F3   | 2, 10, 0, 1
\definecolor{ETHPurple10}{HTML}{F8E8F3}
% 20%   | 239, 208, 227   | EFD0E3   | 4, 20, 0, 1
\definecolor{ETHPurple20}{HTML}{EFD0E3}
% 40%   | 220, 158, 201   | DC9EC9   | 7, 40, 0, 4
\definecolor{ETHPurple40}{HTML}{DC9EC9}
% 60%   | 202, 108, 174   | CA6CAE   | 13, 60, 0, 6
\definecolor{ETHPurple60}{HTML}{CA6CAE}
% 80%   | 183, 59, 146    | B73B92   | 18, 80, 0, 8
\definecolor{ETHPurple80}{HTML}{B73B92}
% 120%  | 140, 10, 89     | 8C0A59   | 22, 100, 0, 35
\definecolor{ETHPurple120}{HTML}{8C0A59}

% ==============================================================
% 7. ETH Grey Shades
% ==============================================================
% Shade | RGB             | HEX      | CMYK
% ------|-----------------|----------|---------------
% 10%   | 241, 241, 241   | F1F1F1   | 0, 0, 0, 7
\definecolor{ETHGrey10}{HTML}{F1F1F1}
% 20%   | 226, 226, 226   | E2E2E2   | 0, 0, 0, 14
\definecolor{ETHGrey20}{HTML}{E2E2E2}
% 40%   | 197, 197, 197   | C5C5C5   | 0, 0, 0, 28
\definecolor{ETHGrey40}{HTML}{C5C5C5}
% 60%   | 169, 169, 169   | A9A9A9   | 0, 0, 0, 42
\definecolor{ETHGrey60}{HTML}{A9A9A9}
% 80%   | 140, 140, 140   | 8C8C8C   | 0, 0, 0, 56
\definecolor{ETHGrey80}{HTML}{8C8C8C}
% 120%  | 87, 87, 87      | 575757   | 0, 0, 0, 81
\definecolor{ETHGrey120}{HTML}{575757}
%   \textcolor{ETHBlue}{Hello ETH!}
% Usage in your main .tex:
%   \usepackage{xcolor} % or colortbl, etc.
%   % eth.tex
% Defines ETH brand colors based on:
% https://ethz.ch/staffnet/en/service/communication/corporate-design/colours.html
% 
% ------------------------------------------------------------------
% ETH Corporate Design – Primary Colors and Colour Shades Definitions
%
% PRIMARY ETH CORPORATE COLORS
%
% Colour       RGB             HEX       CMYK                 Pantone    RAL
% ----------------------------------------------------------------------------
% ETH Blue     33, 92, 175     #215CAF   100,57,0,0           2935       5005 Signalblau
% ETH Petrol   0, 120, 148     #007894   100,25,30,10         633        5009 Azurblau
% ETH Green    98, 115, 19     #627313   55,10,100,30         364        6010 Grasgrün
% ETH Bronze   142, 103, 19    #8E6713   30,36,100,25         4495       7008 Khakigrau
% ETH Red      183, 53, 45     #B7352D   0,90,80,17           1797       3031 Orientrot
% ETH Purple   167, 17, 122    #A7117A   22,100,0,10          234        4006 Verkehrspurpur
% ETH Grey     111, 111, 111   #6F6F6F   0,0,0,70             Cool Gray 11   7046 Telegrau 2
% ------------------------------------------------------------------
% ETH Corporate Design – Colour Shades Definitions
%
% 1. ETH Blue Shades
% 2. ETH Petrol Shades
% 3. ETH Green Shades
% 4. ETH Bronze Shades
% 5. ETH Red Shades
% 6. ETH Purple Shades
% 7. ETH Grey Shades
% 
% Last updated: 2025-03-08
%
% Usage:
%   \input{eth.tex}
%   \textcolor{ETHBlue}{Hello ETH!}
% Usage in your main .tex:
%   \usepackage{xcolor} % or colortbl, etc.
%   \input{eth.tex}
%   \textcolor{ETHBlue}{Hello from ETH!}
\NeedsTeXFormat{LaTeX2e}
\ProvidesFile{eth.tex}[2025/03/08 v1.0 ETH brand color definitions]

\RequirePackage{xcolor}

% ==============================================================
% Primary ETH Corporate Colors
% ==============================================================
\definecolor{ETHBlue}{HTML}{215CAF}    % ETH Blue: RGB: 33, 92, 175; CMYK: 100,57,0,0; Pantone: 2935; RAL: 5005 Signalblau
\definecolor{ETHPetrol}{HTML}{007894}   % ETH Petrol: RGB: 0,120,148; CMYK: 100,25,30,10; Pantone: 633; RAL: 5009 Azurblau
\definecolor{ETHGreen}{HTML}{627313}    % ETH Green: RGB: 98,115,19; CMYK: 55,10,100,30; Pantone: 364; RAL: 6010 Grasgrün
\definecolor{ETHBronze}{HTML}{8E6713}    % ETH Bronze: RGB: 142,103,19; CMYK: 30,36,100,25; Pantone: 4495; RAL: 7008 Khakigrau
\definecolor{ETHRed}{HTML}{B7352D}       % ETH Red: RGB: 183,53,45; CMYK: 0,90,80,17; Pantone: 1797; RAL: 3031 Orientrot
\definecolor{ETHPurple}{HTML}{A7117A}     % ETH Purple: RGB: 167,17,122; CMYK: 22,100,0,10; Pantone: 234; RAL: 4006 Verkehrspurpur
\definecolor{ETHGrey}{HTML}{6F6F6F}       % ETH Grey: RGB: 111,111,111; CMYK: 0,0,0,70; Pantone: Cool Gray 11; RAL: 7046 Telegrau 2


% ------------------------------------------------------------------
% Extended / Complementary Color Palette
% ------------------------------------------------------------------
\definecolor{ETHTeal}{HTML}{008C95}
\definecolor{ETHGreen}{HTML}{00B38B}
\definecolor{ETHDarkBlue}{HTML}{1D2447}
\definecolor{ETHLightBlue}{HTML}{5BB6D6}
\definecolor{ETHOrange}{HTML}{F39200}
\definecolor{ETHRed}{HTML}{C8002A}
\definecolor{ETHWarmGray}{HTML}{DAD7D2}
\definecolor{ETHBeige}{HTML}{D7CEC1}
\definecolor{ETHDarkBrown}{HTML}{7F4F3C}
\definecolor{ETHDarkPink}{HTML}{EB67BD}
\definecolor{ETHDarkPurple}{HTML}{5F2167}
\definecolor{ETHDarkMagenta}{HTML}{A3488E}
\definecolor{ETHDarkGray}{HTML}{333333}
\definecolor{ETHGray}{HTML}{75787B}
\definecolor{ETHLightGray}{HTML}{E2E2E2}
\definecolor{ETHWhite}{HTML}{FFFFFF}
\definecolor{ETHBlack}{HTML}{000000}

% ==============================================================
% 1. ETH Blue Shades
% ==============================================================
% Shade | RGB             | HEX      | CMYK
% ------|-----------------|----------|---------------
% 10%   | 233, 239, 247   | E9EFF7   | 10, 6, 0, 0
\definecolor{ETHBlue10}{HTML}{E9EFF7}
% 20%   | 211, 222, 239   | D3DEEF   | 20, 11, 0, 0
\definecolor{ETHBlue20}{HTML}{D3DEEF}
% 40%   | 166, 190, 223   | A6BEDF   | 40, 23, 0, 0
\definecolor{ETHBlue40}{HTML}{A6BEDF}
% 60%   | 122, 157, 207   | 7A9DCF   | 60, 34, 0, 0
\definecolor{ETHBlue60}{HTML}{7A9DCF}
% 80%   | 77, 125, 191    | 4D7DBF   | 80, 46, 0, 0
\definecolor{ETHBlue80}{HTML}{4D7DBF}
% 120%  | 8, 64, 126      | 08407E   | 100, 62, 0, 30
\definecolor{ETHBlue120}{HTML}{08407E}

% ==============================================================
% 2. ETH Petrol Shades
% ==============================================================
% Shade | RGB             | HEX      | CMYK
% ------|-----------------|----------|---------------
% 10%   | 231, 244, 247   | E7F4F7   | 12, 0, 5, 0
\definecolor{ETHPetrol10}{HTML}{E7F4F7}
% 20%   | 204, 228, 234   | CCE4EA   | 20, 3, 7, 0
\definecolor{ETHPetrol20}{HTML}{CCE4EA}
% 40%   | 153, 202, 213   | 99CAD5   | 40, 7, 12, 4
\definecolor{ETHPetrol40}{HTML}{99CAD5}
% 60%   | 102, 175, 192   | 66AFC0   | 60, 14, 18, 6
\definecolor{ETHPetrol60}{HTML}{66AFC0}
% 80%   | 51, 149, 171    | 3395AB   | 80, 20, 24, 8
\definecolor{ETHPetrol80}{HTML}{3395AB}
% 120%  | 0, 89, 109      | 00596D   | 100, 25, 30, 38
\definecolor{ETHPetrol120}{HTML}{00596D}

% ==============================================================
% 3. ETH Green Shades
% ==============================================================
% Shade | RGB             | HEX      | CMYK
% ------|-----------------|----------|---------------
% 10%   | 239, 241, 231   | EEF1E7   | 6, 1, 10, 3
\definecolor{ETHGreen10}{HTML}{EEF1E7}
% 20%   | 224, 227, 208   | E0E3D0   | 11, 2, 20, 6
\definecolor{ETHGreen20}{HTML}{E0E3D0}
% 40%   | 192, 199, 161   | C0C7A1   | 22, 4, 40, 12
\definecolor{ETHGreen40}{HTML}{C0C7A1}
% 60%   | 161, 171, 113   | A1AB71   | 33, 6, 60, 18
\definecolor{ETHGreen60}{HTML}{A1AB71}
% 80%   | 129, 143, 66    | 818F42   | 44, 8, 80, 24
\definecolor{ETHGreen80}{HTML}{818F42}
% 120%  | 54, 82, 19      | 365213   | 55, 10, 100, 65
\definecolor{ETHGreen120}{HTML}{365213}

% ==============================================================
% 4. ETH Bronze Shades
% ==============================================================
% Shade | RGB             | HEX      | CMYK
% ------|-----------------|----------|---------------
% 10%   | 244, 240, 231   | F4F0E7   | 3, 4, 10, 3
\definecolor{ETHBronze10}{HTML}{F4F0E7}
% 20%   | 232, 225, 208   | E8E1D0   | 6, 7, 20, 5
\definecolor{ETHBronze20}{HTML}{E8E1D0}
% 40%   | 210, 194, 161   | D2C2A1   | 12, 14, 40, 10
\definecolor{ETHBronze40}{HTML}{D2C2A1}
% 60%   | 187, 164, 113   | BBA471   | 18, 22, 60, 15
\definecolor{ETHBronze60}{HTML}{BBA471}
% 80%   | 165, 133, 66    | A58542   | 24, 29, 80, 20
\definecolor{ETHBronze80}{HTML}{A58542}
% 120%  | 112, 79, 18     | 704F12   | 30, 36, 100, 55
\definecolor{ETHBronze120}{HTML}{704F12}

% ==============================================================
% 5. ETH Red Shades
% ==============================================================
% Shade | RGB             | HEX      | CMYK
% ------|-----------------|----------|---------------
% 10%   | 248, 235, 234   | F8EBEA   | 0, 9, 6, 0
\definecolor{ETHRed10}{HTML}{F8EBEA}
% 20%   | 241, 215, 213   | F1D7D5   | 0, 18, 13, 4
\definecolor{ETHRed20}{HTML}{F1D7D5}
% 40%   | 226, 174, 171   | E2AEAB   | 0, 36, 26, 8
\definecolor{ETHRed40}{HTML}{E2AEAB}
% 60%   | 212, 134, 129   | D48681   | 0, 54, 39, 11
\definecolor{ETHRed60}{HTML}{D48681}
% 80%   | 197, 93, 87     | C55D57   | (using HEX)
\definecolor{ETHRed80}{HTML}{C55D57}
% 120%  | 150, 39, 45     | 96272D   | 0, 100, 80, 40
\definecolor{ETHRed120}{HTML}{96272D}

% ==============================================================
% 6. ETH Purple Shades
% ==============================================================
% Shade | RGB             | HEX      | CMYK
% ------|-----------------|----------|---------------
% 10%   | 248, 232, 243   | F8E8F3   | 2, 10, 0, 1
\definecolor{ETHPurple10}{HTML}{F8E8F3}
% 20%   | 239, 208, 227   | EFD0E3   | 4, 20, 0, 1
\definecolor{ETHPurple20}{HTML}{EFD0E3}
% 40%   | 220, 158, 201   | DC9EC9   | 7, 40, 0, 4
\definecolor{ETHPurple40}{HTML}{DC9EC9}
% 60%   | 202, 108, 174   | CA6CAE   | 13, 60, 0, 6
\definecolor{ETHPurple60}{HTML}{CA6CAE}
% 80%   | 183, 59, 146    | B73B92   | 18, 80, 0, 8
\definecolor{ETHPurple80}{HTML}{B73B92}
% 120%  | 140, 10, 89     | 8C0A59   | 22, 100, 0, 35
\definecolor{ETHPurple120}{HTML}{8C0A59}

% ==============================================================
% 7. ETH Grey Shades
% ==============================================================
% Shade | RGB             | HEX      | CMYK
% ------|-----------------|----------|---------------
% 10%   | 241, 241, 241   | F1F1F1   | 0, 0, 0, 7
\definecolor{ETHGrey10}{HTML}{F1F1F1}
% 20%   | 226, 226, 226   | E2E2E2   | 0, 0, 0, 14
\definecolor{ETHGrey20}{HTML}{E2E2E2}
% 40%   | 197, 197, 197   | C5C5C5   | 0, 0, 0, 28
\definecolor{ETHGrey40}{HTML}{C5C5C5}
% 60%   | 169, 169, 169   | A9A9A9   | 0, 0, 0, 42
\definecolor{ETHGrey60}{HTML}{A9A9A9}
% 80%   | 140, 140, 140   | 8C8C8C   | 0, 0, 0, 56
\definecolor{ETHGrey80}{HTML}{8C8C8C}
% 120%  | 87, 87, 87      | 575757   | 0, 0, 0, 81
\definecolor{ETHGrey120}{HTML}{575757}
%   \textcolor{ETHBlue}{Hello from ETH!}
\NeedsTeXFormat{LaTeX2e}
\ProvidesFile{eth.tex}[2025/03/08 v1.0 ETH brand color definitions]

\RequirePackage{xcolor}

% ==============================================================
% Primary ETH Corporate Colors
% ==============================================================
\definecolor{ETHBlue}{HTML}{215CAF}    % ETH Blue: RGB: 33, 92, 175; CMYK: 100,57,0,0; Pantone: 2935; RAL: 5005 Signalblau
\definecolor{ETHPetrol}{HTML}{007894}   % ETH Petrol: RGB: 0,120,148; CMYK: 100,25,30,10; Pantone: 633; RAL: 5009 Azurblau
\definecolor{ETHGreen}{HTML}{627313}    % ETH Green: RGB: 98,115,19; CMYK: 55,10,100,30; Pantone: 364; RAL: 6010 Grasgrün
\definecolor{ETHBronze}{HTML}{8E6713}    % ETH Bronze: RGB: 142,103,19; CMYK: 30,36,100,25; Pantone: 4495; RAL: 7008 Khakigrau
\definecolor{ETHRed}{HTML}{B7352D}       % ETH Red: RGB: 183,53,45; CMYK: 0,90,80,17; Pantone: 1797; RAL: 3031 Orientrot
\definecolor{ETHPurple}{HTML}{A7117A}     % ETH Purple: RGB: 167,17,122; CMYK: 22,100,0,10; Pantone: 234; RAL: 4006 Verkehrspurpur
\definecolor{ETHGrey}{HTML}{6F6F6F}       % ETH Grey: RGB: 111,111,111; CMYK: 0,0,0,70; Pantone: Cool Gray 11; RAL: 7046 Telegrau 2


% ------------------------------------------------------------------
% Extended / Complementary Color Palette
% ------------------------------------------------------------------
\definecolor{ETHTeal}{HTML}{008C95}
\definecolor{ETHGreen}{HTML}{00B38B}
\definecolor{ETHDarkBlue}{HTML}{1D2447}
\definecolor{ETHLightBlue}{HTML}{5BB6D6}
\definecolor{ETHOrange}{HTML}{F39200}
\definecolor{ETHRed}{HTML}{C8002A}
\definecolor{ETHWarmGray}{HTML}{DAD7D2}
\definecolor{ETHBeige}{HTML}{D7CEC1}
\definecolor{ETHDarkBrown}{HTML}{7F4F3C}
\definecolor{ETHDarkPink}{HTML}{EB67BD}
\definecolor{ETHDarkPurple}{HTML}{5F2167}
\definecolor{ETHDarkMagenta}{HTML}{A3488E}
\definecolor{ETHDarkGray}{HTML}{333333}
\definecolor{ETHGray}{HTML}{75787B}
\definecolor{ETHLightGray}{HTML}{E2E2E2}
\definecolor{ETHWhite}{HTML}{FFFFFF}
\definecolor{ETHBlack}{HTML}{000000}

% ==============================================================
% 1. ETH Blue Shades
% ==============================================================
% Shade | RGB             | HEX      | CMYK
% ------|-----------------|----------|---------------
% 10%   | 233, 239, 247   | E9EFF7   | 10, 6, 0, 0
\definecolor{ETHBlue10}{HTML}{E9EFF7}
% 20%   | 211, 222, 239   | D3DEEF   | 20, 11, 0, 0
\definecolor{ETHBlue20}{HTML}{D3DEEF}
% 40%   | 166, 190, 223   | A6BEDF   | 40, 23, 0, 0
\definecolor{ETHBlue40}{HTML}{A6BEDF}
% 60%   | 122, 157, 207   | 7A9DCF   | 60, 34, 0, 0
\definecolor{ETHBlue60}{HTML}{7A9DCF}
% 80%   | 77, 125, 191    | 4D7DBF   | 80, 46, 0, 0
\definecolor{ETHBlue80}{HTML}{4D7DBF}
% 120%  | 8, 64, 126      | 08407E   | 100, 62, 0, 30
\definecolor{ETHBlue120}{HTML}{08407E}

% ==============================================================
% 2. ETH Petrol Shades
% ==============================================================
% Shade | RGB             | HEX      | CMYK
% ------|-----------------|----------|---------------
% 10%   | 231, 244, 247   | E7F4F7   | 12, 0, 5, 0
\definecolor{ETHPetrol10}{HTML}{E7F4F7}
% 20%   | 204, 228, 234   | CCE4EA   | 20, 3, 7, 0
\definecolor{ETHPetrol20}{HTML}{CCE4EA}
% 40%   | 153, 202, 213   | 99CAD5   | 40, 7, 12, 4
\definecolor{ETHPetrol40}{HTML}{99CAD5}
% 60%   | 102, 175, 192   | 66AFC0   | 60, 14, 18, 6
\definecolor{ETHPetrol60}{HTML}{66AFC0}
% 80%   | 51, 149, 171    | 3395AB   | 80, 20, 24, 8
\definecolor{ETHPetrol80}{HTML}{3395AB}
% 120%  | 0, 89, 109      | 00596D   | 100, 25, 30, 38
\definecolor{ETHPetrol120}{HTML}{00596D}

% ==============================================================
% 3. ETH Green Shades
% ==============================================================
% Shade | RGB             | HEX      | CMYK
% ------|-----------------|----------|---------------
% 10%   | 239, 241, 231   | EEF1E7   | 6, 1, 10, 3
\definecolor{ETHGreen10}{HTML}{EEF1E7}
% 20%   | 224, 227, 208   | E0E3D0   | 11, 2, 20, 6
\definecolor{ETHGreen20}{HTML}{E0E3D0}
% 40%   | 192, 199, 161   | C0C7A1   | 22, 4, 40, 12
\definecolor{ETHGreen40}{HTML}{C0C7A1}
% 60%   | 161, 171, 113   | A1AB71   | 33, 6, 60, 18
\definecolor{ETHGreen60}{HTML}{A1AB71}
% 80%   | 129, 143, 66    | 818F42   | 44, 8, 80, 24
\definecolor{ETHGreen80}{HTML}{818F42}
% 120%  | 54, 82, 19      | 365213   | 55, 10, 100, 65
\definecolor{ETHGreen120}{HTML}{365213}

% ==============================================================
% 4. ETH Bronze Shades
% ==============================================================
% Shade | RGB             | HEX      | CMYK
% ------|-----------------|----------|---------------
% 10%   | 244, 240, 231   | F4F0E7   | 3, 4, 10, 3
\definecolor{ETHBronze10}{HTML}{F4F0E7}
% 20%   | 232, 225, 208   | E8E1D0   | 6, 7, 20, 5
\definecolor{ETHBronze20}{HTML}{E8E1D0}
% 40%   | 210, 194, 161   | D2C2A1   | 12, 14, 40, 10
\definecolor{ETHBronze40}{HTML}{D2C2A1}
% 60%   | 187, 164, 113   | BBA471   | 18, 22, 60, 15
\definecolor{ETHBronze60}{HTML}{BBA471}
% 80%   | 165, 133, 66    | A58542   | 24, 29, 80, 20
\definecolor{ETHBronze80}{HTML}{A58542}
% 120%  | 112, 79, 18     | 704F12   | 30, 36, 100, 55
\definecolor{ETHBronze120}{HTML}{704F12}

% ==============================================================
% 5. ETH Red Shades
% ==============================================================
% Shade | RGB             | HEX      | CMYK
% ------|-----------------|----------|---------------
% 10%   | 248, 235, 234   | F8EBEA   | 0, 9, 6, 0
\definecolor{ETHRed10}{HTML}{F8EBEA}
% 20%   | 241, 215, 213   | F1D7D5   | 0, 18, 13, 4
\definecolor{ETHRed20}{HTML}{F1D7D5}
% 40%   | 226, 174, 171   | E2AEAB   | 0, 36, 26, 8
\definecolor{ETHRed40}{HTML}{E2AEAB}
% 60%   | 212, 134, 129   | D48681   | 0, 54, 39, 11
\definecolor{ETHRed60}{HTML}{D48681}
% 80%   | 197, 93, 87     | C55D57   | (using HEX)
\definecolor{ETHRed80}{HTML}{C55D57}
% 120%  | 150, 39, 45     | 96272D   | 0, 100, 80, 40
\definecolor{ETHRed120}{HTML}{96272D}

% ==============================================================
% 6. ETH Purple Shades
% ==============================================================
% Shade | RGB             | HEX      | CMYK
% ------|-----------------|----------|---------------
% 10%   | 248, 232, 243   | F8E8F3   | 2, 10, 0, 1
\definecolor{ETHPurple10}{HTML}{F8E8F3}
% 20%   | 239, 208, 227   | EFD0E3   | 4, 20, 0, 1
\definecolor{ETHPurple20}{HTML}{EFD0E3}
% 40%   | 220, 158, 201   | DC9EC9   | 7, 40, 0, 4
\definecolor{ETHPurple40}{HTML}{DC9EC9}
% 60%   | 202, 108, 174   | CA6CAE   | 13, 60, 0, 6
\definecolor{ETHPurple60}{HTML}{CA6CAE}
% 80%   | 183, 59, 146    | B73B92   | 18, 80, 0, 8
\definecolor{ETHPurple80}{HTML}{B73B92}
% 120%  | 140, 10, 89     | 8C0A59   | 22, 100, 0, 35
\definecolor{ETHPurple120}{HTML}{8C0A59}

% ==============================================================
% 7. ETH Grey Shades
% ==============================================================
% Shade | RGB             | HEX      | CMYK
% ------|-----------------|----------|---------------
% 10%   | 241, 241, 241   | F1F1F1   | 0, 0, 0, 7
\definecolor{ETHGrey10}{HTML}{F1F1F1}
% 20%   | 226, 226, 226   | E2E2E2   | 0, 0, 0, 14
\definecolor{ETHGrey20}{HTML}{E2E2E2}
% 40%   | 197, 197, 197   | C5C5C5   | 0, 0, 0, 28
\definecolor{ETHGrey40}{HTML}{C5C5C5}
% 60%   | 169, 169, 169   | A9A9A9   | 0, 0, 0, 42
\definecolor{ETHGrey60}{HTML}{A9A9A9}
% 80%   | 140, 140, 140   | 8C8C8C   | 0, 0, 0, 56
\definecolor{ETHGrey80}{HTML}{8C8C8C}
% 120%  | 87, 87, 87      | 575757   | 0, 0, 0, 81
\definecolor{ETHGrey120}{HTML}{575757}
%   \textcolor{ETHBlue}{Hello from ETH!}
\NeedsTeXFormat{LaTeX2e}
\ProvidesFile{eth.tex}[2025/03/08 v1.0 ETH brand color definitions]

\RequirePackage{xcolor}

% ==============================================================
% Primary ETH Corporate Colors
% ==============================================================
\definecolor{ETHBlue}{HTML}{215CAF}    % ETH Blue: RGB: 33, 92, 175; CMYK: 100,57,0,0; Pantone: 2935; RAL: 5005 Signalblau
\definecolor{ETHPetrol}{HTML}{007894}   % ETH Petrol: RGB: 0,120,148; CMYK: 100,25,30,10; Pantone: 633; RAL: 5009 Azurblau
\definecolor{ETHGreen}{HTML}{627313}    % ETH Green: RGB: 98,115,19; CMYK: 55,10,100,30; Pantone: 364; RAL: 6010 Grasgrün
\definecolor{ETHBronze}{HTML}{8E6713}    % ETH Bronze: RGB: 142,103,19; CMYK: 30,36,100,25; Pantone: 4495; RAL: 7008 Khakigrau
\definecolor{ETHRed}{HTML}{B7352D}       % ETH Red: RGB: 183,53,45; CMYK: 0,90,80,17; Pantone: 1797; RAL: 3031 Orientrot
\definecolor{ETHPurple}{HTML}{A7117A}     % ETH Purple: RGB: 167,17,122; CMYK: 22,100,0,10; Pantone: 234; RAL: 4006 Verkehrspurpur
\definecolor{ETHGrey}{HTML}{6F6F6F}       % ETH Grey: RGB: 111,111,111; CMYK: 0,0,0,70; Pantone: Cool Gray 11; RAL: 7046 Telegrau 2


% ------------------------------------------------------------------
% Extended / Complementary Color Palette
% ------------------------------------------------------------------
\definecolor{ETHTeal}{HTML}{008C95}
\definecolor{ETHGreen}{HTML}{00B38B}
\definecolor{ETHDarkBlue}{HTML}{1D2447}
\definecolor{ETHLightBlue}{HTML}{5BB6D6}
\definecolor{ETHOrange}{HTML}{F39200}
\definecolor{ETHRed}{HTML}{C8002A}
\definecolor{ETHWarmGray}{HTML}{DAD7D2}
\definecolor{ETHBeige}{HTML}{D7CEC1}
\definecolor{ETHDarkBrown}{HTML}{7F4F3C}
\definecolor{ETHDarkPink}{HTML}{EB67BD}
\definecolor{ETHDarkPurple}{HTML}{5F2167}
\definecolor{ETHDarkMagenta}{HTML}{A3488E}
\definecolor{ETHDarkGray}{HTML}{333333}
\definecolor{ETHGray}{HTML}{75787B}
\definecolor{ETHLightGray}{HTML}{E2E2E2}
\definecolor{ETHWhite}{HTML}{FFFFFF}
\definecolor{ETHBlack}{HTML}{000000}

% ==============================================================
% 1. ETH Blue Shades
% ==============================================================
% Shade | RGB             | HEX      | CMYK
% ------|-----------------|----------|---------------
% 10%   | 233, 239, 247   | E9EFF7   | 10, 6, 0, 0
\definecolor{ETHBlue10}{HTML}{E9EFF7}
% 20%   | 211, 222, 239   | D3DEEF   | 20, 11, 0, 0
\definecolor{ETHBlue20}{HTML}{D3DEEF}
% 40%   | 166, 190, 223   | A6BEDF   | 40, 23, 0, 0
\definecolor{ETHBlue40}{HTML}{A6BEDF}
% 60%   | 122, 157, 207   | 7A9DCF   | 60, 34, 0, 0
\definecolor{ETHBlue60}{HTML}{7A9DCF}
% 80%   | 77, 125, 191    | 4D7DBF   | 80, 46, 0, 0
\definecolor{ETHBlue80}{HTML}{4D7DBF}
% 120%  | 8, 64, 126      | 08407E   | 100, 62, 0, 30
\definecolor{ETHBlue120}{HTML}{08407E}

% ==============================================================
% 2. ETH Petrol Shades
% ==============================================================
% Shade | RGB             | HEX      | CMYK
% ------|-----------------|----------|---------------
% 10%   | 231, 244, 247   | E7F4F7   | 12, 0, 5, 0
\definecolor{ETHPetrol10}{HTML}{E7F4F7}
% 20%   | 204, 228, 234   | CCE4EA   | 20, 3, 7, 0
\definecolor{ETHPetrol20}{HTML}{CCE4EA}
% 40%   | 153, 202, 213   | 99CAD5   | 40, 7, 12, 4
\definecolor{ETHPetrol40}{HTML}{99CAD5}
% 60%   | 102, 175, 192   | 66AFC0   | 60, 14, 18, 6
\definecolor{ETHPetrol60}{HTML}{66AFC0}
% 80%   | 51, 149, 171    | 3395AB   | 80, 20, 24, 8
\definecolor{ETHPetrol80}{HTML}{3395AB}
% 120%  | 0, 89, 109      | 00596D   | 100, 25, 30, 38
\definecolor{ETHPetrol120}{HTML}{00596D}

% ==============================================================
% 3. ETH Green Shades
% ==============================================================
% Shade | RGB             | HEX      | CMYK
% ------|-----------------|----------|---------------
% 10%   | 239, 241, 231   | EEF1E7   | 6, 1, 10, 3
\definecolor{ETHGreen10}{HTML}{EEF1E7}
% 20%   | 224, 227, 208   | E0E3D0   | 11, 2, 20, 6
\definecolor{ETHGreen20}{HTML}{E0E3D0}
% 40%   | 192, 199, 161   | C0C7A1   | 22, 4, 40, 12
\definecolor{ETHGreen40}{HTML}{C0C7A1}
% 60%   | 161, 171, 113   | A1AB71   | 33, 6, 60, 18
\definecolor{ETHGreen60}{HTML}{A1AB71}
% 80%   | 129, 143, 66    | 818F42   | 44, 8, 80, 24
\definecolor{ETHGreen80}{HTML}{818F42}
% 120%  | 54, 82, 19      | 365213   | 55, 10, 100, 65
\definecolor{ETHGreen120}{HTML}{365213}

% ==============================================================
% 4. ETH Bronze Shades
% ==============================================================
% Shade | RGB             | HEX      | CMYK
% ------|-----------------|----------|---------------
% 10%   | 244, 240, 231   | F4F0E7   | 3, 4, 10, 3
\definecolor{ETHBronze10}{HTML}{F4F0E7}
% 20%   | 232, 225, 208   | E8E1D0   | 6, 7, 20, 5
\definecolor{ETHBronze20}{HTML}{E8E1D0}
% 40%   | 210, 194, 161   | D2C2A1   | 12, 14, 40, 10
\definecolor{ETHBronze40}{HTML}{D2C2A1}
% 60%   | 187, 164, 113   | BBA471   | 18, 22, 60, 15
\definecolor{ETHBronze60}{HTML}{BBA471}
% 80%   | 165, 133, 66    | A58542   | 24, 29, 80, 20
\definecolor{ETHBronze80}{HTML}{A58542}
% 120%  | 112, 79, 18     | 704F12   | 30, 36, 100, 55
\definecolor{ETHBronze120}{HTML}{704F12}

% ==============================================================
% 5. ETH Red Shades
% ==============================================================
% Shade | RGB             | HEX      | CMYK
% ------|-----------------|----------|---------------
% 10%   | 248, 235, 234   | F8EBEA   | 0, 9, 6, 0
\definecolor{ETHRed10}{HTML}{F8EBEA}
% 20%   | 241, 215, 213   | F1D7D5   | 0, 18, 13, 4
\definecolor{ETHRed20}{HTML}{F1D7D5}
% 40%   | 226, 174, 171   | E2AEAB   | 0, 36, 26, 8
\definecolor{ETHRed40}{HTML}{E2AEAB}
% 60%   | 212, 134, 129   | D48681   | 0, 54, 39, 11
\definecolor{ETHRed60}{HTML}{D48681}
% 80%   | 197, 93, 87     | C55D57   | (using HEX)
\definecolor{ETHRed80}{HTML}{C55D57}
% 120%  | 150, 39, 45     | 96272D   | 0, 100, 80, 40
\definecolor{ETHRed120}{HTML}{96272D}

% ==============================================================
% 6. ETH Purple Shades
% ==============================================================
% Shade | RGB             | HEX      | CMYK
% ------|-----------------|----------|---------------
% 10%   | 248, 232, 243   | F8E8F3   | 2, 10, 0, 1
\definecolor{ETHPurple10}{HTML}{F8E8F3}
% 20%   | 239, 208, 227   | EFD0E3   | 4, 20, 0, 1
\definecolor{ETHPurple20}{HTML}{EFD0E3}
% 40%   | 220, 158, 201   | DC9EC9   | 7, 40, 0, 4
\definecolor{ETHPurple40}{HTML}{DC9EC9}
% 60%   | 202, 108, 174   | CA6CAE   | 13, 60, 0, 6
\definecolor{ETHPurple60}{HTML}{CA6CAE}
% 80%   | 183, 59, 146    | B73B92   | 18, 80, 0, 8
\definecolor{ETHPurple80}{HTML}{B73B92}
% 120%  | 140, 10, 89     | 8C0A59   | 22, 100, 0, 35
\definecolor{ETHPurple120}{HTML}{8C0A59}

% ==============================================================
% 7. ETH Grey Shades
% ==============================================================
% Shade | RGB             | HEX      | CMYK
% ------|-----------------|----------|---------------
% 10%   | 241, 241, 241   | F1F1F1   | 0, 0, 0, 7
\definecolor{ETHGrey10}{HTML}{F1F1F1}
% 20%   | 226, 226, 226   | E2E2E2   | 0, 0, 0, 14
\definecolor{ETHGrey20}{HTML}{E2E2E2}
% 40%   | 197, 197, 197   | C5C5C5   | 0, 0, 0, 28
\definecolor{ETHGrey40}{HTML}{C5C5C5}
% 60%   | 169, 169, 169   | A9A9A9   | 0, 0, 0, 42
\definecolor{ETHGrey60}{HTML}{A9A9A9}
% 80%   | 140, 140, 140   | 8C8C8C   | 0, 0, 0, 56
\definecolor{ETHGrey80}{HTML}{8C8C8C}
% 120%  | 87, 87, 87      | 575757   | 0, 0, 0, 81
\definecolor{ETHGrey120}{HTML}{575757}  % Load ETH corporate colours and shade definitions
% eth.tex
% Defines ETH brand colors based on:
% https://ethz.ch/staffnet/en/service/communication/corporate-design/colours.html
% 
% ------------------------------------------------------------------
% ETH Corporate Design – Primary Colors and Colour Shades Definitions
%
% PRIMARY ETH CORPORATE COLORS
%
% Colour       RGB             HEX       CMYK                 Pantone    RAL
% ----------------------------------------------------------------------------
% ETH Blue     33, 92, 175     #215CAF   100,57,0,0           2935       5005 Signalblau
% ETH Petrol   0, 120, 148     #007894   100,25,30,10         633        5009 Azurblau
% ETH Green    98, 115, 19     #627313   55,10,100,30         364        6010 Grasgrün
% ETH Bronze   142, 103, 19    #8E6713   30,36,100,25         4495       7008 Khakigrau
% ETH Red      183, 53, 45     #B7352D   0,90,80,17           1797       3031 Orientrot
% ETH Purple   167, 17, 122    #A7117A   22,100,0,10          234        4006 Verkehrspurpur
% ETH Grey     111, 111, 111   #6F6F6F   0,0,0,70             Cool Gray 11   7046 Telegrau 2
% ------------------------------------------------------------------
% ETH Corporate Design – Colour Shades Definitions
%
% 1. ETH Blue Shades
% 2. ETH Petrol Shades
% 3. ETH Green Shades
% 4. ETH Bronze Shades
% 5. ETH Red Shades
% 6. ETH Purple Shades
% 7. ETH Grey Shades
% 
% Last updated: 2025-03-08
%
% Usage:
%   % eth.tex
% Defines ETH brand colors based on:
% https://ethz.ch/staffnet/en/service/communication/corporate-design/colours.html
% 
% ------------------------------------------------------------------
% ETH Corporate Design – Primary Colors and Colour Shades Definitions
%
% PRIMARY ETH CORPORATE COLORS
%
% Colour       RGB             HEX       CMYK                 Pantone    RAL
% ----------------------------------------------------------------------------
% ETH Blue     33, 92, 175     #215CAF   100,57,0,0           2935       5005 Signalblau
% ETH Petrol   0, 120, 148     #007894   100,25,30,10         633        5009 Azurblau
% ETH Green    98, 115, 19     #627313   55,10,100,30         364        6010 Grasgrün
% ETH Bronze   142, 103, 19    #8E6713   30,36,100,25         4495       7008 Khakigrau
% ETH Red      183, 53, 45     #B7352D   0,90,80,17           1797       3031 Orientrot
% ETH Purple   167, 17, 122    #A7117A   22,100,0,10          234        4006 Verkehrspurpur
% ETH Grey     111, 111, 111   #6F6F6F   0,0,0,70             Cool Gray 11   7046 Telegrau 2
% ------------------------------------------------------------------
% ETH Corporate Design – Colour Shades Definitions
%
% 1. ETH Blue Shades
% 2. ETH Petrol Shades
% 3. ETH Green Shades
% 4. ETH Bronze Shades
% 5. ETH Red Shades
% 6. ETH Purple Shades
% 7. ETH Grey Shades
% 
% Last updated: 2025-03-08
%
% Usage:
%   % eth.tex
% Defines ETH brand colors based on:
% https://ethz.ch/staffnet/en/service/communication/corporate-design/colours.html
% 
% ------------------------------------------------------------------
% ETH Corporate Design – Primary Colors and Colour Shades Definitions
%
% PRIMARY ETH CORPORATE COLORS
%
% Colour       RGB             HEX       CMYK                 Pantone    RAL
% ----------------------------------------------------------------------------
% ETH Blue     33, 92, 175     #215CAF   100,57,0,0           2935       5005 Signalblau
% ETH Petrol   0, 120, 148     #007894   100,25,30,10         633        5009 Azurblau
% ETH Green    98, 115, 19     #627313   55,10,100,30         364        6010 Grasgrün
% ETH Bronze   142, 103, 19    #8E6713   30,36,100,25         4495       7008 Khakigrau
% ETH Red      183, 53, 45     #B7352D   0,90,80,17           1797       3031 Orientrot
% ETH Purple   167, 17, 122    #A7117A   22,100,0,10          234        4006 Verkehrspurpur
% ETH Grey     111, 111, 111   #6F6F6F   0,0,0,70             Cool Gray 11   7046 Telegrau 2
% ------------------------------------------------------------------
% ETH Corporate Design – Colour Shades Definitions
%
% 1. ETH Blue Shades
% 2. ETH Petrol Shades
% 3. ETH Green Shades
% 4. ETH Bronze Shades
% 5. ETH Red Shades
% 6. ETH Purple Shades
% 7. ETH Grey Shades
% 
% Last updated: 2025-03-08
%
% Usage:
%   \input{eth.tex}
%   \textcolor{ETHBlue}{Hello ETH!}
% Usage in your main .tex:
%   \usepackage{xcolor} % or colortbl, etc.
%   \input{eth.tex}
%   \textcolor{ETHBlue}{Hello from ETH!}
\NeedsTeXFormat{LaTeX2e}
\ProvidesFile{eth.tex}[2025/03/08 v1.0 ETH brand color definitions]

\RequirePackage{xcolor}

% ==============================================================
% Primary ETH Corporate Colors
% ==============================================================
\definecolor{ETHBlue}{HTML}{215CAF}    % ETH Blue: RGB: 33, 92, 175; CMYK: 100,57,0,0; Pantone: 2935; RAL: 5005 Signalblau
\definecolor{ETHPetrol}{HTML}{007894}   % ETH Petrol: RGB: 0,120,148; CMYK: 100,25,30,10; Pantone: 633; RAL: 5009 Azurblau
\definecolor{ETHGreen}{HTML}{627313}    % ETH Green: RGB: 98,115,19; CMYK: 55,10,100,30; Pantone: 364; RAL: 6010 Grasgrün
\definecolor{ETHBronze}{HTML}{8E6713}    % ETH Bronze: RGB: 142,103,19; CMYK: 30,36,100,25; Pantone: 4495; RAL: 7008 Khakigrau
\definecolor{ETHRed}{HTML}{B7352D}       % ETH Red: RGB: 183,53,45; CMYK: 0,90,80,17; Pantone: 1797; RAL: 3031 Orientrot
\definecolor{ETHPurple}{HTML}{A7117A}     % ETH Purple: RGB: 167,17,122; CMYK: 22,100,0,10; Pantone: 234; RAL: 4006 Verkehrspurpur
\definecolor{ETHGrey}{HTML}{6F6F6F}       % ETH Grey: RGB: 111,111,111; CMYK: 0,0,0,70; Pantone: Cool Gray 11; RAL: 7046 Telegrau 2


% ------------------------------------------------------------------
% Extended / Complementary Color Palette
% ------------------------------------------------------------------
\definecolor{ETHTeal}{HTML}{008C95}
\definecolor{ETHGreen}{HTML}{00B38B}
\definecolor{ETHDarkBlue}{HTML}{1D2447}
\definecolor{ETHLightBlue}{HTML}{5BB6D6}
\definecolor{ETHOrange}{HTML}{F39200}
\definecolor{ETHRed}{HTML}{C8002A}
\definecolor{ETHWarmGray}{HTML}{DAD7D2}
\definecolor{ETHBeige}{HTML}{D7CEC1}
\definecolor{ETHDarkBrown}{HTML}{7F4F3C}
\definecolor{ETHDarkPink}{HTML}{EB67BD}
\definecolor{ETHDarkPurple}{HTML}{5F2167}
\definecolor{ETHDarkMagenta}{HTML}{A3488E}
\definecolor{ETHDarkGray}{HTML}{333333}
\definecolor{ETHGray}{HTML}{75787B}
\definecolor{ETHLightGray}{HTML}{E2E2E2}
\definecolor{ETHWhite}{HTML}{FFFFFF}
\definecolor{ETHBlack}{HTML}{000000}

% ==============================================================
% 1. ETH Blue Shades
% ==============================================================
% Shade | RGB             | HEX      | CMYK
% ------|-----------------|----------|---------------
% 10%   | 233, 239, 247   | E9EFF7   | 10, 6, 0, 0
\definecolor{ETHBlue10}{HTML}{E9EFF7}
% 20%   | 211, 222, 239   | D3DEEF   | 20, 11, 0, 0
\definecolor{ETHBlue20}{HTML}{D3DEEF}
% 40%   | 166, 190, 223   | A6BEDF   | 40, 23, 0, 0
\definecolor{ETHBlue40}{HTML}{A6BEDF}
% 60%   | 122, 157, 207   | 7A9DCF   | 60, 34, 0, 0
\definecolor{ETHBlue60}{HTML}{7A9DCF}
% 80%   | 77, 125, 191    | 4D7DBF   | 80, 46, 0, 0
\definecolor{ETHBlue80}{HTML}{4D7DBF}
% 120%  | 8, 64, 126      | 08407E   | 100, 62, 0, 30
\definecolor{ETHBlue120}{HTML}{08407E}

% ==============================================================
% 2. ETH Petrol Shades
% ==============================================================
% Shade | RGB             | HEX      | CMYK
% ------|-----------------|----------|---------------
% 10%   | 231, 244, 247   | E7F4F7   | 12, 0, 5, 0
\definecolor{ETHPetrol10}{HTML}{E7F4F7}
% 20%   | 204, 228, 234   | CCE4EA   | 20, 3, 7, 0
\definecolor{ETHPetrol20}{HTML}{CCE4EA}
% 40%   | 153, 202, 213   | 99CAD5   | 40, 7, 12, 4
\definecolor{ETHPetrol40}{HTML}{99CAD5}
% 60%   | 102, 175, 192   | 66AFC0   | 60, 14, 18, 6
\definecolor{ETHPetrol60}{HTML}{66AFC0}
% 80%   | 51, 149, 171    | 3395AB   | 80, 20, 24, 8
\definecolor{ETHPetrol80}{HTML}{3395AB}
% 120%  | 0, 89, 109      | 00596D   | 100, 25, 30, 38
\definecolor{ETHPetrol120}{HTML}{00596D}

% ==============================================================
% 3. ETH Green Shades
% ==============================================================
% Shade | RGB             | HEX      | CMYK
% ------|-----------------|----------|---------------
% 10%   | 239, 241, 231   | EEF1E7   | 6, 1, 10, 3
\definecolor{ETHGreen10}{HTML}{EEF1E7}
% 20%   | 224, 227, 208   | E0E3D0   | 11, 2, 20, 6
\definecolor{ETHGreen20}{HTML}{E0E3D0}
% 40%   | 192, 199, 161   | C0C7A1   | 22, 4, 40, 12
\definecolor{ETHGreen40}{HTML}{C0C7A1}
% 60%   | 161, 171, 113   | A1AB71   | 33, 6, 60, 18
\definecolor{ETHGreen60}{HTML}{A1AB71}
% 80%   | 129, 143, 66    | 818F42   | 44, 8, 80, 24
\definecolor{ETHGreen80}{HTML}{818F42}
% 120%  | 54, 82, 19      | 365213   | 55, 10, 100, 65
\definecolor{ETHGreen120}{HTML}{365213}

% ==============================================================
% 4. ETH Bronze Shades
% ==============================================================
% Shade | RGB             | HEX      | CMYK
% ------|-----------------|----------|---------------
% 10%   | 244, 240, 231   | F4F0E7   | 3, 4, 10, 3
\definecolor{ETHBronze10}{HTML}{F4F0E7}
% 20%   | 232, 225, 208   | E8E1D0   | 6, 7, 20, 5
\definecolor{ETHBronze20}{HTML}{E8E1D0}
% 40%   | 210, 194, 161   | D2C2A1   | 12, 14, 40, 10
\definecolor{ETHBronze40}{HTML}{D2C2A1}
% 60%   | 187, 164, 113   | BBA471   | 18, 22, 60, 15
\definecolor{ETHBronze60}{HTML}{BBA471}
% 80%   | 165, 133, 66    | A58542   | 24, 29, 80, 20
\definecolor{ETHBronze80}{HTML}{A58542}
% 120%  | 112, 79, 18     | 704F12   | 30, 36, 100, 55
\definecolor{ETHBronze120}{HTML}{704F12}

% ==============================================================
% 5. ETH Red Shades
% ==============================================================
% Shade | RGB             | HEX      | CMYK
% ------|-----------------|----------|---------------
% 10%   | 248, 235, 234   | F8EBEA   | 0, 9, 6, 0
\definecolor{ETHRed10}{HTML}{F8EBEA}
% 20%   | 241, 215, 213   | F1D7D5   | 0, 18, 13, 4
\definecolor{ETHRed20}{HTML}{F1D7D5}
% 40%   | 226, 174, 171   | E2AEAB   | 0, 36, 26, 8
\definecolor{ETHRed40}{HTML}{E2AEAB}
% 60%   | 212, 134, 129   | D48681   | 0, 54, 39, 11
\definecolor{ETHRed60}{HTML}{D48681}
% 80%   | 197, 93, 87     | C55D57   | (using HEX)
\definecolor{ETHRed80}{HTML}{C55D57}
% 120%  | 150, 39, 45     | 96272D   | 0, 100, 80, 40
\definecolor{ETHRed120}{HTML}{96272D}

% ==============================================================
% 6. ETH Purple Shades
% ==============================================================
% Shade | RGB             | HEX      | CMYK
% ------|-----------------|----------|---------------
% 10%   | 248, 232, 243   | F8E8F3   | 2, 10, 0, 1
\definecolor{ETHPurple10}{HTML}{F8E8F3}
% 20%   | 239, 208, 227   | EFD0E3   | 4, 20, 0, 1
\definecolor{ETHPurple20}{HTML}{EFD0E3}
% 40%   | 220, 158, 201   | DC9EC9   | 7, 40, 0, 4
\definecolor{ETHPurple40}{HTML}{DC9EC9}
% 60%   | 202, 108, 174   | CA6CAE   | 13, 60, 0, 6
\definecolor{ETHPurple60}{HTML}{CA6CAE}
% 80%   | 183, 59, 146    | B73B92   | 18, 80, 0, 8
\definecolor{ETHPurple80}{HTML}{B73B92}
% 120%  | 140, 10, 89     | 8C0A59   | 22, 100, 0, 35
\definecolor{ETHPurple120}{HTML}{8C0A59}

% ==============================================================
% 7. ETH Grey Shades
% ==============================================================
% Shade | RGB             | HEX      | CMYK
% ------|-----------------|----------|---------------
% 10%   | 241, 241, 241   | F1F1F1   | 0, 0, 0, 7
\definecolor{ETHGrey10}{HTML}{F1F1F1}
% 20%   | 226, 226, 226   | E2E2E2   | 0, 0, 0, 14
\definecolor{ETHGrey20}{HTML}{E2E2E2}
% 40%   | 197, 197, 197   | C5C5C5   | 0, 0, 0, 28
\definecolor{ETHGrey40}{HTML}{C5C5C5}
% 60%   | 169, 169, 169   | A9A9A9   | 0, 0, 0, 42
\definecolor{ETHGrey60}{HTML}{A9A9A9}
% 80%   | 140, 140, 140   | 8C8C8C   | 0, 0, 0, 56
\definecolor{ETHGrey80}{HTML}{8C8C8C}
% 120%  | 87, 87, 87      | 575757   | 0, 0, 0, 81
\definecolor{ETHGrey120}{HTML}{575757}
%   \textcolor{ETHBlue}{Hello ETH!}
% Usage in your main .tex:
%   \usepackage{xcolor} % or colortbl, etc.
%   % eth.tex
% Defines ETH brand colors based on:
% https://ethz.ch/staffnet/en/service/communication/corporate-design/colours.html
% 
% ------------------------------------------------------------------
% ETH Corporate Design – Primary Colors and Colour Shades Definitions
%
% PRIMARY ETH CORPORATE COLORS
%
% Colour       RGB             HEX       CMYK                 Pantone    RAL
% ----------------------------------------------------------------------------
% ETH Blue     33, 92, 175     #215CAF   100,57,0,0           2935       5005 Signalblau
% ETH Petrol   0, 120, 148     #007894   100,25,30,10         633        5009 Azurblau
% ETH Green    98, 115, 19     #627313   55,10,100,30         364        6010 Grasgrün
% ETH Bronze   142, 103, 19    #8E6713   30,36,100,25         4495       7008 Khakigrau
% ETH Red      183, 53, 45     #B7352D   0,90,80,17           1797       3031 Orientrot
% ETH Purple   167, 17, 122    #A7117A   22,100,0,10          234        4006 Verkehrspurpur
% ETH Grey     111, 111, 111   #6F6F6F   0,0,0,70             Cool Gray 11   7046 Telegrau 2
% ------------------------------------------------------------------
% ETH Corporate Design – Colour Shades Definitions
%
% 1. ETH Blue Shades
% 2. ETH Petrol Shades
% 3. ETH Green Shades
% 4. ETH Bronze Shades
% 5. ETH Red Shades
% 6. ETH Purple Shades
% 7. ETH Grey Shades
% 
% Last updated: 2025-03-08
%
% Usage:
%   \input{eth.tex}
%   \textcolor{ETHBlue}{Hello ETH!}
% Usage in your main .tex:
%   \usepackage{xcolor} % or colortbl, etc.
%   \input{eth.tex}
%   \textcolor{ETHBlue}{Hello from ETH!}
\NeedsTeXFormat{LaTeX2e}
\ProvidesFile{eth.tex}[2025/03/08 v1.0 ETH brand color definitions]

\RequirePackage{xcolor}

% ==============================================================
% Primary ETH Corporate Colors
% ==============================================================
\definecolor{ETHBlue}{HTML}{215CAF}    % ETH Blue: RGB: 33, 92, 175; CMYK: 100,57,0,0; Pantone: 2935; RAL: 5005 Signalblau
\definecolor{ETHPetrol}{HTML}{007894}   % ETH Petrol: RGB: 0,120,148; CMYK: 100,25,30,10; Pantone: 633; RAL: 5009 Azurblau
\definecolor{ETHGreen}{HTML}{627313}    % ETH Green: RGB: 98,115,19; CMYK: 55,10,100,30; Pantone: 364; RAL: 6010 Grasgrün
\definecolor{ETHBronze}{HTML}{8E6713}    % ETH Bronze: RGB: 142,103,19; CMYK: 30,36,100,25; Pantone: 4495; RAL: 7008 Khakigrau
\definecolor{ETHRed}{HTML}{B7352D}       % ETH Red: RGB: 183,53,45; CMYK: 0,90,80,17; Pantone: 1797; RAL: 3031 Orientrot
\definecolor{ETHPurple}{HTML}{A7117A}     % ETH Purple: RGB: 167,17,122; CMYK: 22,100,0,10; Pantone: 234; RAL: 4006 Verkehrspurpur
\definecolor{ETHGrey}{HTML}{6F6F6F}       % ETH Grey: RGB: 111,111,111; CMYK: 0,0,0,70; Pantone: Cool Gray 11; RAL: 7046 Telegrau 2


% ------------------------------------------------------------------
% Extended / Complementary Color Palette
% ------------------------------------------------------------------
\definecolor{ETHTeal}{HTML}{008C95}
\definecolor{ETHGreen}{HTML}{00B38B}
\definecolor{ETHDarkBlue}{HTML}{1D2447}
\definecolor{ETHLightBlue}{HTML}{5BB6D6}
\definecolor{ETHOrange}{HTML}{F39200}
\definecolor{ETHRed}{HTML}{C8002A}
\definecolor{ETHWarmGray}{HTML}{DAD7D2}
\definecolor{ETHBeige}{HTML}{D7CEC1}
\definecolor{ETHDarkBrown}{HTML}{7F4F3C}
\definecolor{ETHDarkPink}{HTML}{EB67BD}
\definecolor{ETHDarkPurple}{HTML}{5F2167}
\definecolor{ETHDarkMagenta}{HTML}{A3488E}
\definecolor{ETHDarkGray}{HTML}{333333}
\definecolor{ETHGray}{HTML}{75787B}
\definecolor{ETHLightGray}{HTML}{E2E2E2}
\definecolor{ETHWhite}{HTML}{FFFFFF}
\definecolor{ETHBlack}{HTML}{000000}

% ==============================================================
% 1. ETH Blue Shades
% ==============================================================
% Shade | RGB             | HEX      | CMYK
% ------|-----------------|----------|---------------
% 10%   | 233, 239, 247   | E9EFF7   | 10, 6, 0, 0
\definecolor{ETHBlue10}{HTML}{E9EFF7}
% 20%   | 211, 222, 239   | D3DEEF   | 20, 11, 0, 0
\definecolor{ETHBlue20}{HTML}{D3DEEF}
% 40%   | 166, 190, 223   | A6BEDF   | 40, 23, 0, 0
\definecolor{ETHBlue40}{HTML}{A6BEDF}
% 60%   | 122, 157, 207   | 7A9DCF   | 60, 34, 0, 0
\definecolor{ETHBlue60}{HTML}{7A9DCF}
% 80%   | 77, 125, 191    | 4D7DBF   | 80, 46, 0, 0
\definecolor{ETHBlue80}{HTML}{4D7DBF}
% 120%  | 8, 64, 126      | 08407E   | 100, 62, 0, 30
\definecolor{ETHBlue120}{HTML}{08407E}

% ==============================================================
% 2. ETH Petrol Shades
% ==============================================================
% Shade | RGB             | HEX      | CMYK
% ------|-----------------|----------|---------------
% 10%   | 231, 244, 247   | E7F4F7   | 12, 0, 5, 0
\definecolor{ETHPetrol10}{HTML}{E7F4F7}
% 20%   | 204, 228, 234   | CCE4EA   | 20, 3, 7, 0
\definecolor{ETHPetrol20}{HTML}{CCE4EA}
% 40%   | 153, 202, 213   | 99CAD5   | 40, 7, 12, 4
\definecolor{ETHPetrol40}{HTML}{99CAD5}
% 60%   | 102, 175, 192   | 66AFC0   | 60, 14, 18, 6
\definecolor{ETHPetrol60}{HTML}{66AFC0}
% 80%   | 51, 149, 171    | 3395AB   | 80, 20, 24, 8
\definecolor{ETHPetrol80}{HTML}{3395AB}
% 120%  | 0, 89, 109      | 00596D   | 100, 25, 30, 38
\definecolor{ETHPetrol120}{HTML}{00596D}

% ==============================================================
% 3. ETH Green Shades
% ==============================================================
% Shade | RGB             | HEX      | CMYK
% ------|-----------------|----------|---------------
% 10%   | 239, 241, 231   | EEF1E7   | 6, 1, 10, 3
\definecolor{ETHGreen10}{HTML}{EEF1E7}
% 20%   | 224, 227, 208   | E0E3D0   | 11, 2, 20, 6
\definecolor{ETHGreen20}{HTML}{E0E3D0}
% 40%   | 192, 199, 161   | C0C7A1   | 22, 4, 40, 12
\definecolor{ETHGreen40}{HTML}{C0C7A1}
% 60%   | 161, 171, 113   | A1AB71   | 33, 6, 60, 18
\definecolor{ETHGreen60}{HTML}{A1AB71}
% 80%   | 129, 143, 66    | 818F42   | 44, 8, 80, 24
\definecolor{ETHGreen80}{HTML}{818F42}
% 120%  | 54, 82, 19      | 365213   | 55, 10, 100, 65
\definecolor{ETHGreen120}{HTML}{365213}

% ==============================================================
% 4. ETH Bronze Shades
% ==============================================================
% Shade | RGB             | HEX      | CMYK
% ------|-----------------|----------|---------------
% 10%   | 244, 240, 231   | F4F0E7   | 3, 4, 10, 3
\definecolor{ETHBronze10}{HTML}{F4F0E7}
% 20%   | 232, 225, 208   | E8E1D0   | 6, 7, 20, 5
\definecolor{ETHBronze20}{HTML}{E8E1D0}
% 40%   | 210, 194, 161   | D2C2A1   | 12, 14, 40, 10
\definecolor{ETHBronze40}{HTML}{D2C2A1}
% 60%   | 187, 164, 113   | BBA471   | 18, 22, 60, 15
\definecolor{ETHBronze60}{HTML}{BBA471}
% 80%   | 165, 133, 66    | A58542   | 24, 29, 80, 20
\definecolor{ETHBronze80}{HTML}{A58542}
% 120%  | 112, 79, 18     | 704F12   | 30, 36, 100, 55
\definecolor{ETHBronze120}{HTML}{704F12}

% ==============================================================
% 5. ETH Red Shades
% ==============================================================
% Shade | RGB             | HEX      | CMYK
% ------|-----------------|----------|---------------
% 10%   | 248, 235, 234   | F8EBEA   | 0, 9, 6, 0
\definecolor{ETHRed10}{HTML}{F8EBEA}
% 20%   | 241, 215, 213   | F1D7D5   | 0, 18, 13, 4
\definecolor{ETHRed20}{HTML}{F1D7D5}
% 40%   | 226, 174, 171   | E2AEAB   | 0, 36, 26, 8
\definecolor{ETHRed40}{HTML}{E2AEAB}
% 60%   | 212, 134, 129   | D48681   | 0, 54, 39, 11
\definecolor{ETHRed60}{HTML}{D48681}
% 80%   | 197, 93, 87     | C55D57   | (using HEX)
\definecolor{ETHRed80}{HTML}{C55D57}
% 120%  | 150, 39, 45     | 96272D   | 0, 100, 80, 40
\definecolor{ETHRed120}{HTML}{96272D}

% ==============================================================
% 6. ETH Purple Shades
% ==============================================================
% Shade | RGB             | HEX      | CMYK
% ------|-----------------|----------|---------------
% 10%   | 248, 232, 243   | F8E8F3   | 2, 10, 0, 1
\definecolor{ETHPurple10}{HTML}{F8E8F3}
% 20%   | 239, 208, 227   | EFD0E3   | 4, 20, 0, 1
\definecolor{ETHPurple20}{HTML}{EFD0E3}
% 40%   | 220, 158, 201   | DC9EC9   | 7, 40, 0, 4
\definecolor{ETHPurple40}{HTML}{DC9EC9}
% 60%   | 202, 108, 174   | CA6CAE   | 13, 60, 0, 6
\definecolor{ETHPurple60}{HTML}{CA6CAE}
% 80%   | 183, 59, 146    | B73B92   | 18, 80, 0, 8
\definecolor{ETHPurple80}{HTML}{B73B92}
% 120%  | 140, 10, 89     | 8C0A59   | 22, 100, 0, 35
\definecolor{ETHPurple120}{HTML}{8C0A59}

% ==============================================================
% 7. ETH Grey Shades
% ==============================================================
% Shade | RGB             | HEX      | CMYK
% ------|-----------------|----------|---------------
% 10%   | 241, 241, 241   | F1F1F1   | 0, 0, 0, 7
\definecolor{ETHGrey10}{HTML}{F1F1F1}
% 20%   | 226, 226, 226   | E2E2E2   | 0, 0, 0, 14
\definecolor{ETHGrey20}{HTML}{E2E2E2}
% 40%   | 197, 197, 197   | C5C5C5   | 0, 0, 0, 28
\definecolor{ETHGrey40}{HTML}{C5C5C5}
% 60%   | 169, 169, 169   | A9A9A9   | 0, 0, 0, 42
\definecolor{ETHGrey60}{HTML}{A9A9A9}
% 80%   | 140, 140, 140   | 8C8C8C   | 0, 0, 0, 56
\definecolor{ETHGrey80}{HTML}{8C8C8C}
% 120%  | 87, 87, 87      | 575757   | 0, 0, 0, 81
\definecolor{ETHGrey120}{HTML}{575757}
%   \textcolor{ETHBlue}{Hello from ETH!}
\NeedsTeXFormat{LaTeX2e}
\ProvidesFile{eth.tex}[2025/03/08 v1.0 ETH brand color definitions]

\RequirePackage{xcolor}

% ==============================================================
% Primary ETH Corporate Colors
% ==============================================================
\definecolor{ETHBlue}{HTML}{215CAF}    % ETH Blue: RGB: 33, 92, 175; CMYK: 100,57,0,0; Pantone: 2935; RAL: 5005 Signalblau
\definecolor{ETHPetrol}{HTML}{007894}   % ETH Petrol: RGB: 0,120,148; CMYK: 100,25,30,10; Pantone: 633; RAL: 5009 Azurblau
\definecolor{ETHGreen}{HTML}{627313}    % ETH Green: RGB: 98,115,19; CMYK: 55,10,100,30; Pantone: 364; RAL: 6010 Grasgrün
\definecolor{ETHBronze}{HTML}{8E6713}    % ETH Bronze: RGB: 142,103,19; CMYK: 30,36,100,25; Pantone: 4495; RAL: 7008 Khakigrau
\definecolor{ETHRed}{HTML}{B7352D}       % ETH Red: RGB: 183,53,45; CMYK: 0,90,80,17; Pantone: 1797; RAL: 3031 Orientrot
\definecolor{ETHPurple}{HTML}{A7117A}     % ETH Purple: RGB: 167,17,122; CMYK: 22,100,0,10; Pantone: 234; RAL: 4006 Verkehrspurpur
\definecolor{ETHGrey}{HTML}{6F6F6F}       % ETH Grey: RGB: 111,111,111; CMYK: 0,0,0,70; Pantone: Cool Gray 11; RAL: 7046 Telegrau 2


% ------------------------------------------------------------------
% Extended / Complementary Color Palette
% ------------------------------------------------------------------
\definecolor{ETHTeal}{HTML}{008C95}
\definecolor{ETHGreen}{HTML}{00B38B}
\definecolor{ETHDarkBlue}{HTML}{1D2447}
\definecolor{ETHLightBlue}{HTML}{5BB6D6}
\definecolor{ETHOrange}{HTML}{F39200}
\definecolor{ETHRed}{HTML}{C8002A}
\definecolor{ETHWarmGray}{HTML}{DAD7D2}
\definecolor{ETHBeige}{HTML}{D7CEC1}
\definecolor{ETHDarkBrown}{HTML}{7F4F3C}
\definecolor{ETHDarkPink}{HTML}{EB67BD}
\definecolor{ETHDarkPurple}{HTML}{5F2167}
\definecolor{ETHDarkMagenta}{HTML}{A3488E}
\definecolor{ETHDarkGray}{HTML}{333333}
\definecolor{ETHGray}{HTML}{75787B}
\definecolor{ETHLightGray}{HTML}{E2E2E2}
\definecolor{ETHWhite}{HTML}{FFFFFF}
\definecolor{ETHBlack}{HTML}{000000}

% ==============================================================
% 1. ETH Blue Shades
% ==============================================================
% Shade | RGB             | HEX      | CMYK
% ------|-----------------|----------|---------------
% 10%   | 233, 239, 247   | E9EFF7   | 10, 6, 0, 0
\definecolor{ETHBlue10}{HTML}{E9EFF7}
% 20%   | 211, 222, 239   | D3DEEF   | 20, 11, 0, 0
\definecolor{ETHBlue20}{HTML}{D3DEEF}
% 40%   | 166, 190, 223   | A6BEDF   | 40, 23, 0, 0
\definecolor{ETHBlue40}{HTML}{A6BEDF}
% 60%   | 122, 157, 207   | 7A9DCF   | 60, 34, 0, 0
\definecolor{ETHBlue60}{HTML}{7A9DCF}
% 80%   | 77, 125, 191    | 4D7DBF   | 80, 46, 0, 0
\definecolor{ETHBlue80}{HTML}{4D7DBF}
% 120%  | 8, 64, 126      | 08407E   | 100, 62, 0, 30
\definecolor{ETHBlue120}{HTML}{08407E}

% ==============================================================
% 2. ETH Petrol Shades
% ==============================================================
% Shade | RGB             | HEX      | CMYK
% ------|-----------------|----------|---------------
% 10%   | 231, 244, 247   | E7F4F7   | 12, 0, 5, 0
\definecolor{ETHPetrol10}{HTML}{E7F4F7}
% 20%   | 204, 228, 234   | CCE4EA   | 20, 3, 7, 0
\definecolor{ETHPetrol20}{HTML}{CCE4EA}
% 40%   | 153, 202, 213   | 99CAD5   | 40, 7, 12, 4
\definecolor{ETHPetrol40}{HTML}{99CAD5}
% 60%   | 102, 175, 192   | 66AFC0   | 60, 14, 18, 6
\definecolor{ETHPetrol60}{HTML}{66AFC0}
% 80%   | 51, 149, 171    | 3395AB   | 80, 20, 24, 8
\definecolor{ETHPetrol80}{HTML}{3395AB}
% 120%  | 0, 89, 109      | 00596D   | 100, 25, 30, 38
\definecolor{ETHPetrol120}{HTML}{00596D}

% ==============================================================
% 3. ETH Green Shades
% ==============================================================
% Shade | RGB             | HEX      | CMYK
% ------|-----------------|----------|---------------
% 10%   | 239, 241, 231   | EEF1E7   | 6, 1, 10, 3
\definecolor{ETHGreen10}{HTML}{EEF1E7}
% 20%   | 224, 227, 208   | E0E3D0   | 11, 2, 20, 6
\definecolor{ETHGreen20}{HTML}{E0E3D0}
% 40%   | 192, 199, 161   | C0C7A1   | 22, 4, 40, 12
\definecolor{ETHGreen40}{HTML}{C0C7A1}
% 60%   | 161, 171, 113   | A1AB71   | 33, 6, 60, 18
\definecolor{ETHGreen60}{HTML}{A1AB71}
% 80%   | 129, 143, 66    | 818F42   | 44, 8, 80, 24
\definecolor{ETHGreen80}{HTML}{818F42}
% 120%  | 54, 82, 19      | 365213   | 55, 10, 100, 65
\definecolor{ETHGreen120}{HTML}{365213}

% ==============================================================
% 4. ETH Bronze Shades
% ==============================================================
% Shade | RGB             | HEX      | CMYK
% ------|-----------------|----------|---------------
% 10%   | 244, 240, 231   | F4F0E7   | 3, 4, 10, 3
\definecolor{ETHBronze10}{HTML}{F4F0E7}
% 20%   | 232, 225, 208   | E8E1D0   | 6, 7, 20, 5
\definecolor{ETHBronze20}{HTML}{E8E1D0}
% 40%   | 210, 194, 161   | D2C2A1   | 12, 14, 40, 10
\definecolor{ETHBronze40}{HTML}{D2C2A1}
% 60%   | 187, 164, 113   | BBA471   | 18, 22, 60, 15
\definecolor{ETHBronze60}{HTML}{BBA471}
% 80%   | 165, 133, 66    | A58542   | 24, 29, 80, 20
\definecolor{ETHBronze80}{HTML}{A58542}
% 120%  | 112, 79, 18     | 704F12   | 30, 36, 100, 55
\definecolor{ETHBronze120}{HTML}{704F12}

% ==============================================================
% 5. ETH Red Shades
% ==============================================================
% Shade | RGB             | HEX      | CMYK
% ------|-----------------|----------|---------------
% 10%   | 248, 235, 234   | F8EBEA   | 0, 9, 6, 0
\definecolor{ETHRed10}{HTML}{F8EBEA}
% 20%   | 241, 215, 213   | F1D7D5   | 0, 18, 13, 4
\definecolor{ETHRed20}{HTML}{F1D7D5}
% 40%   | 226, 174, 171   | E2AEAB   | 0, 36, 26, 8
\definecolor{ETHRed40}{HTML}{E2AEAB}
% 60%   | 212, 134, 129   | D48681   | 0, 54, 39, 11
\definecolor{ETHRed60}{HTML}{D48681}
% 80%   | 197, 93, 87     | C55D57   | (using HEX)
\definecolor{ETHRed80}{HTML}{C55D57}
% 120%  | 150, 39, 45     | 96272D   | 0, 100, 80, 40
\definecolor{ETHRed120}{HTML}{96272D}

% ==============================================================
% 6. ETH Purple Shades
% ==============================================================
% Shade | RGB             | HEX      | CMYK
% ------|-----------------|----------|---------------
% 10%   | 248, 232, 243   | F8E8F3   | 2, 10, 0, 1
\definecolor{ETHPurple10}{HTML}{F8E8F3}
% 20%   | 239, 208, 227   | EFD0E3   | 4, 20, 0, 1
\definecolor{ETHPurple20}{HTML}{EFD0E3}
% 40%   | 220, 158, 201   | DC9EC9   | 7, 40, 0, 4
\definecolor{ETHPurple40}{HTML}{DC9EC9}
% 60%   | 202, 108, 174   | CA6CAE   | 13, 60, 0, 6
\definecolor{ETHPurple60}{HTML}{CA6CAE}
% 80%   | 183, 59, 146    | B73B92   | 18, 80, 0, 8
\definecolor{ETHPurple80}{HTML}{B73B92}
% 120%  | 140, 10, 89     | 8C0A59   | 22, 100, 0, 35
\definecolor{ETHPurple120}{HTML}{8C0A59}

% ==============================================================
% 7. ETH Grey Shades
% ==============================================================
% Shade | RGB             | HEX      | CMYK
% ------|-----------------|----------|---------------
% 10%   | 241, 241, 241   | F1F1F1   | 0, 0, 0, 7
\definecolor{ETHGrey10}{HTML}{F1F1F1}
% 20%   | 226, 226, 226   | E2E2E2   | 0, 0, 0, 14
\definecolor{ETHGrey20}{HTML}{E2E2E2}
% 40%   | 197, 197, 197   | C5C5C5   | 0, 0, 0, 28
\definecolor{ETHGrey40}{HTML}{C5C5C5}
% 60%   | 169, 169, 169   | A9A9A9   | 0, 0, 0, 42
\definecolor{ETHGrey60}{HTML}{A9A9A9}
% 80%   | 140, 140, 140   | 8C8C8C   | 0, 0, 0, 56
\definecolor{ETHGrey80}{HTML}{8C8C8C}
% 120%  | 87, 87, 87      | 575757   | 0, 0, 0, 81
\definecolor{ETHGrey120}{HTML}{575757}
%   \textcolor{ETHBlue}{Hello ETH!}
% Usage in your main .tex:
%   \usepackage{xcolor} % or colortbl, etc.
%   % eth.tex
% Defines ETH brand colors based on:
% https://ethz.ch/staffnet/en/service/communication/corporate-design/colours.html
% 
% ------------------------------------------------------------------
% ETH Corporate Design – Primary Colors and Colour Shades Definitions
%
% PRIMARY ETH CORPORATE COLORS
%
% Colour       RGB             HEX       CMYK                 Pantone    RAL
% ----------------------------------------------------------------------------
% ETH Blue     33, 92, 175     #215CAF   100,57,0,0           2935       5005 Signalblau
% ETH Petrol   0, 120, 148     #007894   100,25,30,10         633        5009 Azurblau
% ETH Green    98, 115, 19     #627313   55,10,100,30         364        6010 Grasgrün
% ETH Bronze   142, 103, 19    #8E6713   30,36,100,25         4495       7008 Khakigrau
% ETH Red      183, 53, 45     #B7352D   0,90,80,17           1797       3031 Orientrot
% ETH Purple   167, 17, 122    #A7117A   22,100,0,10          234        4006 Verkehrspurpur
% ETH Grey     111, 111, 111   #6F6F6F   0,0,0,70             Cool Gray 11   7046 Telegrau 2
% ------------------------------------------------------------------
% ETH Corporate Design – Colour Shades Definitions
%
% 1. ETH Blue Shades
% 2. ETH Petrol Shades
% 3. ETH Green Shades
% 4. ETH Bronze Shades
% 5. ETH Red Shades
% 6. ETH Purple Shades
% 7. ETH Grey Shades
% 
% Last updated: 2025-03-08
%
% Usage:
%   % eth.tex
% Defines ETH brand colors based on:
% https://ethz.ch/staffnet/en/service/communication/corporate-design/colours.html
% 
% ------------------------------------------------------------------
% ETH Corporate Design – Primary Colors and Colour Shades Definitions
%
% PRIMARY ETH CORPORATE COLORS
%
% Colour       RGB             HEX       CMYK                 Pantone    RAL
% ----------------------------------------------------------------------------
% ETH Blue     33, 92, 175     #215CAF   100,57,0,0           2935       5005 Signalblau
% ETH Petrol   0, 120, 148     #007894   100,25,30,10         633        5009 Azurblau
% ETH Green    98, 115, 19     #627313   55,10,100,30         364        6010 Grasgrün
% ETH Bronze   142, 103, 19    #8E6713   30,36,100,25         4495       7008 Khakigrau
% ETH Red      183, 53, 45     #B7352D   0,90,80,17           1797       3031 Orientrot
% ETH Purple   167, 17, 122    #A7117A   22,100,0,10          234        4006 Verkehrspurpur
% ETH Grey     111, 111, 111   #6F6F6F   0,0,0,70             Cool Gray 11   7046 Telegrau 2
% ------------------------------------------------------------------
% ETH Corporate Design – Colour Shades Definitions
%
% 1. ETH Blue Shades
% 2. ETH Petrol Shades
% 3. ETH Green Shades
% 4. ETH Bronze Shades
% 5. ETH Red Shades
% 6. ETH Purple Shades
% 7. ETH Grey Shades
% 
% Last updated: 2025-03-08
%
% Usage:
%   \input{eth.tex}
%   \textcolor{ETHBlue}{Hello ETH!}
% Usage in your main .tex:
%   \usepackage{xcolor} % or colortbl, etc.
%   \input{eth.tex}
%   \textcolor{ETHBlue}{Hello from ETH!}
\NeedsTeXFormat{LaTeX2e}
\ProvidesFile{eth.tex}[2025/03/08 v1.0 ETH brand color definitions]

\RequirePackage{xcolor}

% ==============================================================
% Primary ETH Corporate Colors
% ==============================================================
\definecolor{ETHBlue}{HTML}{215CAF}    % ETH Blue: RGB: 33, 92, 175; CMYK: 100,57,0,0; Pantone: 2935; RAL: 5005 Signalblau
\definecolor{ETHPetrol}{HTML}{007894}   % ETH Petrol: RGB: 0,120,148; CMYK: 100,25,30,10; Pantone: 633; RAL: 5009 Azurblau
\definecolor{ETHGreen}{HTML}{627313}    % ETH Green: RGB: 98,115,19; CMYK: 55,10,100,30; Pantone: 364; RAL: 6010 Grasgrün
\definecolor{ETHBronze}{HTML}{8E6713}    % ETH Bronze: RGB: 142,103,19; CMYK: 30,36,100,25; Pantone: 4495; RAL: 7008 Khakigrau
\definecolor{ETHRed}{HTML}{B7352D}       % ETH Red: RGB: 183,53,45; CMYK: 0,90,80,17; Pantone: 1797; RAL: 3031 Orientrot
\definecolor{ETHPurple}{HTML}{A7117A}     % ETH Purple: RGB: 167,17,122; CMYK: 22,100,0,10; Pantone: 234; RAL: 4006 Verkehrspurpur
\definecolor{ETHGrey}{HTML}{6F6F6F}       % ETH Grey: RGB: 111,111,111; CMYK: 0,0,0,70; Pantone: Cool Gray 11; RAL: 7046 Telegrau 2


% ------------------------------------------------------------------
% Extended / Complementary Color Palette
% ------------------------------------------------------------------
\definecolor{ETHTeal}{HTML}{008C95}
\definecolor{ETHGreen}{HTML}{00B38B}
\definecolor{ETHDarkBlue}{HTML}{1D2447}
\definecolor{ETHLightBlue}{HTML}{5BB6D6}
\definecolor{ETHOrange}{HTML}{F39200}
\definecolor{ETHRed}{HTML}{C8002A}
\definecolor{ETHWarmGray}{HTML}{DAD7D2}
\definecolor{ETHBeige}{HTML}{D7CEC1}
\definecolor{ETHDarkBrown}{HTML}{7F4F3C}
\definecolor{ETHDarkPink}{HTML}{EB67BD}
\definecolor{ETHDarkPurple}{HTML}{5F2167}
\definecolor{ETHDarkMagenta}{HTML}{A3488E}
\definecolor{ETHDarkGray}{HTML}{333333}
\definecolor{ETHGray}{HTML}{75787B}
\definecolor{ETHLightGray}{HTML}{E2E2E2}
\definecolor{ETHWhite}{HTML}{FFFFFF}
\definecolor{ETHBlack}{HTML}{000000}

% ==============================================================
% 1. ETH Blue Shades
% ==============================================================
% Shade | RGB             | HEX      | CMYK
% ------|-----------------|----------|---------------
% 10%   | 233, 239, 247   | E9EFF7   | 10, 6, 0, 0
\definecolor{ETHBlue10}{HTML}{E9EFF7}
% 20%   | 211, 222, 239   | D3DEEF   | 20, 11, 0, 0
\definecolor{ETHBlue20}{HTML}{D3DEEF}
% 40%   | 166, 190, 223   | A6BEDF   | 40, 23, 0, 0
\definecolor{ETHBlue40}{HTML}{A6BEDF}
% 60%   | 122, 157, 207   | 7A9DCF   | 60, 34, 0, 0
\definecolor{ETHBlue60}{HTML}{7A9DCF}
% 80%   | 77, 125, 191    | 4D7DBF   | 80, 46, 0, 0
\definecolor{ETHBlue80}{HTML}{4D7DBF}
% 120%  | 8, 64, 126      | 08407E   | 100, 62, 0, 30
\definecolor{ETHBlue120}{HTML}{08407E}

% ==============================================================
% 2. ETH Petrol Shades
% ==============================================================
% Shade | RGB             | HEX      | CMYK
% ------|-----------------|----------|---------------
% 10%   | 231, 244, 247   | E7F4F7   | 12, 0, 5, 0
\definecolor{ETHPetrol10}{HTML}{E7F4F7}
% 20%   | 204, 228, 234   | CCE4EA   | 20, 3, 7, 0
\definecolor{ETHPetrol20}{HTML}{CCE4EA}
% 40%   | 153, 202, 213   | 99CAD5   | 40, 7, 12, 4
\definecolor{ETHPetrol40}{HTML}{99CAD5}
% 60%   | 102, 175, 192   | 66AFC0   | 60, 14, 18, 6
\definecolor{ETHPetrol60}{HTML}{66AFC0}
% 80%   | 51, 149, 171    | 3395AB   | 80, 20, 24, 8
\definecolor{ETHPetrol80}{HTML}{3395AB}
% 120%  | 0, 89, 109      | 00596D   | 100, 25, 30, 38
\definecolor{ETHPetrol120}{HTML}{00596D}

% ==============================================================
% 3. ETH Green Shades
% ==============================================================
% Shade | RGB             | HEX      | CMYK
% ------|-----------------|----------|---------------
% 10%   | 239, 241, 231   | EEF1E7   | 6, 1, 10, 3
\definecolor{ETHGreen10}{HTML}{EEF1E7}
% 20%   | 224, 227, 208   | E0E3D0   | 11, 2, 20, 6
\definecolor{ETHGreen20}{HTML}{E0E3D0}
% 40%   | 192, 199, 161   | C0C7A1   | 22, 4, 40, 12
\definecolor{ETHGreen40}{HTML}{C0C7A1}
% 60%   | 161, 171, 113   | A1AB71   | 33, 6, 60, 18
\definecolor{ETHGreen60}{HTML}{A1AB71}
% 80%   | 129, 143, 66    | 818F42   | 44, 8, 80, 24
\definecolor{ETHGreen80}{HTML}{818F42}
% 120%  | 54, 82, 19      | 365213   | 55, 10, 100, 65
\definecolor{ETHGreen120}{HTML}{365213}

% ==============================================================
% 4. ETH Bronze Shades
% ==============================================================
% Shade | RGB             | HEX      | CMYK
% ------|-----------------|----------|---------------
% 10%   | 244, 240, 231   | F4F0E7   | 3, 4, 10, 3
\definecolor{ETHBronze10}{HTML}{F4F0E7}
% 20%   | 232, 225, 208   | E8E1D0   | 6, 7, 20, 5
\definecolor{ETHBronze20}{HTML}{E8E1D0}
% 40%   | 210, 194, 161   | D2C2A1   | 12, 14, 40, 10
\definecolor{ETHBronze40}{HTML}{D2C2A1}
% 60%   | 187, 164, 113   | BBA471   | 18, 22, 60, 15
\definecolor{ETHBronze60}{HTML}{BBA471}
% 80%   | 165, 133, 66    | A58542   | 24, 29, 80, 20
\definecolor{ETHBronze80}{HTML}{A58542}
% 120%  | 112, 79, 18     | 704F12   | 30, 36, 100, 55
\definecolor{ETHBronze120}{HTML}{704F12}

% ==============================================================
% 5. ETH Red Shades
% ==============================================================
% Shade | RGB             | HEX      | CMYK
% ------|-----------------|----------|---------------
% 10%   | 248, 235, 234   | F8EBEA   | 0, 9, 6, 0
\definecolor{ETHRed10}{HTML}{F8EBEA}
% 20%   | 241, 215, 213   | F1D7D5   | 0, 18, 13, 4
\definecolor{ETHRed20}{HTML}{F1D7D5}
% 40%   | 226, 174, 171   | E2AEAB   | 0, 36, 26, 8
\definecolor{ETHRed40}{HTML}{E2AEAB}
% 60%   | 212, 134, 129   | D48681   | 0, 54, 39, 11
\definecolor{ETHRed60}{HTML}{D48681}
% 80%   | 197, 93, 87     | C55D57   | (using HEX)
\definecolor{ETHRed80}{HTML}{C55D57}
% 120%  | 150, 39, 45     | 96272D   | 0, 100, 80, 40
\definecolor{ETHRed120}{HTML}{96272D}

% ==============================================================
% 6. ETH Purple Shades
% ==============================================================
% Shade | RGB             | HEX      | CMYK
% ------|-----------------|----------|---------------
% 10%   | 248, 232, 243   | F8E8F3   | 2, 10, 0, 1
\definecolor{ETHPurple10}{HTML}{F8E8F3}
% 20%   | 239, 208, 227   | EFD0E3   | 4, 20, 0, 1
\definecolor{ETHPurple20}{HTML}{EFD0E3}
% 40%   | 220, 158, 201   | DC9EC9   | 7, 40, 0, 4
\definecolor{ETHPurple40}{HTML}{DC9EC9}
% 60%   | 202, 108, 174   | CA6CAE   | 13, 60, 0, 6
\definecolor{ETHPurple60}{HTML}{CA6CAE}
% 80%   | 183, 59, 146    | B73B92   | 18, 80, 0, 8
\definecolor{ETHPurple80}{HTML}{B73B92}
% 120%  | 140, 10, 89     | 8C0A59   | 22, 100, 0, 35
\definecolor{ETHPurple120}{HTML}{8C0A59}

% ==============================================================
% 7. ETH Grey Shades
% ==============================================================
% Shade | RGB             | HEX      | CMYK
% ------|-----------------|----------|---------------
% 10%   | 241, 241, 241   | F1F1F1   | 0, 0, 0, 7
\definecolor{ETHGrey10}{HTML}{F1F1F1}
% 20%   | 226, 226, 226   | E2E2E2   | 0, 0, 0, 14
\definecolor{ETHGrey20}{HTML}{E2E2E2}
% 40%   | 197, 197, 197   | C5C5C5   | 0, 0, 0, 28
\definecolor{ETHGrey40}{HTML}{C5C5C5}
% 60%   | 169, 169, 169   | A9A9A9   | 0, 0, 0, 42
\definecolor{ETHGrey60}{HTML}{A9A9A9}
% 80%   | 140, 140, 140   | 8C8C8C   | 0, 0, 0, 56
\definecolor{ETHGrey80}{HTML}{8C8C8C}
% 120%  | 87, 87, 87      | 575757   | 0, 0, 0, 81
\definecolor{ETHGrey120}{HTML}{575757}
%   \textcolor{ETHBlue}{Hello ETH!}
% Usage in your main .tex:
%   \usepackage{xcolor} % or colortbl, etc.
%   % eth.tex
% Defines ETH brand colors based on:
% https://ethz.ch/staffnet/en/service/communication/corporate-design/colours.html
% 
% ------------------------------------------------------------------
% ETH Corporate Design – Primary Colors and Colour Shades Definitions
%
% PRIMARY ETH CORPORATE COLORS
%
% Colour       RGB             HEX       CMYK                 Pantone    RAL
% ----------------------------------------------------------------------------
% ETH Blue     33, 92, 175     #215CAF   100,57,0,0           2935       5005 Signalblau
% ETH Petrol   0, 120, 148     #007894   100,25,30,10         633        5009 Azurblau
% ETH Green    98, 115, 19     #627313   55,10,100,30         364        6010 Grasgrün
% ETH Bronze   142, 103, 19    #8E6713   30,36,100,25         4495       7008 Khakigrau
% ETH Red      183, 53, 45     #B7352D   0,90,80,17           1797       3031 Orientrot
% ETH Purple   167, 17, 122    #A7117A   22,100,0,10          234        4006 Verkehrspurpur
% ETH Grey     111, 111, 111   #6F6F6F   0,0,0,70             Cool Gray 11   7046 Telegrau 2
% ------------------------------------------------------------------
% ETH Corporate Design – Colour Shades Definitions
%
% 1. ETH Blue Shades
% 2. ETH Petrol Shades
% 3. ETH Green Shades
% 4. ETH Bronze Shades
% 5. ETH Red Shades
% 6. ETH Purple Shades
% 7. ETH Grey Shades
% 
% Last updated: 2025-03-08
%
% Usage:
%   \input{eth.tex}
%   \textcolor{ETHBlue}{Hello ETH!}
% Usage in your main .tex:
%   \usepackage{xcolor} % or colortbl, etc.
%   \input{eth.tex}
%   \textcolor{ETHBlue}{Hello from ETH!}
\NeedsTeXFormat{LaTeX2e}
\ProvidesFile{eth.tex}[2025/03/08 v1.0 ETH brand color definitions]

\RequirePackage{xcolor}

% ==============================================================
% Primary ETH Corporate Colors
% ==============================================================
\definecolor{ETHBlue}{HTML}{215CAF}    % ETH Blue: RGB: 33, 92, 175; CMYK: 100,57,0,0; Pantone: 2935; RAL: 5005 Signalblau
\definecolor{ETHPetrol}{HTML}{007894}   % ETH Petrol: RGB: 0,120,148; CMYK: 100,25,30,10; Pantone: 633; RAL: 5009 Azurblau
\definecolor{ETHGreen}{HTML}{627313}    % ETH Green: RGB: 98,115,19; CMYK: 55,10,100,30; Pantone: 364; RAL: 6010 Grasgrün
\definecolor{ETHBronze}{HTML}{8E6713}    % ETH Bronze: RGB: 142,103,19; CMYK: 30,36,100,25; Pantone: 4495; RAL: 7008 Khakigrau
\definecolor{ETHRed}{HTML}{B7352D}       % ETH Red: RGB: 183,53,45; CMYK: 0,90,80,17; Pantone: 1797; RAL: 3031 Orientrot
\definecolor{ETHPurple}{HTML}{A7117A}     % ETH Purple: RGB: 167,17,122; CMYK: 22,100,0,10; Pantone: 234; RAL: 4006 Verkehrspurpur
\definecolor{ETHGrey}{HTML}{6F6F6F}       % ETH Grey: RGB: 111,111,111; CMYK: 0,0,0,70; Pantone: Cool Gray 11; RAL: 7046 Telegrau 2


% ------------------------------------------------------------------
% Extended / Complementary Color Palette
% ------------------------------------------------------------------
\definecolor{ETHTeal}{HTML}{008C95}
\definecolor{ETHGreen}{HTML}{00B38B}
\definecolor{ETHDarkBlue}{HTML}{1D2447}
\definecolor{ETHLightBlue}{HTML}{5BB6D6}
\definecolor{ETHOrange}{HTML}{F39200}
\definecolor{ETHRed}{HTML}{C8002A}
\definecolor{ETHWarmGray}{HTML}{DAD7D2}
\definecolor{ETHBeige}{HTML}{D7CEC1}
\definecolor{ETHDarkBrown}{HTML}{7F4F3C}
\definecolor{ETHDarkPink}{HTML}{EB67BD}
\definecolor{ETHDarkPurple}{HTML}{5F2167}
\definecolor{ETHDarkMagenta}{HTML}{A3488E}
\definecolor{ETHDarkGray}{HTML}{333333}
\definecolor{ETHGray}{HTML}{75787B}
\definecolor{ETHLightGray}{HTML}{E2E2E2}
\definecolor{ETHWhite}{HTML}{FFFFFF}
\definecolor{ETHBlack}{HTML}{000000}

% ==============================================================
% 1. ETH Blue Shades
% ==============================================================
% Shade | RGB             | HEX      | CMYK
% ------|-----------------|----------|---------------
% 10%   | 233, 239, 247   | E9EFF7   | 10, 6, 0, 0
\definecolor{ETHBlue10}{HTML}{E9EFF7}
% 20%   | 211, 222, 239   | D3DEEF   | 20, 11, 0, 0
\definecolor{ETHBlue20}{HTML}{D3DEEF}
% 40%   | 166, 190, 223   | A6BEDF   | 40, 23, 0, 0
\definecolor{ETHBlue40}{HTML}{A6BEDF}
% 60%   | 122, 157, 207   | 7A9DCF   | 60, 34, 0, 0
\definecolor{ETHBlue60}{HTML}{7A9DCF}
% 80%   | 77, 125, 191    | 4D7DBF   | 80, 46, 0, 0
\definecolor{ETHBlue80}{HTML}{4D7DBF}
% 120%  | 8, 64, 126      | 08407E   | 100, 62, 0, 30
\definecolor{ETHBlue120}{HTML}{08407E}

% ==============================================================
% 2. ETH Petrol Shades
% ==============================================================
% Shade | RGB             | HEX      | CMYK
% ------|-----------------|----------|---------------
% 10%   | 231, 244, 247   | E7F4F7   | 12, 0, 5, 0
\definecolor{ETHPetrol10}{HTML}{E7F4F7}
% 20%   | 204, 228, 234   | CCE4EA   | 20, 3, 7, 0
\definecolor{ETHPetrol20}{HTML}{CCE4EA}
% 40%   | 153, 202, 213   | 99CAD5   | 40, 7, 12, 4
\definecolor{ETHPetrol40}{HTML}{99CAD5}
% 60%   | 102, 175, 192   | 66AFC0   | 60, 14, 18, 6
\definecolor{ETHPetrol60}{HTML}{66AFC0}
% 80%   | 51, 149, 171    | 3395AB   | 80, 20, 24, 8
\definecolor{ETHPetrol80}{HTML}{3395AB}
% 120%  | 0, 89, 109      | 00596D   | 100, 25, 30, 38
\definecolor{ETHPetrol120}{HTML}{00596D}

% ==============================================================
% 3. ETH Green Shades
% ==============================================================
% Shade | RGB             | HEX      | CMYK
% ------|-----------------|----------|---------------
% 10%   | 239, 241, 231   | EEF1E7   | 6, 1, 10, 3
\definecolor{ETHGreen10}{HTML}{EEF1E7}
% 20%   | 224, 227, 208   | E0E3D0   | 11, 2, 20, 6
\definecolor{ETHGreen20}{HTML}{E0E3D0}
% 40%   | 192, 199, 161   | C0C7A1   | 22, 4, 40, 12
\definecolor{ETHGreen40}{HTML}{C0C7A1}
% 60%   | 161, 171, 113   | A1AB71   | 33, 6, 60, 18
\definecolor{ETHGreen60}{HTML}{A1AB71}
% 80%   | 129, 143, 66    | 818F42   | 44, 8, 80, 24
\definecolor{ETHGreen80}{HTML}{818F42}
% 120%  | 54, 82, 19      | 365213   | 55, 10, 100, 65
\definecolor{ETHGreen120}{HTML}{365213}

% ==============================================================
% 4. ETH Bronze Shades
% ==============================================================
% Shade | RGB             | HEX      | CMYK
% ------|-----------------|----------|---------------
% 10%   | 244, 240, 231   | F4F0E7   | 3, 4, 10, 3
\definecolor{ETHBronze10}{HTML}{F4F0E7}
% 20%   | 232, 225, 208   | E8E1D0   | 6, 7, 20, 5
\definecolor{ETHBronze20}{HTML}{E8E1D0}
% 40%   | 210, 194, 161   | D2C2A1   | 12, 14, 40, 10
\definecolor{ETHBronze40}{HTML}{D2C2A1}
% 60%   | 187, 164, 113   | BBA471   | 18, 22, 60, 15
\definecolor{ETHBronze60}{HTML}{BBA471}
% 80%   | 165, 133, 66    | A58542   | 24, 29, 80, 20
\definecolor{ETHBronze80}{HTML}{A58542}
% 120%  | 112, 79, 18     | 704F12   | 30, 36, 100, 55
\definecolor{ETHBronze120}{HTML}{704F12}

% ==============================================================
% 5. ETH Red Shades
% ==============================================================
% Shade | RGB             | HEX      | CMYK
% ------|-----------------|----------|---------------
% 10%   | 248, 235, 234   | F8EBEA   | 0, 9, 6, 0
\definecolor{ETHRed10}{HTML}{F8EBEA}
% 20%   | 241, 215, 213   | F1D7D5   | 0, 18, 13, 4
\definecolor{ETHRed20}{HTML}{F1D7D5}
% 40%   | 226, 174, 171   | E2AEAB   | 0, 36, 26, 8
\definecolor{ETHRed40}{HTML}{E2AEAB}
% 60%   | 212, 134, 129   | D48681   | 0, 54, 39, 11
\definecolor{ETHRed60}{HTML}{D48681}
% 80%   | 197, 93, 87     | C55D57   | (using HEX)
\definecolor{ETHRed80}{HTML}{C55D57}
% 120%  | 150, 39, 45     | 96272D   | 0, 100, 80, 40
\definecolor{ETHRed120}{HTML}{96272D}

% ==============================================================
% 6. ETH Purple Shades
% ==============================================================
% Shade | RGB             | HEX      | CMYK
% ------|-----------------|----------|---------------
% 10%   | 248, 232, 243   | F8E8F3   | 2, 10, 0, 1
\definecolor{ETHPurple10}{HTML}{F8E8F3}
% 20%   | 239, 208, 227   | EFD0E3   | 4, 20, 0, 1
\definecolor{ETHPurple20}{HTML}{EFD0E3}
% 40%   | 220, 158, 201   | DC9EC9   | 7, 40, 0, 4
\definecolor{ETHPurple40}{HTML}{DC9EC9}
% 60%   | 202, 108, 174   | CA6CAE   | 13, 60, 0, 6
\definecolor{ETHPurple60}{HTML}{CA6CAE}
% 80%   | 183, 59, 146    | B73B92   | 18, 80, 0, 8
\definecolor{ETHPurple80}{HTML}{B73B92}
% 120%  | 140, 10, 89     | 8C0A59   | 22, 100, 0, 35
\definecolor{ETHPurple120}{HTML}{8C0A59}

% ==============================================================
% 7. ETH Grey Shades
% ==============================================================
% Shade | RGB             | HEX      | CMYK
% ------|-----------------|----------|---------------
% 10%   | 241, 241, 241   | F1F1F1   | 0, 0, 0, 7
\definecolor{ETHGrey10}{HTML}{F1F1F1}
% 20%   | 226, 226, 226   | E2E2E2   | 0, 0, 0, 14
\definecolor{ETHGrey20}{HTML}{E2E2E2}
% 40%   | 197, 197, 197   | C5C5C5   | 0, 0, 0, 28
\definecolor{ETHGrey40}{HTML}{C5C5C5}
% 60%   | 169, 169, 169   | A9A9A9   | 0, 0, 0, 42
\definecolor{ETHGrey60}{HTML}{A9A9A9}
% 80%   | 140, 140, 140   | 8C8C8C   | 0, 0, 0, 56
\definecolor{ETHGrey80}{HTML}{8C8C8C}
% 120%  | 87, 87, 87      | 575757   | 0, 0, 0, 81
\definecolor{ETHGrey120}{HTML}{575757}
%   \textcolor{ETHBlue}{Hello from ETH!}
\NeedsTeXFormat{LaTeX2e}
\ProvidesFile{eth.tex}[2025/03/08 v1.0 ETH brand color definitions]

\RequirePackage{xcolor}

% ==============================================================
% Primary ETH Corporate Colors
% ==============================================================
\definecolor{ETHBlue}{HTML}{215CAF}    % ETH Blue: RGB: 33, 92, 175; CMYK: 100,57,0,0; Pantone: 2935; RAL: 5005 Signalblau
\definecolor{ETHPetrol}{HTML}{007894}   % ETH Petrol: RGB: 0,120,148; CMYK: 100,25,30,10; Pantone: 633; RAL: 5009 Azurblau
\definecolor{ETHGreen}{HTML}{627313}    % ETH Green: RGB: 98,115,19; CMYK: 55,10,100,30; Pantone: 364; RAL: 6010 Grasgrün
\definecolor{ETHBronze}{HTML}{8E6713}    % ETH Bronze: RGB: 142,103,19; CMYK: 30,36,100,25; Pantone: 4495; RAL: 7008 Khakigrau
\definecolor{ETHRed}{HTML}{B7352D}       % ETH Red: RGB: 183,53,45; CMYK: 0,90,80,17; Pantone: 1797; RAL: 3031 Orientrot
\definecolor{ETHPurple}{HTML}{A7117A}     % ETH Purple: RGB: 167,17,122; CMYK: 22,100,0,10; Pantone: 234; RAL: 4006 Verkehrspurpur
\definecolor{ETHGrey}{HTML}{6F6F6F}       % ETH Grey: RGB: 111,111,111; CMYK: 0,0,0,70; Pantone: Cool Gray 11; RAL: 7046 Telegrau 2


% ------------------------------------------------------------------
% Extended / Complementary Color Palette
% ------------------------------------------------------------------
\definecolor{ETHTeal}{HTML}{008C95}
\definecolor{ETHGreen}{HTML}{00B38B}
\definecolor{ETHDarkBlue}{HTML}{1D2447}
\definecolor{ETHLightBlue}{HTML}{5BB6D6}
\definecolor{ETHOrange}{HTML}{F39200}
\definecolor{ETHRed}{HTML}{C8002A}
\definecolor{ETHWarmGray}{HTML}{DAD7D2}
\definecolor{ETHBeige}{HTML}{D7CEC1}
\definecolor{ETHDarkBrown}{HTML}{7F4F3C}
\definecolor{ETHDarkPink}{HTML}{EB67BD}
\definecolor{ETHDarkPurple}{HTML}{5F2167}
\definecolor{ETHDarkMagenta}{HTML}{A3488E}
\definecolor{ETHDarkGray}{HTML}{333333}
\definecolor{ETHGray}{HTML}{75787B}
\definecolor{ETHLightGray}{HTML}{E2E2E2}
\definecolor{ETHWhite}{HTML}{FFFFFF}
\definecolor{ETHBlack}{HTML}{000000}

% ==============================================================
% 1. ETH Blue Shades
% ==============================================================
% Shade | RGB             | HEX      | CMYK
% ------|-----------------|----------|---------------
% 10%   | 233, 239, 247   | E9EFF7   | 10, 6, 0, 0
\definecolor{ETHBlue10}{HTML}{E9EFF7}
% 20%   | 211, 222, 239   | D3DEEF   | 20, 11, 0, 0
\definecolor{ETHBlue20}{HTML}{D3DEEF}
% 40%   | 166, 190, 223   | A6BEDF   | 40, 23, 0, 0
\definecolor{ETHBlue40}{HTML}{A6BEDF}
% 60%   | 122, 157, 207   | 7A9DCF   | 60, 34, 0, 0
\definecolor{ETHBlue60}{HTML}{7A9DCF}
% 80%   | 77, 125, 191    | 4D7DBF   | 80, 46, 0, 0
\definecolor{ETHBlue80}{HTML}{4D7DBF}
% 120%  | 8, 64, 126      | 08407E   | 100, 62, 0, 30
\definecolor{ETHBlue120}{HTML}{08407E}

% ==============================================================
% 2. ETH Petrol Shades
% ==============================================================
% Shade | RGB             | HEX      | CMYK
% ------|-----------------|----------|---------------
% 10%   | 231, 244, 247   | E7F4F7   | 12, 0, 5, 0
\definecolor{ETHPetrol10}{HTML}{E7F4F7}
% 20%   | 204, 228, 234   | CCE4EA   | 20, 3, 7, 0
\definecolor{ETHPetrol20}{HTML}{CCE4EA}
% 40%   | 153, 202, 213   | 99CAD5   | 40, 7, 12, 4
\definecolor{ETHPetrol40}{HTML}{99CAD5}
% 60%   | 102, 175, 192   | 66AFC0   | 60, 14, 18, 6
\definecolor{ETHPetrol60}{HTML}{66AFC0}
% 80%   | 51, 149, 171    | 3395AB   | 80, 20, 24, 8
\definecolor{ETHPetrol80}{HTML}{3395AB}
% 120%  | 0, 89, 109      | 00596D   | 100, 25, 30, 38
\definecolor{ETHPetrol120}{HTML}{00596D}

% ==============================================================
% 3. ETH Green Shades
% ==============================================================
% Shade | RGB             | HEX      | CMYK
% ------|-----------------|----------|---------------
% 10%   | 239, 241, 231   | EEF1E7   | 6, 1, 10, 3
\definecolor{ETHGreen10}{HTML}{EEF1E7}
% 20%   | 224, 227, 208   | E0E3D0   | 11, 2, 20, 6
\definecolor{ETHGreen20}{HTML}{E0E3D0}
% 40%   | 192, 199, 161   | C0C7A1   | 22, 4, 40, 12
\definecolor{ETHGreen40}{HTML}{C0C7A1}
% 60%   | 161, 171, 113   | A1AB71   | 33, 6, 60, 18
\definecolor{ETHGreen60}{HTML}{A1AB71}
% 80%   | 129, 143, 66    | 818F42   | 44, 8, 80, 24
\definecolor{ETHGreen80}{HTML}{818F42}
% 120%  | 54, 82, 19      | 365213   | 55, 10, 100, 65
\definecolor{ETHGreen120}{HTML}{365213}

% ==============================================================
% 4. ETH Bronze Shades
% ==============================================================
% Shade | RGB             | HEX      | CMYK
% ------|-----------------|----------|---------------
% 10%   | 244, 240, 231   | F4F0E7   | 3, 4, 10, 3
\definecolor{ETHBronze10}{HTML}{F4F0E7}
% 20%   | 232, 225, 208   | E8E1D0   | 6, 7, 20, 5
\definecolor{ETHBronze20}{HTML}{E8E1D0}
% 40%   | 210, 194, 161   | D2C2A1   | 12, 14, 40, 10
\definecolor{ETHBronze40}{HTML}{D2C2A1}
% 60%   | 187, 164, 113   | BBA471   | 18, 22, 60, 15
\definecolor{ETHBronze60}{HTML}{BBA471}
% 80%   | 165, 133, 66    | A58542   | 24, 29, 80, 20
\definecolor{ETHBronze80}{HTML}{A58542}
% 120%  | 112, 79, 18     | 704F12   | 30, 36, 100, 55
\definecolor{ETHBronze120}{HTML}{704F12}

% ==============================================================
% 5. ETH Red Shades
% ==============================================================
% Shade | RGB             | HEX      | CMYK
% ------|-----------------|----------|---------------
% 10%   | 248, 235, 234   | F8EBEA   | 0, 9, 6, 0
\definecolor{ETHRed10}{HTML}{F8EBEA}
% 20%   | 241, 215, 213   | F1D7D5   | 0, 18, 13, 4
\definecolor{ETHRed20}{HTML}{F1D7D5}
% 40%   | 226, 174, 171   | E2AEAB   | 0, 36, 26, 8
\definecolor{ETHRed40}{HTML}{E2AEAB}
% 60%   | 212, 134, 129   | D48681   | 0, 54, 39, 11
\definecolor{ETHRed60}{HTML}{D48681}
% 80%   | 197, 93, 87     | C55D57   | (using HEX)
\definecolor{ETHRed80}{HTML}{C55D57}
% 120%  | 150, 39, 45     | 96272D   | 0, 100, 80, 40
\definecolor{ETHRed120}{HTML}{96272D}

% ==============================================================
% 6. ETH Purple Shades
% ==============================================================
% Shade | RGB             | HEX      | CMYK
% ------|-----------------|----------|---------------
% 10%   | 248, 232, 243   | F8E8F3   | 2, 10, 0, 1
\definecolor{ETHPurple10}{HTML}{F8E8F3}
% 20%   | 239, 208, 227   | EFD0E3   | 4, 20, 0, 1
\definecolor{ETHPurple20}{HTML}{EFD0E3}
% 40%   | 220, 158, 201   | DC9EC9   | 7, 40, 0, 4
\definecolor{ETHPurple40}{HTML}{DC9EC9}
% 60%   | 202, 108, 174   | CA6CAE   | 13, 60, 0, 6
\definecolor{ETHPurple60}{HTML}{CA6CAE}
% 80%   | 183, 59, 146    | B73B92   | 18, 80, 0, 8
\definecolor{ETHPurple80}{HTML}{B73B92}
% 120%  | 140, 10, 89     | 8C0A59   | 22, 100, 0, 35
\definecolor{ETHPurple120}{HTML}{8C0A59}

% ==============================================================
% 7. ETH Grey Shades
% ==============================================================
% Shade | RGB             | HEX      | CMYK
% ------|-----------------|----------|---------------
% 10%   | 241, 241, 241   | F1F1F1   | 0, 0, 0, 7
\definecolor{ETHGrey10}{HTML}{F1F1F1}
% 20%   | 226, 226, 226   | E2E2E2   | 0, 0, 0, 14
\definecolor{ETHGrey20}{HTML}{E2E2E2}
% 40%   | 197, 197, 197   | C5C5C5   | 0, 0, 0, 28
\definecolor{ETHGrey40}{HTML}{C5C5C5}
% 60%   | 169, 169, 169   | A9A9A9   | 0, 0, 0, 42
\definecolor{ETHGrey60}{HTML}{A9A9A9}
% 80%   | 140, 140, 140   | 8C8C8C   | 0, 0, 0, 56
\definecolor{ETHGrey80}{HTML}{8C8C8C}
% 120%  | 87, 87, 87      | 575757   | 0, 0, 0, 81
\definecolor{ETHGrey120}{HTML}{575757}
%   \textcolor{ETHBlue}{Hello from ETH!}
\NeedsTeXFormat{LaTeX2e}
\ProvidesFile{eth.tex}[2025/03/08 v1.0 ETH brand color definitions]

\RequirePackage{xcolor}

% ==============================================================
% Primary ETH Corporate Colors
% ==============================================================
\definecolor{ETHBlue}{HTML}{215CAF}    % ETH Blue: RGB: 33, 92, 175; CMYK: 100,57,0,0; Pantone: 2935; RAL: 5005 Signalblau
\definecolor{ETHPetrol}{HTML}{007894}   % ETH Petrol: RGB: 0,120,148; CMYK: 100,25,30,10; Pantone: 633; RAL: 5009 Azurblau
\definecolor{ETHGreen}{HTML}{627313}    % ETH Green: RGB: 98,115,19; CMYK: 55,10,100,30; Pantone: 364; RAL: 6010 Grasgrün
\definecolor{ETHBronze}{HTML}{8E6713}    % ETH Bronze: RGB: 142,103,19; CMYK: 30,36,100,25; Pantone: 4495; RAL: 7008 Khakigrau
\definecolor{ETHRed}{HTML}{B7352D}       % ETH Red: RGB: 183,53,45; CMYK: 0,90,80,17; Pantone: 1797; RAL: 3031 Orientrot
\definecolor{ETHPurple}{HTML}{A7117A}     % ETH Purple: RGB: 167,17,122; CMYK: 22,100,0,10; Pantone: 234; RAL: 4006 Verkehrspurpur
\definecolor{ETHGrey}{HTML}{6F6F6F}       % ETH Grey: RGB: 111,111,111; CMYK: 0,0,0,70; Pantone: Cool Gray 11; RAL: 7046 Telegrau 2


% ------------------------------------------------------------------
% Extended / Complementary Color Palette
% ------------------------------------------------------------------
\definecolor{ETHTeal}{HTML}{008C95}
\definecolor{ETHGreen}{HTML}{00B38B}
\definecolor{ETHDarkBlue}{HTML}{1D2447}
\definecolor{ETHLightBlue}{HTML}{5BB6D6}
\definecolor{ETHOrange}{HTML}{F39200}
\definecolor{ETHRed}{HTML}{C8002A}
\definecolor{ETHWarmGray}{HTML}{DAD7D2}
\definecolor{ETHBeige}{HTML}{D7CEC1}
\definecolor{ETHDarkBrown}{HTML}{7F4F3C}
\definecolor{ETHDarkPink}{HTML}{EB67BD}
\definecolor{ETHDarkPurple}{HTML}{5F2167}
\definecolor{ETHDarkMagenta}{HTML}{A3488E}
\definecolor{ETHDarkGray}{HTML}{333333}
\definecolor{ETHGray}{HTML}{75787B}
\definecolor{ETHLightGray}{HTML}{E2E2E2}
\definecolor{ETHWhite}{HTML}{FFFFFF}
\definecolor{ETHBlack}{HTML}{000000}

% ==============================================================
% 1. ETH Blue Shades
% ==============================================================
% Shade | RGB             | HEX      | CMYK
% ------|-----------------|----------|---------------
% 10%   | 233, 239, 247   | E9EFF7   | 10, 6, 0, 0
\definecolor{ETHBlue10}{HTML}{E9EFF7}
% 20%   | 211, 222, 239   | D3DEEF   | 20, 11, 0, 0
\definecolor{ETHBlue20}{HTML}{D3DEEF}
% 40%   | 166, 190, 223   | A6BEDF   | 40, 23, 0, 0
\definecolor{ETHBlue40}{HTML}{A6BEDF}
% 60%   | 122, 157, 207   | 7A9DCF   | 60, 34, 0, 0
\definecolor{ETHBlue60}{HTML}{7A9DCF}
% 80%   | 77, 125, 191    | 4D7DBF   | 80, 46, 0, 0
\definecolor{ETHBlue80}{HTML}{4D7DBF}
% 120%  | 8, 64, 126      | 08407E   | 100, 62, 0, 30
\definecolor{ETHBlue120}{HTML}{08407E}

% ==============================================================
% 2. ETH Petrol Shades
% ==============================================================
% Shade | RGB             | HEX      | CMYK
% ------|-----------------|----------|---------------
% 10%   | 231, 244, 247   | E7F4F7   | 12, 0, 5, 0
\definecolor{ETHPetrol10}{HTML}{E7F4F7}
% 20%   | 204, 228, 234   | CCE4EA   | 20, 3, 7, 0
\definecolor{ETHPetrol20}{HTML}{CCE4EA}
% 40%   | 153, 202, 213   | 99CAD5   | 40, 7, 12, 4
\definecolor{ETHPetrol40}{HTML}{99CAD5}
% 60%   | 102, 175, 192   | 66AFC0   | 60, 14, 18, 6
\definecolor{ETHPetrol60}{HTML}{66AFC0}
% 80%   | 51, 149, 171    | 3395AB   | 80, 20, 24, 8
\definecolor{ETHPetrol80}{HTML}{3395AB}
% 120%  | 0, 89, 109      | 00596D   | 100, 25, 30, 38
\definecolor{ETHPetrol120}{HTML}{00596D}

% ==============================================================
% 3. ETH Green Shades
% ==============================================================
% Shade | RGB             | HEX      | CMYK
% ------|-----------------|----------|---------------
% 10%   | 239, 241, 231   | EEF1E7   | 6, 1, 10, 3
\definecolor{ETHGreen10}{HTML}{EEF1E7}
% 20%   | 224, 227, 208   | E0E3D0   | 11, 2, 20, 6
\definecolor{ETHGreen20}{HTML}{E0E3D0}
% 40%   | 192, 199, 161   | C0C7A1   | 22, 4, 40, 12
\definecolor{ETHGreen40}{HTML}{C0C7A1}
% 60%   | 161, 171, 113   | A1AB71   | 33, 6, 60, 18
\definecolor{ETHGreen60}{HTML}{A1AB71}
% 80%   | 129, 143, 66    | 818F42   | 44, 8, 80, 24
\definecolor{ETHGreen80}{HTML}{818F42}
% 120%  | 54, 82, 19      | 365213   | 55, 10, 100, 65
\definecolor{ETHGreen120}{HTML}{365213}

% ==============================================================
% 4. ETH Bronze Shades
% ==============================================================
% Shade | RGB             | HEX      | CMYK
% ------|-----------------|----------|---------------
% 10%   | 244, 240, 231   | F4F0E7   | 3, 4, 10, 3
\definecolor{ETHBronze10}{HTML}{F4F0E7}
% 20%   | 232, 225, 208   | E8E1D0   | 6, 7, 20, 5
\definecolor{ETHBronze20}{HTML}{E8E1D0}
% 40%   | 210, 194, 161   | D2C2A1   | 12, 14, 40, 10
\definecolor{ETHBronze40}{HTML}{D2C2A1}
% 60%   | 187, 164, 113   | BBA471   | 18, 22, 60, 15
\definecolor{ETHBronze60}{HTML}{BBA471}
% 80%   | 165, 133, 66    | A58542   | 24, 29, 80, 20
\definecolor{ETHBronze80}{HTML}{A58542}
% 120%  | 112, 79, 18     | 704F12   | 30, 36, 100, 55
\definecolor{ETHBronze120}{HTML}{704F12}

% ==============================================================
% 5. ETH Red Shades
% ==============================================================
% Shade | RGB             | HEX      | CMYK
% ------|-----------------|----------|---------------
% 10%   | 248, 235, 234   | F8EBEA   | 0, 9, 6, 0
\definecolor{ETHRed10}{HTML}{F8EBEA}
% 20%   | 241, 215, 213   | F1D7D5   | 0, 18, 13, 4
\definecolor{ETHRed20}{HTML}{F1D7D5}
% 40%   | 226, 174, 171   | E2AEAB   | 0, 36, 26, 8
\definecolor{ETHRed40}{HTML}{E2AEAB}
% 60%   | 212, 134, 129   | D48681   | 0, 54, 39, 11
\definecolor{ETHRed60}{HTML}{D48681}
% 80%   | 197, 93, 87     | C55D57   | (using HEX)
\definecolor{ETHRed80}{HTML}{C55D57}
% 120%  | 150, 39, 45     | 96272D   | 0, 100, 80, 40
\definecolor{ETHRed120}{HTML}{96272D}

% ==============================================================
% 6. ETH Purple Shades
% ==============================================================
% Shade | RGB             | HEX      | CMYK
% ------|-----------------|----------|---------------
% 10%   | 248, 232, 243   | F8E8F3   | 2, 10, 0, 1
\definecolor{ETHPurple10}{HTML}{F8E8F3}
% 20%   | 239, 208, 227   | EFD0E3   | 4, 20, 0, 1
\definecolor{ETHPurple20}{HTML}{EFD0E3}
% 40%   | 220, 158, 201   | DC9EC9   | 7, 40, 0, 4
\definecolor{ETHPurple40}{HTML}{DC9EC9}
% 60%   | 202, 108, 174   | CA6CAE   | 13, 60, 0, 6
\definecolor{ETHPurple60}{HTML}{CA6CAE}
% 80%   | 183, 59, 146    | B73B92   | 18, 80, 0, 8
\definecolor{ETHPurple80}{HTML}{B73B92}
% 120%  | 140, 10, 89     | 8C0A59   | 22, 100, 0, 35
\definecolor{ETHPurple120}{HTML}{8C0A59}

% ==============================================================
% 7. ETH Grey Shades
% ==============================================================
% Shade | RGB             | HEX      | CMYK
% ------|-----------------|----------|---------------
% 10%   | 241, 241, 241   | F1F1F1   | 0, 0, 0, 7
\definecolor{ETHGrey10}{HTML}{F1F1F1}
% 20%   | 226, 226, 226   | E2E2E2   | 0, 0, 0, 14
\definecolor{ETHGrey20}{HTML}{E2E2E2}
% 40%   | 197, 197, 197   | C5C5C5   | 0, 0, 0, 28
\definecolor{ETHGrey40}{HTML}{C5C5C5}
% 60%   | 169, 169, 169   | A9A9A9   | 0, 0, 0, 42
\definecolor{ETHGrey60}{HTML}{A9A9A9}
% 80%   | 140, 140, 140   | 8C8C8C   | 0, 0, 0, 56
\definecolor{ETHGrey80}{HTML}{8C8C8C}
% 120%  | 87, 87, 87      | 575757   | 0, 0, 0, 81
\definecolor{ETHGrey120}{HTML}{575757}  % Load ETH corporate colours and shade definitions

\colorlet{CTurl}{ETHBlue}      % Override CTcitation with ETHBlue
\colorlet{CTtitle}{ETHBlue}      % Override CTcitation with ETHBlue


% Additional general configurations (packages, macros, etc.) can be added below.


% Biblatex
% \usepackage[
%   style=nature,%
%   %style=science, article-title=true,%
%   natbib=true,%
%   clearlang=true,%
%   backend=biber,%
% ]{biblatex}

% Add this line to suppress the split bibliography warning
\BiblatexSplitbibDefernumbersWarningOff

% https://mirrors.ibiblio.org/CTAN/macros/latex/contrib/biblatex/doc/biblatex.pdf
\ExecuteBibliographyOptions{%
  %--- Backend --- --- ---
  bibwarn=true, %
  bibencoding=auto, % (ascii, inputenc, <encoding>)
  %--- Sorting --- --- ---
  sorting=none, % (bib, los) The sorting order of the list of shorthands =nty, ntd, nyt, ndt, nyvt, ndvt, anyt, andt, anyvt, optandvt, ynt, dnt, ydnt, ddnt, none, count, debug,
  % other options: 
  % nty        Sort by name, title, year.
  % nyt        Sort by name, year, title.
  % nyvt       Sort by name, year, volume, title.
  % anyt       Sort by alphabetic label, name, year, title.
  % anyvt      Sort by alphabetic label, name, year, volume, title.
  % ynt        Sort by year, name, title.
  % ydnt       Sort by year (descending), name, title.
  % none       Do not sort at all. All entries are processed in citation order.
  % debug      Sort by entry key. This is intended for debugging only.
  %
  sortcase=true,
  sortcites=true, % do/do not sort citations according to bib	
  %--- Dates --- --- ---
  date=comp,  % (short, long, terse, comp, iso8601)
  %	origdate=
  %	eventdate=
  %	urldate=
  %	alldates=
  datezeros=true, %
  dateabbrev=true, %
  %--- General Options --- --- ---
  maxnames=3,
  minnames=1,
  maxbibnames=100, % do not abbreviate names in bibliography
  %	autocite= % (plain, inline, footnote, superscript) 
  autopunct=true,
  language=auto,
  autolang=none, % (none, hyphen, other, other*)
  block=none, % (none, space, par, nbpar, ragged)
  notetype=foot+end, % (foot+end, footonly, endonly)
  hyperref=true, % (true, false, auto)
  backref=false,
  backrefstyle=three, % (none, three, two, two+, three+, all+)
  backrefsetstyle=setonly, %
  indexing=false, % 
  % options:
  % true       Enable indexing globally.
  % false      Disable indexing globally.
  % cite       Enable indexing in citations only.
  % bib        Enable indexing in the bibliography only.
  refsection=chapter, % (none, part, chapter, section, subsection)
  refsegment=none, % (none, part, chapter, section, subsection)
  abbreviate=true, % (true, false)
  defernumbers=false, % 
  punctfont=false, % 
  arxiv=abs, % (ps, pdf, format)	
  %--- Style Options --- --- ---	
  isbn=false,%
  url=false,%
  doi=false,%
  eprint=false,%	
}%	

% Suppress all date fields except the year
\AtEveryBibitem{%
  \clearfield{day}%
  \clearfield{month}%
  \clearfield{endday}%
  \clearfield{endmonth}%
}

\DeclareRedundantLanguages{en,EN,English}{english}

% Use only the first page number in a given range
\DeclareFieldFormat{pages}{\mkfirstpage{#1}}

% Footnote without number
% \newcommand\blfootnote[1]{%
%   \begingroup
%   \renewcommand\thefootnote{}\footnote{#1}%
%   \addtocounter{footnote}{-1}%
%   \endgroup
% }
% \newcommand\blfootnote[1]{%
%   \begingroup
%     \renewcommand\thefootnote{}%
%     \footnote{%
%         % \begingroup
%         % \linespread{1}\selectfont % or \begin{spacing}{1} ... \end{spacing}
%         % \vspace{0.5em}%    <-- Adds extra vertical space at the top of the footnote
%         % Set the vertical gap between the main text and the footnotes locally:
%         \setlength{\skip\footins}{10em}%
%         % \setlength{\skip\footins}{12pt plus 4pt minus 2pt}
%         \noindent %         <-- Removes the first-line indentation
%         \textit{#1}%       <-- Italicizes the footnote text
%         % \endgroup
%     }%
%     \addtocounter{footnote}{-1}%
%   \endgroup
% }
% Set the vertical gap between the main text and the footnotes
\setlength{\skip\footins}{4em plus 4em minus 2em} % sets the gap for all footnotes
% \setlength{\footnotemargin}{-.5em}%
\newcommand\blfootnote[1]{%
  \begingroup
    \renewcommand\thefootnote{}%
    \footnote{%
      \noindent
      \textit{#1}%
    }%
    \addtocounter{footnote}{-1}%
  \endgroup
}

\ExplSyntaxOn
% Define a function for string substitution
% \cs_new:Npn \minna_replace:nn #1 #2
%   {
%     \tl_replace_all:Nnn \l_tmpa_tl {#1} {#2}
%   }

% Wrapper macro for ease of use
\NewDocumentCommand{\ReplaceString}{ m m m }
  {
    \tl_set:Nn \l_tmpa_tl {#1}
    \minna_replace:nn {#2} {#3}
    \tl_use:N \l_tmpa_tl
  }
\ExplSyntaxOff



% ********************************************************************
% Fine-tune hyperreferences (hyperref should be called last)
% ********************************************************************

\usepackage[dvipsnames]{xcolor}


\PassOptionsToPackage{pdftex,hyperfootnotes=false,pdfpagelabels}{hyperref}
\usepackage{hyperref}  % backref linktocpage pagebackref
\pdfcompresslevel=9
\pdfadjustspacing=1

% \usepackage{hyperxmp}
%\pdfcompresslevel=9
%\pdfadjustspacing=1

% \usepackage{hyperref}  % backref linktocpage pagebackref

\hypersetup{%
  %draft, % hyperref's draft mode, for printing see below
  colorlinks=true, linktocpage=true, pdfstartpage=3, pdfstartview=FitV,%
  % uncomment the following line if you want to have black links (e.g., for printing)
  %colorlinks=false, linktocpage=false, pdfstartpage=3, pdfstartview=FitV, pdfborder={0 0 0},%
  breaklinks=true, pageanchor=true,%
  pdfpagemode=UseNone, %
  % pdfpagemode=UseOutlines,%
  plainpages=false, bookmarksnumbered, bookmarksopen=true, bookmarksopenlevel=1,%
  hypertexnames=true, pdfhighlight=/O,%nesting=true,%frenchlinks,%
  urlcolor=CTurl, linkcolor=CTlink, citecolor=CTcitation, %pagecolor=RoyalBlue,%
  %urlcolor=Black, linkcolor=Black, citecolor=Black, %pagecolor=Black,%
  pdftitle={\myTitle},%
  pdfauthor={\textcopyright\ \myName, \myUni, \myFaculty},%
  pdfsubject={},%
  pdfkeywords={},%
  pdfcreator={pdfLaTeX},%
  pdfproducer={LaTeX with hyperref and classicthesis}%
}


% ********************************************************************
% Setup autoreferences (hyperref and babel)
% ********************************************************************
% There are some issues regarding autorefnames
% http://www.tex.ac.uk/cgi-bin/texfaq2html?label=latexwords
% you have to redefine the macros for the
% language you use, e.g., american, ngerman
% (as chosen when loading babel/AtBeginDocument)
% ********************************************************************
 \makeatletter
 \@ifpackageloaded{babel}%
   {%
     \addto\extrasamerican{%
       \renewcommand*{\figureautorefname}{Figure}%
       \renewcommand*{\tableautorefname}{Table}%
       \renewcommand*{\partautorefname}{Part}%
       \renewcommand*{\chapterautorefname}{Chapter}%
       \renewcommand*{\sectionautorefname}{Section}%
       \renewcommand*{\subsectionautorefname}{Section}%
       \renewcommand*{\subsubsectionautorefname}{Section}%
     }%
     \addto\extrasngerman{%
       \renewcommand*{\paragraphautorefname}{Absatz}%
       \renewcommand*{\subparagraphautorefname}{Unterabsatz}%
       \renewcommand*{\footnoteautorefname}{Fu\"snote}%
       \renewcommand*{\FancyVerbLineautorefname}{Zeile}%
       \renewcommand*{\theoremautorefname}{Theorem}%
       \renewcommand*{\appendixautorefname}{Anhang}%
       \renewcommand*{\equationautorefname}{Gleichung}%
       \renewcommand*{\itemautorefname}{Punkt}%
     }%
       % Fix to getting autorefs for subfigures right (thanks to Belinda Vogt for changing the definition)
       \providecommand{\subfigureautorefname}{\figureautorefname}%
     }{\relax}
 \makeatother

% (Better) alternative to \autoref is \cref via the cleveref package
%\usepackage{cleveref}
%\crefformat{part}{Part #2\MakeUppercase{#1}#3}